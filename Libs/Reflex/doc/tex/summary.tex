\part{Summary}

\chapter{Overview}
\Reflex is the scripting language for \Rapture. \Rapture uses \Reflex scripts to customize the environment by giving developers the ability to control the behaviour of key parts of the system. \Reflex scripts are used in \Rapture operations, events, scheduled tasks and to control the filter/map behaviour of views. In addition \Reflex can be used externally to \Rapture as a scripting language for a \emph{client} of the system.

\Reflex is a simple, lightweight procedural language with very specific operators and syntax that show its close interation with the \Rapture environment. \Reflex is written to run on top of the Java Virtual Machine (JVM) and can also take advantage of all of the native capabilities (and libraries) associated with a JVM. In particular the \Rapture JVM can be extended with analytics libraries that can be invoked by \Reflex to perform more complex tasks.

This document takes the reader through the features and syntax of \Reflex through the use of examples. By the end of the document the reader should be able to construct and run their own scripts from within a \Rapture context.

