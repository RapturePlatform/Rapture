\chapter{Runner API}
\index{Runner API}

The document API for \Rapture is often abbreviated to \emph{Doc}. The API is used
to manipulate the presence and the content of document repositories in \Rapture.

In the abstract a document repository in \Rapture is a key/value store with optional
enhancements. The key in \Rapture corresponds to a URI for the document and where the
context is not obvious the scheme of the uri is \verb+document://+. In all document
API calls this scheme may be omitted.

Document repositories in \Rapture are backed by concrete data storage systems. When
you define a repository in \Rapture you provide a configuration string that is used
by \Rapture to route your request to a low level driver that interacts with the
underlying system. The format of this configuration string will be described in
the API call for creating a repository.

Document repositories can also be versioned. When you update a document in a
versioned repository the previous history of that document is preserved. In fact you can
qualify the URI of a document with the @ symbol and a version number to retrieve
previous versions of a document. Omitting the @ symbol will always retrieve the
latest version of a document.

Documents in repositories can also have metadata associated with them. \Rapture
automatically maintains some of this metadata - the time the document was created, the
user that created it. But a developer can use metadata update calls to add their
own attributes to documents in \Rapture.

The URI of a document in a repository implies a folder-like structure with the
forward slash delineating these folders. There are document API calls to treat a
document repository like a file system -- these are useful when constructing
browsable user interfaces to a repository.

\subsection{Methods}

\subsection{CreateServerGroup}
\index{CreateServerGroup}
\label{Api:CreateServerGroup}
\begin{verbatim}
   RaptureServerGroup createServerGroup (
           String    name
           String    description
   )
\end{verbatim}
%\begin{lstlisting}[language=reflex]
%ret = #runner.createServerGroup(name,description);
%\end{lstlisting}
\input{runner/createServerGroup}


\rule{15cm}{2pt}
\subsection{DeleteServerGroup}
\index{DeleteServerGroup}
\label{Api:DeleteServerGroup}
\begin{verbatim}
   void deleteServerGroup (
           String    name
   )
\end{verbatim}
%\begin{lstlisting}[language=reflex]
%ret = #runner.deleteServerGroup(name);
%\end{lstlisting}
\input{runner/deleteServerGroup}


\rule{15cm}{2pt}
\subsection{GetAllServerGroups}
\index{GetAllServerGroups}
\label{Api:GetAllServerGroups}
\begin{verbatim}
   List<RaptureServerGroup> getAllServerGroups (
   )
\end{verbatim}
%\begin{lstlisting}[language=reflex]
%ret = #runner.getAllServerGroups();
%\end{lstlisting}
\input{runner/getAllServerGroups}


\rule{15cm}{2pt}
\subsection{GetAllApplicationDefinitions}
\index{GetAllApplicationDefinitions}
\label{Api:GetAllApplicationDefinitions}
\begin{verbatim}
   List<RaptureApplicationDefinition> getAllApplicationDefinitions (
   )
\end{verbatim}
%\begin{lstlisting}[language=reflex]
%ret = #runner.getAllApplicationDefinitions();
%\end{lstlisting}
\input{runner/getAllApplicationDefinitions}


\rule{15cm}{2pt}
\subsection{GetAllLibraryDefinitions}
\index{GetAllLibraryDefinitions}
\label{Api:GetAllLibraryDefinitions}
\begin{verbatim}
   List<RaptureLibraryDefinition> getAllLibraryDefinitions (
   )
\end{verbatim}
%\begin{lstlisting}[language=reflex]
%ret = #runner.getAllLibraryDefinitions();
%\end{lstlisting}
\input{runner/getAllLibraryDefinitions}


\rule{15cm}{2pt}
\subsection{GetAllApplicationInstances}
\index{GetAllApplicationInstances}
\label{Api:GetAllApplicationInstances}
\begin{verbatim}
   List<RaptureApplicationInstance> getAllApplicationInstances (
   )
\end{verbatim}
%\begin{lstlisting}[language=reflex]
%ret = #runner.getAllApplicationInstances();
%\end{lstlisting}
\input{runner/getAllApplicationInstances}


\rule{15cm}{2pt}
\subsection{GetServerGroup}
\index{GetServerGroup}
\label{Api:GetServerGroup}
\begin{verbatim}
   RaptureServerGroup getServerGroup (
           String    name
   )
\end{verbatim}
%\begin{lstlisting}[language=reflex]
%ret = #runner.getServerGroup(name);
%\end{lstlisting}
\input{runner/getServerGroup}


\rule{15cm}{2pt}
\subsection{AddGroupInclusion}
\index{AddGroupInclusion}
\label{Api:AddGroupInclusion}
\begin{verbatim}
   RaptureServerGroup addGroupInclusion (
           String    name
           String    inclusion
   )
\end{verbatim}
%\begin{lstlisting}[language=reflex]
%ret = #runner.addGroupInclusion(name,inclusion);
%\end{lstlisting}
\input{runner/addGroupInclusion}


\rule{15cm}{2pt}
\subsection{RemoveGroupInclusion}
\index{RemoveGroupInclusion}
\label{Api:RemoveGroupInclusion}
\begin{verbatim}
   RaptureServerGroup removeGroupInclusion (
           String    name
           String    inclusion
   )
\end{verbatim}
%\begin{lstlisting}[language=reflex]
%ret = #runner.removeGroupInclusion(name,inclusion);
%\end{lstlisting}
\input{runner/removeGroupInclusion}


\rule{15cm}{2pt}
\subsection{AddGroupExclusion}
\index{AddGroupExclusion}
\label{Api:AddGroupExclusion}
\begin{verbatim}
   RaptureServerGroup addGroupExclusion (
           String    name
           String    exclusion
   )
\end{verbatim}
%\begin{lstlisting}[language=reflex]
%ret = #runner.addGroupExclusion(name,exclusion);
%\end{lstlisting}
\input{runner/addGroupExclusion}


\rule{15cm}{2pt}
\subsection{RemoveGroupExclusion}
\index{RemoveGroupExclusion}
\label{Api:RemoveGroupExclusion}
\begin{verbatim}
   RaptureServerGroup removeGroupExclusion (
           String    name
           String    exclusion
   )
\end{verbatim}
%\begin{lstlisting}[language=reflex]
%ret = #runner.removeGroupExclusion(name,exclusion);
%\end{lstlisting}
\input{runner/removeGroupExclusion}


\rule{15cm}{2pt}
\subsection{RemoveGroupEntry}
\index{RemoveGroupEntry}
\label{Api:RemoveGroupEntry}
\begin{verbatim}
   RaptureServerGroup removeGroupEntry (
           String    name
           String    entry
   )
\end{verbatim}
%\begin{lstlisting}[language=reflex]
%ret = #runner.removeGroupEntry(name,entry);
%\end{lstlisting}
\input{runner/removeGroupEntry}


\rule{15cm}{2pt}
\subsection{CreateApplicationDefinition}
\index{CreateApplicationDefinition}
\label{Api:CreateApplicationDefinition}
\begin{verbatim}
   RaptureApplicationDefinition createApplicationDefinition (
           String    name
           String    ver
           String    description
   )
\end{verbatim}
%\begin{lstlisting}[language=reflex]
%ret = #runner.createApplicationDefinition(name,ver,description);
%\end{lstlisting}
\input{runner/createApplicationDefinition}


\rule{15cm}{2pt}
\subsection{DeleteApplicationDefinition}
\index{DeleteApplicationDefinition}
\label{Api:DeleteApplicationDefinition}
\begin{verbatim}
   void deleteApplicationDefinition (
           String    name
   )
\end{verbatim}
%\begin{lstlisting}[language=reflex]
%ret = #runner.deleteApplicationDefinition(name);
%\end{lstlisting}
\input{runner/deleteApplicationDefinition}


\rule{15cm}{2pt}
\subsection{UpdateApplicationVersion}
\index{UpdateApplicationVersion}
\label{Api:UpdateApplicationVersion}
\begin{verbatim}
   RaptureApplicationDefinition updateApplicationVersion (
           String    name
           String    ver
   )
\end{verbatim}
%\begin{lstlisting}[language=reflex]
%ret = #runner.updateApplicationVersion(name,ver);
%\end{lstlisting}
\input{runner/updateApplicationVersion}


\rule{15cm}{2pt}
\subsection{CreateLibraryDefinition}
\index{CreateLibraryDefinition}
\label{Api:CreateLibraryDefinition}
\begin{verbatim}
   RaptureLibraryDefinition createLibraryDefinition (
           String    name
           String    ver
           String    description
   )
\end{verbatim}
%\begin{lstlisting}[language=reflex]
%ret = #runner.createLibraryDefinition(name,ver,description);
%\end{lstlisting}
\input{runner/createLibraryDefinition}


\rule{15cm}{2pt}
\subsection{DeleteLibraryDefinition}
\index{DeleteLibraryDefinition}
\label{Api:DeleteLibraryDefinition}
\begin{verbatim}
   void deleteLibraryDefinition (
           String    name
   )
\end{verbatim}
%\begin{lstlisting}[language=reflex]
%ret = #runner.deleteLibraryDefinition(name);
%\end{lstlisting}
\input{runner/deleteLibraryDefinition}


\rule{15cm}{2pt}
\subsection{GetLibraryDefinition}
\index{GetLibraryDefinition}
\label{Api:GetLibraryDefinition}
\begin{verbatim}
   RaptureLibraryDefinition getLibraryDefinition (
           String    name
   )
\end{verbatim}
%\begin{lstlisting}[language=reflex]
%ret = #runner.getLibraryDefinition(name);
%\end{lstlisting}
\input{runner/getLibraryDefinition}


\rule{15cm}{2pt}
\subsection{UpdateLibraryVersion}
\index{UpdateLibraryVersion}
\label{Api:UpdateLibraryVersion}
\begin{verbatim}
   RaptureLibraryDefinition updateLibraryVersion (
           String    name
           String    ver
   )
\end{verbatim}
%\begin{lstlisting}[language=reflex]
%ret = #runner.updateLibraryVersion(name,ver);
%\end{lstlisting}
\input{runner/updateLibraryVersion}


\rule{15cm}{2pt}
\subsection{AddLibraryToGroup}
\index{AddLibraryToGroup}
\label{Api:AddLibraryToGroup}
\begin{verbatim}
   RaptureServerGroup addLibraryToGroup (
           String    serverGroup
           String    libraryName
   )
\end{verbatim}
%\begin{lstlisting}[language=reflex]
%ret = #runner.addLibraryToGroup(serverGroup,libraryName);
%\end{lstlisting}
\input{runner/addLibraryToGroup}


\rule{15cm}{2pt}
\subsection{RemoveLibraryFromGroup}
\index{RemoveLibraryFromGroup}
\label{Api:RemoveLibraryFromGroup}
\begin{verbatim}
   RaptureServerGroup removeLibraryFromGroup (
           String    serverGroup
           String    libraryName
   )
\end{verbatim}
%\begin{lstlisting}[language=reflex]
%ret = #runner.removeLibraryFromGroup(serverGroup,libraryName);
%\end{lstlisting}
\input{runner/removeLibraryFromGroup}


\rule{15cm}{2pt}
\subsection{CreateApplicationInstance}
\index{CreateApplicationInstance}
\label{Api:CreateApplicationInstance}
\begin{verbatim}
   RaptureApplicationInstance createApplicationInstance (
           String    name
           String    description
           String    serverGroup
           String    appName
           String    timeRange
           int    retryCount
           String    parameters
           String    apiUser
   )
\end{verbatim}
%\begin{lstlisting}[language=reflex]
%ret = #runner.createApplicationInstance(name,description,serverGroup,appName,timeRange,retryCount,parameters,apiUser);
%\end{lstlisting}
\input{runner/createApplicationInstance}


\rule{15cm}{2pt}
\subsection{RunApplication}
\index{RunApplication}
\label{Api:RunApplication}
\begin{verbatim}
   RaptureApplicationStatus runApplication (
           String    appName
           String    queueName
           Map<String,String>    parameterInput
           Map<String,String>    parameterOutput
   )
\end{verbatim}
%\begin{lstlisting}[language=reflex]
%ret = #runner.runApplication(appName,queueName,parameterInput,parameterOutput);
%\end{lstlisting}
\input{runner/runApplication}


\rule{15cm}{2pt}
\subsection{RunCustomApplication}
\index{RunCustomApplication}
\label{Api:RunCustomApplication}
\begin{verbatim}
   RaptureApplicationStatus runCustomApplication (
           String    appName
           String    queueName
           Map<String,String>    parameterInput
           Map<String,String>    parameterOutput
           String    customApplicationPath
   )
\end{verbatim}
%\begin{lstlisting}[language=reflex]
%ret = #runner.runCustomApplication(appName,queueName,parameterInput,parameterOutput,customApplicationPath);
%\end{lstlisting}
\input{runner/runCustomApplication}


\rule{15cm}{2pt}
\subsection{GetApplicationStatus}
\index{GetApplicationStatus}
\label{Api:GetApplicationStatus}
\begin{verbatim}
   RaptureApplicationStatus getApplicationStatus (
           String    applicationStatusURI
   )
\end{verbatim}
%\begin{lstlisting}[language=reflex]
%ret = #runner.getApplicationStatus(applicationStatusURI);
%\end{lstlisting}
\input{runner/getApplicationStatus}


\rule{15cm}{2pt}
\subsection{GetApplicationStatuses}
\index{GetApplicationStatuses}
\label{Api:GetApplicationStatuses}
\begin{verbatim}
   List<RaptureApplicationStatus> getApplicationStatuses (
           String    date
   )
\end{verbatim}
%\begin{lstlisting}[language=reflex]
%ret = #runner.getApplicationStatuses(date);
%\end{lstlisting}
\input{runner/getApplicationStatuses}


\rule{15cm}{2pt}
\subsection{GetApplicationStatusDates}
\index{GetApplicationStatusDates}
\label{Api:GetApplicationStatusDates}
\begin{verbatim}
   List<String> getApplicationStatusDates (
   )
\end{verbatim}
%\begin{lstlisting}[language=reflex]
%ret = #runner.getApplicationStatusDates();
%\end{lstlisting}
\input{runner/getApplicationStatusDates}


\rule{15cm}{2pt}
\subsection{ArchiveApplicationStatuses}
\index{ArchiveApplicationStatuses}
\label{Api:ArchiveApplicationStatuses}
\begin{verbatim}
   void archiveApplicationStatuses (
   )
\end{verbatim}
%\begin{lstlisting}[language=reflex]
%ret = #runner.archiveApplicationStatuses();
%\end{lstlisting}
\input{runner/archiveApplicationStatuses}


\rule{15cm}{2pt}
\subsection{ChangeApplicationStatus}
\index{ChangeApplicationStatus}
\label{Api:ChangeApplicationStatus}
\begin{verbatim}
   RaptureApplicationStatus changeApplicationStatus (
           String    applicationStatusURI
           RaptureApplicationStatusStep    statusCode
           String    message
   )
\end{verbatim}
%\begin{lstlisting}[language=reflex]
%ret = #runner.changeApplicationStatus(applicationStatusURI,statusCode,message);
%\end{lstlisting}
\input{runner/changeApplicationStatus}


\rule{15cm}{2pt}
\subsection{RecordStatusMessages}
\index{RecordStatusMessages}
\label{Api:RecordStatusMessages}
\begin{verbatim}
   void recordStatusMessages (
           String    applicationStatusURI
           List<String>    messages
   )
\end{verbatim}
%\begin{lstlisting}[language=reflex]
%ret = #runner.recordStatusMessages(applicationStatusURI,messages);
%\end{lstlisting}
\input{runner/recordStatusMessages}


\rule{15cm}{2pt}
\subsection{TerminateApplication}
\index{TerminateApplication}
\label{Api:TerminateApplication}
\begin{verbatim}
   RaptureApplicationStatus terminateApplication (
           String    applicationStatusURI
           String    reasonMessage
   )
\end{verbatim}
%\begin{lstlisting}[language=reflex]
%ret = #runner.terminateApplication(applicationStatusURI,reasonMessage);
%\end{lstlisting}
\input{runner/terminateApplication}


\rule{15cm}{2pt}
\subsection{DeleteApplicationInstance}
\index{DeleteApplicationInstance}
\label{Api:DeleteApplicationInstance}
\begin{verbatim}
   void deleteApplicationInstance (
           String    name
           String    serverGroup
   )
\end{verbatim}
%\begin{lstlisting}[language=reflex]
%ret = #runner.deleteApplicationInstance(name,serverGroup);
%\end{lstlisting}
\input{runner/deleteApplicationInstance}


\rule{15cm}{2pt}
\subsection{GetApplicationInstance}
\index{GetApplicationInstance}
\label{Api:GetApplicationInstance}
\begin{verbatim}
   RaptureApplicationInstance getApplicationInstance (
           String    name
           String    serverGroup
   )
\end{verbatim}
%\begin{lstlisting}[language=reflex]
%ret = #runner.getApplicationInstance(name,serverGroup);
%\end{lstlisting}
\input{runner/getApplicationInstance}


\rule{15cm}{2pt}
\subsection{UpdateStatus}
\index{UpdateStatus}
\label{Api:UpdateStatus}
\begin{verbatim}
   void updateStatus (
           String    name
           String    serverGroup
           String    myServer
           String    status
           boolean    finished
   )
\end{verbatim}
%\begin{lstlisting}[language=reflex]
%ret = #runner.updateStatus(name,serverGroup,myServer,status,finished);
%\end{lstlisting}
\input{runner/updateStatus}


\rule{15cm}{2pt}
\subsection{GetApplicationsForServerGroup}
\index{GetApplicationsForServerGroup}
\label{Api:GetApplicationsForServerGroup}
\begin{verbatim}
   List<String> getApplicationsForServerGroup (
           String    serverGroup
   )
\end{verbatim}
%\begin{lstlisting}[language=reflex]
%ret = #runner.getApplicationsForServerGroup(serverGroup);
%\end{lstlisting}
\input{runner/getApplicationsForServerGroup}


\rule{15cm}{2pt}
\subsection{GetApplicationsForServer}
\index{GetApplicationsForServer}
\label{Api:GetApplicationsForServer}
\begin{verbatim}
   List<RaptureApplicationInstance> getApplicationsForServer (
           String    serverName
   )
\end{verbatim}
%\begin{lstlisting}[language=reflex]
%ret = #runner.getApplicationsForServer(serverName);
%\end{lstlisting}
\input{runner/getApplicationsForServer}


\rule{15cm}{2pt}
\subsection{GetApplicationDefinition}
\index{GetApplicationDefinition}
\label{Api:GetApplicationDefinition}
\begin{verbatim}
   RaptureApplicationDefinition getApplicationDefinition (
           String    name
   )
\end{verbatim}
%\begin{lstlisting}[language=reflex]
%ret = #runner.getApplicationDefinition(name);
%\end{lstlisting}
\input{runner/getApplicationDefinition}


\rule{15cm}{2pt}
\subsection{SetRunnerConfig}
\index{SetRunnerConfig}
\label{Api:SetRunnerConfig}
\begin{verbatim}
   void setRunnerConfig (
           String    name
           String    value
   )
\end{verbatim}
%\begin{lstlisting}[language=reflex]
%ret = #runner.setRunnerConfig(name,value);
%\end{lstlisting}
\input{runner/setRunnerConfig}


\rule{15cm}{2pt}
\subsection{DeleteRunnerConfig}
\index{DeleteRunnerConfig}
\label{Api:DeleteRunnerConfig}
\begin{verbatim}
   void deleteRunnerConfig (
           String    name
   )
\end{verbatim}
%\begin{lstlisting}[language=reflex]
%ret = #runner.deleteRunnerConfig(name);
%\end{lstlisting}
\input{runner/deleteRunnerConfig}


\rule{15cm}{2pt}
\subsection{GetRunnerConfig}
\index{GetRunnerConfig}
\label{Api:GetRunnerConfig}
\begin{verbatim}
   RaptureRunnerConfig getRunnerConfig (
   )
\end{verbatim}
%\begin{lstlisting}[language=reflex]
%ret = #runner.getRunnerConfig();
%\end{lstlisting}
\input{runner/getRunnerConfig}


\rule{15cm}{2pt}
\subsection{RecordRunnerStatus}
\index{RecordRunnerStatus}
\label{Api:RecordRunnerStatus}
\begin{verbatim}
   void recordRunnerStatus (
           String    serverName
           String    serverGroup
           String    appInstance
           String    appName
           String    status
   )
\end{verbatim}
%\begin{lstlisting}[language=reflex]
%ret = #runner.recordRunnerStatus(serverName,serverGroup,appInstance,appName,status);
%\end{lstlisting}
\input{runner/recordRunnerStatus}


\rule{15cm}{2pt}
\subsection{RecordInstanceCapabilities}
\index{RecordInstanceCapabilities}
\label{Api:RecordInstanceCapabilities}
\begin{verbatim}
   void recordInstanceCapabilities (
           String    serverName
           String    instanceName
           Map<String,Object>    capabilities
   )
\end{verbatim}
%\begin{lstlisting}[language=reflex]
%ret = #runner.recordInstanceCapabilities(serverName,instanceName,capabilities);
%\end{lstlisting}
\input{runner/recordInstanceCapabilities}


\rule{15cm}{2pt}
\subsection{GetCapabilities}
\index{GetCapabilities}
\label{Api:GetCapabilities}
\begin{verbatim}
   Map<String,RaptureInstanceCapabilities> getCapabilities (
           String    serverName
           List<String>    instanceNames
   )
\end{verbatim}
%\begin{lstlisting}[language=reflex]
%ret = #runner.getCapabilities(serverName,instanceNames);
%\end{lstlisting}
\input{runner/getCapabilities}


\rule{15cm}{2pt}
\subsection{GetRunnerServers}
\index{GetRunnerServers}
\label{Api:GetRunnerServers}
\begin{verbatim}
   List<String> getRunnerServers (
   )
\end{verbatim}
%\begin{lstlisting}[language=reflex]
%ret = #runner.getRunnerServers();
%\end{lstlisting}
\input{runner/getRunnerServers}


\rule{15cm}{2pt}
\subsection{GetRunnerStatus}
\index{GetRunnerStatus}
\label{Api:GetRunnerStatus}
\begin{verbatim}
   RaptureRunnerStatus getRunnerStatus (
           String    serverName
   )
\end{verbatim}
%\begin{lstlisting}[language=reflex]
%ret = #runner.getRunnerStatus(serverName);
%\end{lstlisting}
\input{runner/getRunnerStatus}


\rule{15cm}{2pt}
\subsection{CleanRunnerStatus}
\index{CleanRunnerStatus}
\label{Api:CleanRunnerStatus}
\begin{verbatim}
   void cleanRunnerStatus (
           int    ageInMinutes
   )
\end{verbatim}
%\begin{lstlisting}[language=reflex]
%ret = #runner.cleanRunnerStatus(ageInMinutes);
%\end{lstlisting}
\input{runner/cleanRunnerStatus}


\rule{15cm}{2pt}
\subsection{MarkForRestart}
\index{MarkForRestart}
\label{Api:MarkForRestart}
\begin{verbatim}
   void markForRestart (
           String    serverName
           String    name
   )
\end{verbatim}
%\begin{lstlisting}[language=reflex]
%ret = #runner.markForRestart(serverName,name);
%\end{lstlisting}
\input{runner/markForRestart}


\rule{15cm}{2pt}
