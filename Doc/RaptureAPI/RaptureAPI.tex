\documentclass[12pt,twoside,a4paper]{book}

\renewcommand*\sfdefault{phv}
\renewcommand{\familydefault}{\sfdefault}

%\usepackage{arev}
\usepackage[scaled]{helvet}
\usepackage[T1]{fontenc}

\setcounter{secnumdepth}{0}

%\newcommand{\Rapture}{${\mathbf{Rapture}}$~}
%\newcommand{\Reflex}{${\mathbf{Reflex}}$~}

\newcommand{\Rapture}{Rapture~}
\newcommand{\Reflex}{Reflex~}

\usepackage{listings}
\usepackage[usenames,dvipsnames,svgnames,table]{xcolor}
\usepackage{graphicx}
\usepackage{hyperref}
\usepackage{varioref}

\newcommand{\myTitle}{ RaptureAPI\xspace}
\newcommand{\myClient}{\xspace}
\newcommand{\myName}{Alan Moore\xspace}
\newcommand{\myTime}{February 12, 2016\xspace}
\newcommand{\myCFootnote}{Documentation By\xspace}
\newcommand{\myCompany}{Incapture\xspace}
\newcommand{\myCompanyFull}{Incapture Technologies LLC\xspace}
\newcommand{\myCompanyAddress}{600 Montgomery Street\\San Francisco \\ CA 94111\xspace}

\parskip 5pt

\makeindex

\lstdefinelanguage{reflex}
{
  morekeywords={def,end,for,while,if,else,do,const,println,fromjson,json},
  morecomment=[l]{//},
  morestring=[b]"
}

\definecolor{mygray}{rgb}{0.95,0.95,0.95}
\lstset{basicstyle=\footnotesize\ttfamily,breaklines=true}

\lstset{ %
  aboveskip=7pt,
  belowskip=7pt,
  numberbychapter=false,
  %language=reflex,                % the language of the code
  %basicstyle=\small,           % the size of the fonts that are used for the code
  %numbers=left,                   % where to put the line-numbers
  backgroundcolor=\color{mygray},
  numberstyle=\tiny\color{gray},  % the style that is used for the line-numbers
  keywordstyle=\color{blue},
  stringstyle=\color{red},
  stepnumber=1,                   % the step between two line-numbers. If it's 1, each line
  numbersep=7pt,                  % how far the line-numbers are from the code
  columns=fixed,
  showspaces=false,               % show spaces adding particular underscores
  showstringspaces=false,         % underline spaces within strings
  showtabs=false,                 % show tabs within strings adding particular underscores
  %frame=TB,                   % adds a frame around the code
  rulecolor=\color{black},        % if not set, the frame-color may be changed on line-breaks within not-black text (e.g. commens (green here))
  tabsize=2,                      % sets default tabsize to 2 spaces
  captionpos=b,                   % sets the caption-position to bottom
  breaklines=true,                % sets automatic line breaking
  breakatwhitespace=false,        % sets if automatic breaks should only happen at whitespace
  escapeinside={\%*}{*)},            % if you want to add LaTeX within your code
  morekeywords={*,...}               % if you want to add more keywords to the set
}


\begin{document}

\title{The Rapture API}
\author{Alan Moore}
\date{February 2016}

\makeatletter
    \begin{titlepage}
      \includegraphics[width=0.7\linewidth]{Graphics/RaptureLogo.png}\\[4ex]
        \begin{center}
            {\huge \bfseries  \@title }\\[2ex]
            {\LARGE  \@author}\\[50ex]
            {\large \@date}
        \end{center}
    \end{titlepage}
\makeatother
\thispagestyle{empty}
\newpage

%Add content for page two here (useful for two-sided printing)
%\thispagestyle{empty}
%\newpage

%\maketitle
\tableofcontents
\setcounter{page}{1} %Start the actually document on page 1


\part{Rapture}
\chapter{Background}
\Rapture is a platform system that can be used to build applications that are scalable,
distributed, consistent and coordinated. At its heart \Rapture is simply a well defined
set of libraries with an external facing api that provides an abstraction to a number
of fundamental concepts. This document describes the APIs of \Rapture in detail from the
perspective of a programmer - alongside each API call are sections on their use within
\Rapture and typical use cases for that API or API set.

The general architecture of a \Rapture system is reproduced in Figure~\vref{fig:RaptureDiagram}.

\begin{figure}[htb]
\centering
\includegraphics[width=15cm]{Graphics/rapturecore}
\caption{Rapture Component Parts}
\label{fig:RaptureDiagram}
\end{figure}

Applications that interact with \Rapture, or are \emph{hosted} on \Rapture will
use the \Rapture API to interact with this underlying framework. The goal of
the \Rapture API is that the interaction with \Rapture is invariant to the location
of the application -- the API looks \emph{the same} no matter where the application
resides.

\chapter{Application Locations}
Applications interacting with \Rapture will typicall fall within one of the
following categories.

\begin{itemize}
  \item{Client applications interacting with a remote \Rapture environment.}
  \item{Client \Reflex scripts using the ReflexRunner application.}
  \item{Server applications embedding the \Rapture kernel.}
  \item{Scripts running on a server that is itself running the \Rapture kernel.}
  \item{Extensions to \Reflex or repository or message drivers.}
  \item{Scripts run in the context of an ajax call from a web browser.}
\end{itemize}

\section{Context and Entitlements}
Every interaction with a \Rapture API call is made in the context of a logged in
user. That user, and its entitlement group membership and the parameters passed
to the call are used to determine whether the call can proceed or not. If an
API call is used to run a script on \Rapture or to start a workflow \emph{that} script
or workflow is run in the context of the calling user as well.

In some API use categories the interface used to interact with the \Rapture API
is already bound to a user - typically this is for processes that are running
server side. Client side use cases usually have to \emph{login} to \Rapture first,
providing credentials that get translated by the \Rapture API into a \verb+CallingContext+
value (a token) which can then be used in subsequent calls to identify the user
making the call. In some client side languages there are helper constructs that
can be used to automatically pass in the logged in context to the API calls,
leaving the programmer free to not worry about this aspect.

Wrapper applications such as ReflexRunner log in on the caller's behalf and then
pass that logged in API context to the underlying container that runs the script.

\section{Custom client applications}

Client applications that talk to \Rapture can be written in any of these supported languages as
long as the application can reach (using TCP/IP) a \Rapture API Server.

\begin{itemize}
  \item{Java (or anything that runs on the Java VM and can access Java classes.)}
  \item{.NET}
  \item{Python}
  \item{Javascript (typically node.js, see a later section on architectures for browser applications.)}
  \item{Ruby}
  \item{Go(lang)}
\end{itemize}

The transport between client and server uses a JSON-RPC style of communication which
means that other language support can easily be added. The build process for
\Rapture can autogenerate client side stubs once an initial template has been
created - the authors created the .NET implementation in a few hours.

Typically the use of \Rapture in these applications follows this pattern:

\begin{enumerate}
  \item{Obtain the ip address or name of the \Rapture environment API endpoint.}
  \item{Obtain the user name and password for the use of the API.}
  \item{Call a login function to obtain a calling context.}
  \item{Pass that login context into a wrapper (for future API use) or simply pass the context into future API calls.}
\end{enumerate}

For example in Java here is a simple code extract for the login and API use process:

\begin{lstlisting}[caption={Java simple example}, language=Java]
  String host = "test.incapture.net";
  String username = "test";
  String password = "secret";
  SimpleCredentialsProvider creds = new SimpleCredentialsProvider(username, password);
  HttpLoginApi loginApi = new HttpLoginApi(host, creds);
  loginApi.login();

  ScriptClient client = new ScriptClient(loginApi);
  String content = client.getDoc().getContent("//testRepo/doc/one");
  System.out.println(content);
\end{lstlisting}

This example logs into a \Rapture environment and passes that logged in context to a \verb+ScriptClient+
instance. It is this script client that can then be easily used to interact with \Rapture. The
\verb+getContent+ call in the document API will be described in detail later.

In Python, the equivalent interaction is reproduced below:

\begin{lstlisting}[caption={Python simple example}, language=Python]
  import raptureAPI
  url = 'test.incapture.net'
  username = 'test'
  password = 'secret'
  rapture = raptureAPI.raptureAPI(url, username, password)
  content = rapture.doDoc_GetContent('//testRepo/doc/one')
  print content
\end{lstlisting}

Here we see a similar login approach and then the invocation of the same \Rapture
API call. With the same target \Rapture environment these two code snippets will
produce exactly the same output.

\section{Client \Reflex scripts}

\Reflex scripts running on the client (or the server) are always running in a
container that has already been connected to an environment -- the wrapper is
the piece of code that has logged into \Rapture already.

In \Reflex then the code is even simpler. In fact \Reflex has some additional
syntax sugar for loading documents from \Rapture.

\begin{lstlisting}[caption={Reflex simple example}, language=reflex]
  contentAsMap <-- "//testRepo/doc/one";
  println(contentAsMap);

  // or

  content = #doc.getContent("//testRepo/doc/one");
  contentAsMap = fromjson(content);
  println(contentAsMap);
\end{lstlisting}

In the second access example we convert the raw JSON formatted document from
\Rapture into a \Reflex map structure so as to make the two approaches produce
the same output.

\section{Server side kernel applications}

If a Java (or Java VM) application embeds the \Rapture kernel code within it the
means for calling the \Rapture API can have a number of forms. Code running within
\Rapture has to be much more careful about calling contexts and who is actually
making the call, and there is no need to worry about host urls because the code
is running directly on \Rapture.

One approach to running the same example code is reproduced below:

\begin{lstlisting}[caption={Kernel simple example}, language=Java]
   CallingContext userContext = Kernel.getLogin().login(
          "test", "secret", null);
   String content = Kernel.getDoc().getContent(
          userContext, "//testRepo/doc/one");
   System.out.println(content);
\end{lstlisting}

Note that to run the above code your server application will have had to initialize
and configure itself first, something which is outside the scope of this document
but will trivially be a matter of defining configuration files for connection to
underlying data stores and then calling:

\begin{lstlisting}[caption={Kernel initialization}, language=Java]
   Kernel.initBootstrap();
\end{lstlisting}

\chapter{Application Location Summary}
Applications can sit in many places in the architecture of a solution that
includes a \Rapture system. The goal of the \Rapture API is to make the access
to \Rapture as simple and consistent as possible to all applications so it is
easy for a developer to move their applications within the system architecture or
to switch and change the programming languages used for an application.

\part{API}
\chapter{API Introduction}
The \Rapture API is divided into a number of sections. We've seen in an earlier
section how you may need to use the \emph{login} api to establish a connection
to \Rapture. The other sections of the API will form the rest of this document.
For each section the general positioning of the section with respect to \Rapture
will be described and then the detailed API calls will follow.

\chapter{Doc API}
\index{Doc API}

The document API for \Rapture is often abbreviated to \emph{Doc}. The API is used
to manipulate the presence and the content of document repositories in \Rapture.

In the abstract a document repository in \Rapture is a key/value store with optional
enhancements. The key in \Rapture corresponds to a URI for the document and where the
context is not obvious the scheme of the uri is \verb+document://+. In all document
API calls this scheme may be omitted.

Document repositories in \Rapture are backed by concrete data storage systems. When
you define a repository in \Rapture you provide a configuration string that is used
by \Rapture to route your request to a low level driver that interacts with the
underlying system. The format of this configuration string will be described in
the API call for creating a repository.

Document repositories can also be versioned. When you update a document in a
versioned repository the previous history of that document is preserved. In fact you can
qualify the URI of a document with the @ symbol and a version number to retrieve
previous versions of a document. Omitting the @ symbol will always retrieve the
latest version of a document.

Documents in repositories can also have metadata associated with them. \Rapture
automatically maintains some of this metadata - the time the document was created, the
user that created it. But a developer can use metadata update calls to add their
own attributes to documents in \Rapture.

The URI of a document in a repository implies a folder-like structure with the
forward slash delineating these folders. There are document API calls to treat a
document repository like a file system -- these are useful when constructing
browsable user interfaces to a repository.

\subsection{Methods}

\section{ValidateDocRepo}
\index{ValidateDocRepo}
\label{Api:ValidateDocRepo}
\begin{lstlisting}[style=nonumbers]
   boolean validateDocRepo (
           String    docRepoUri
   )
\end{lstlisting}
\begin{Verbatim}[formatcom=\color{Maroon}]
  Entitlement: /repo/write
\end{Verbatim}
%\begin{lstlisting}[language=reflex]
%ret = #doc.validateDocRepo(docRepoUri);
%\end{lstlisting}
The \verb+validateDocRepo+ call is used to instruct the repository implementation to validate itself. The effect is
implementation specific but it usually means checking that the repository can be connected to and is a valid \Rapture
repository.

The parameter to the call is simply a uri pointing to the repository name. The call can throw an exception if the
repository is not found or the parameter passed is invalid. It returns true if the underlying implementation asserts
that the repository is valid.



\rule{12cm}{2pt}
\section{CreateDocRepo}
\index{CreateDocRepo}
\label{Api:CreateDocRepo}
\begin{lstlisting}[style=nonumbers]
   void createDocRepo (
           String    docRepoUri
           String    config
   )
\end{lstlisting}
\begin{Verbatim}[formatcom=\color{Maroon}]
  Entitlement: /repo/write
\end{Verbatim}
%\begin{lstlisting}[language=reflex]
%ret = #doc.createDocRepo(docRepoUri,config);
%\end{lstlisting}
The \verb+createDocRepo+ is used to create a new document repository in \Rapture. The parameters to the call
look straightforward -- simply the name of the new repository and a configuration string. The configuration string is
in fact a complex instruction written in a repository domain specific language (DSL) that is used to define the
capabilities and underlying implementation of the repository.

The typical configuration string for a versioned repository backed by MongoDB is reproduced below:

\begin{verbatim}
NREP {} USING MONGODB { prefix = 'test' }
\end{verbatim}

The general form of the configuration is:

\begin{verbatim}
[type of document repo] { [ document repo config] }
     USING [underlying implementation] { [ config ]}
     [ ON [ instance] ]
\end{verbatim}

The first part, the type of the document repo, can be either \verb+NREP+ or \verb+REP+. The former
indicates that a versioned repository should be created, the latter that an unversioned repository
should be created.

The document repo config part of the configuration string is currently blank for all document repo types.

The second part of the configuration string defines the underlying implementation and its configuration. In
most cases the configuration associated with the implementation has a \verb+prefix+ parameter that is used to
define a table or a collection or a prefix for such entities in the underlying storage. The underlying implementation
defines what lower level software is used to host the data managed by \Rapture. The following table shows the current
implementations:

\begin{table}[h]
\begin{center}
\begin{tabular}{r l p{8cm}}
  Keyword & Underlying & Configuration \\
  \hline
  MONGODB & MongoDb & The prefix parameter defines the name of the collections used by this repository. To avoid
  conflict this is usually a function of the name of the \Rapture repository. \\
  CASSANDRA & Cassandra & The prefix parameter defines the name of the collections used by this repository. To avoid
  conflict this is usually a function of the name of the \Rapture repository. \\
  POSTGRES & PostgresSql &  The prefix parameter defines the name of the tables used by this repository. To avoid
  conflict this is usually a function of the name of the \Rapture repository. \\
\end{tabular}
\end{center}
\end{table}

Incapture has additional implementations of document repositories for Oracle, JDBC, Redis, ehCache and memcached.

There are some additional directives that can be used in a repository configuration definition.

If the keyword \verb+READONLY+ is used at the start of the configuration (e.g. \verb+READONLY NREP ...+) the repository will
be readonly -- it is assumed that either a different repository configuration is used to write data (and they
share the same \verb+preifx+ but different entitlements) or the underlying data in a repository is a backup of
a repository created with a different system. In any case the \verb+READONLY+ keyword makes all writing API calls to
this repository fail with an exception being thrown.

The \verb+ON+ directive defines which configuration will be used to connect to the underlying store. If
not present the \verb+DEFAULT+ configuration will be used. These keywords are used by the underlying
implementation to load a system specific configuration file, environment variables or property set.

For example the default configuration for MongoDb (\verb+ON DEFAULT+) instructs the MongoDB implementation
to look in three places for a connection string to a MongoDB server -

\begin{itemize}
\item{The environment variable RAPTUREMONGODB-DEFAULT.}
\item{The java property RAPTUREMONGODB-DEFAULT.}
\item{The line beginning default= in the file RaptureMONGODB.cfg on the classpath of the application.}
\end{itemize}

In most cases the implementation will read the value from the file associated with the application server.

Using this technique multiple underlying servers can be used and repositories attached to them using the
\verb+ON+ keyword.



\rule{12cm}{2pt}
\section{DocRepoExists}
\index{DocRepoExists}
\label{Api:DocRepoExists}
\begin{lstlisting}[style=nonumbers]
   boolean docRepoExists (
           String    docRepoUri
   )
\end{lstlisting}
\begin{Verbatim}[formatcom=\color{Maroon}]
  Entitlement: /repo/list
\end{Verbatim}
%\begin{lstlisting}[language=reflex]
%ret = #doc.docRepoExists(docRepoUri);
%\end{lstlisting}
The \verb+docRepoExists+ call is used to test whether a given named repository is known
to the \Rapture environment. The method takes a single repo name as a parameter and returns
true or false according to an existence check. This call is typically used in an installation
script to determine whether a repository should be created using the \verb+createDocRepo+ call.



\rule{12cm}{2pt}
\section{DocExists}
\index{DocExists}
\label{Api:DocExists}
\begin{lstlisting}[style=nonumbers]
   boolean docExists (
           String    docUri
   )
\end{lstlisting}
\begin{Verbatim}[formatcom=\color{Maroon}]
  Entitlement: /repo/list
\end{Verbatim}
%\begin{lstlisting}[language=reflex]
%ret = #doc.docExists(docUri);
%\end{lstlisting}
The \verb+docExists+ API call is used to determine whether there is a document
in a \Rapture system at a given uri. The call is usually more efficient than
attempting to retrieve a document depending on the underlying implementation
of the driver for the repository storage.



\rule{12cm}{2pt}
\section{GetDocRepoConfig}
\index{GetDocRepoConfig}
\label{Api:GetDocRepoConfig}
\begin{lstlisting}[style=nonumbers]
   DocumentRepoConfig getDocRepoConfig (
           String    docRepoUri
   )
\end{lstlisting}
\begin{Verbatim}[formatcom=\color{Maroon}]
  Entitlement: /repo/read
\end{Verbatim}
%\begin{lstlisting}[language=reflex]
%ret = #doc.getDocRepoConfig(docRepoUri);
%\end{lstlisting}
The \verb+getDocRepoConfig+ api call returns the underlying structure of a document repository. The return
value is a complex type that has the following fields:

\begin{table}[h]
  \small
\begin{center}
\begin{tabular}{r l p{7cm}}
  Field & Type & Description \\
  \hline
  description & String & The description of this repository. \\
  config & String & The configuration passed to the createDocRepo call. \\
  authority & String & No longer used. \\
  idGenUri & String & If the repository uses an autogenerated id mechanism this is the implementation to use. (See IDs) \\
  strictCheck & Boolean & If set the repository will validate documents as a JSON format before saving. \\
  indexes & Set<IndexScriptPair> & defined if there are any indices associated with this repository (see Indexes). \\
  documentRepo & RaptureDocConfig & a repeat of the information stored in the authority and config sections. \\
\end{tabular}
\end{center}
\end{table}



\rule{12cm}{2pt}
\section{GetDocRepoStatus}
\index{GetDocRepoStatus}
\label{Api:GetDocRepoStatus}
\begin{lstlisting}[style=nonumbers]
   Map<String,String> getDocRepoStatus (
           String    docRepoUri
   )
\end{lstlisting}
\begin{Verbatim}[formatcom=\color{Maroon}]
  Entitlement: /repo/read
\end{Verbatim}
%\begin{lstlisting}[language=reflex]
%ret = #doc.getDocRepoStatus(docRepoUri);
%\end{lstlisting}
The \verb+getDocRepoStatus+ call is used to query the low level implementation of a repository (its driver)
as to its status. The return value is implementation specific but is usually a set of key/value pairs that
normally include the storage used by the repository.



\rule{12cm}{2pt}
\section{GetDocRepoConfigs}
\index{GetDocRepoConfigs}
\label{Api:GetDocRepoConfigs}
\begin{lstlisting}[style=nonumbers]
   List<DocumentRepoConfig> getDocRepoConfigs (
   )
\end{lstlisting}
\begin{Verbatim}[formatcom=\color{Maroon}]
  Entitlement: /repo/read
\end{Verbatim}
%\begin{lstlisting}[language=reflex]
%ret = #doc.getDocRepoConfigs();
%\end{lstlisting}
The \verb+getDocRepoConfigs+ call returns all of the document repository configurations in use
on a \Rapture server. This call is often used in management user interfaces to provide a high level
starting point for navigating a \Rapture environment.

The object returned in a list from this function is described as part of the \verb+getDocRepoConfig+ call.



\rule{12cm}{2pt}
\section{DeleteDocRepo}
\index{DeleteDocRepo}
\label{Api:DeleteDocRepo}
\begin{lstlisting}[style=nonumbers]
   void deleteDocRepo (
           String    docRepoUri
   )
\end{lstlisting}
\begin{Verbatim}[formatcom=\color{Maroon}]
  Entitlement: /repo/write
\end{Verbatim}
%\begin{lstlisting}[language=reflex]
%ret = #doc.deleteDocRepo(docRepoUri);
%\end{lstlisting}
The \verb+deleteDocRepo+ call removes a document repository from a \Rapture system. The underlying
implementation of a repository driver determines whether the data associated with this repository
is also removed from the system.



\rule{12cm}{2pt}
\section{ArchiveRepoDocs}
\index{ArchiveRepoDocs}
\label{Api:ArchiveRepoDocs}
\begin{lstlisting}[style=nonumbers]
   void archiveRepoDocs (
           String    docRepoUri
           int    versionLimit
           long    timeLimit
           boolean    ensureVersionLimit
   )
\end{lstlisting}
\begin{Verbatim}[formatcom=\color{Maroon}]
  Entitlement: /repo/write
\end{Verbatim}
%\begin{lstlisting}[language=reflex]
%ret = #doc.archiveRepoDocs(docRepoUri,versionLimit,timeLimit,ensureVersionLimit);
%\end{lstlisting}
The \verb+archiveRepoDocs+ call removes older versions of documents from a
repository that match certain criteria.

\begin{table}[h]
\begin{center}
\begin{tabular}{r p{8cm}}
  Parameter & Purpose \\
  \hline
  docRepoUri & If this document uri contains just the name of a repository then all
  documents in the repository will be checked. Otherwise just those matching this prefix will be
  scanned. \\
  versionLimit & If the ensureVersionLimit parameter is true this parameter will determine
  the minimum number of versions to keep fora document. \\
  timeLimit & This parameter determines the modification time of the oldest version to keep during this
  archive process. \\
  ensureVersionLimit & If this parameter is true the versionLimit parameter is used \\
\end{tabular}
\end{center}
\end{table}



\rule{12cm}{2pt}
\section{GetDocAndMeta}
\index{GetDocAndMeta}
\label{Api:GetDocAndMeta}
\begin{lstlisting}[style=nonumbers]
   DocumentWithMeta getDocAndMeta (
           String    docUri
   )
\end{lstlisting}
\begin{Verbatim}[formatcom=\color{Maroon}]
  Entitlement: /data/read/$f(docUri)
\end{Verbatim}
%\begin{lstlisting}[language=reflex]
%ret = #doc.getDocAndMeta(docUri);
%\end{lstlisting}
The \verb+getDocAndMeta+ API call is used to retrieve both the content of
a document and the meta data associated with that document. If the uri does
not have a version qualifier the information returned will refer to the latest
version of the document. The return value is complex is structure -- please
refer to the table at the end of this section to understand the fields within that
structure.

The \verb+DocumentWithMeta+ return value is a complex structure that contains
the following fields:

\begin{table}[H]
  \small
\begin{center}
\begin{tabular}{rl p{7cm}}
  Field & Type & Description \\
  \hline
  displayName & String & the url of this document \\
  metaData & DocumentMetaData & the metadata associated with this document \\
  content & String & the content of this document \\
\end{tabular}
\end{center}
\end{table}



\rule{12cm}{2pt}
\section{GetDocMeta}
\index{GetDocMeta}
\label{Api:GetDocMeta}
\begin{lstlisting}[style=nonumbers]
   DocumentMetadata getDocMeta (
           String    docUri
   )
\end{lstlisting}
\begin{Verbatim}[formatcom=\color{Maroon}]
  Entitlement: /data/list/$f(docUri)
\end{Verbatim}
%\begin{lstlisting}[language=reflex]
%ret = #doc.getDocMeta(docUri);
%\end{lstlisting}
The \verb+getDocMeta+ API call is used to retrieve the meta data associated
with a given document in a repository. If the uri does not have a version
qualifier the information returned will refer to the latest version of the
document. The return value is complex is structure -- please refer to the
table at the end of this section to understand the fields within that
structure.

The \verb+DocumentMetaData+ return value is a complex structure that contains
the following fields:

\begin{table}[h]
\begin{center}
\begin{tabular}{rl p{8cm}}
  Field & Type & Description \\
  \hline
  version & integer & the version number of this document \\
  createdTimestamp & long & when this document was first created \\
  modifiedTimestamp & long & when this document was last modified \\
  user & string & the user that made this modification \\
  comment & string & any comment that was provided when the document was written \\
  deleted & boolean & whether this document is deleted \\
\end{tabular}
\end{center}
\end{table}



\rule{12cm}{2pt}
\section{RevertDoc}
\index{RevertDoc}
\label{Api:RevertDoc}
\begin{lstlisting}[style=nonumbers]
   DocumentWithMeta revertDoc (
           String    docUri
   )
\end{lstlisting}
\begin{Verbatim}[formatcom=\color{Maroon}]
  Entitlement: /data/write/$f(docUri)
\end{Verbatim}
%\begin{lstlisting}[language=reflex]
%ret = #doc.revertDoc(docUri);
%\end{lstlisting}
The \verb+revertDoc+ call loads the previous version of a document and saves
that as the latest version. It will create a new version so that audit
history is preserved.



\rule{12cm}{2pt}
\section{GetDoc}
\index{GetDoc}
\label{Api:GetDoc}
\begin{lstlisting}[style=nonumbers]
   String getDoc (
           String    docUri
   )
\end{lstlisting}
\begin{Verbatim}[formatcom=\color{Maroon}]
  Entitlement: /data/read/$f(docUri)
\end{Verbatim}
%\begin{lstlisting}[language=reflex]
%ret = #doc.getDoc(docUri);
%\end{lstlisting}
The \verb+getDoc+ API call is used to retrieve content from a document repository
in \Rapture. The single parameter \verb+docUri+ is a string that defines the
location of the document within the \Rapture system. It can optionally be
prefixed with the \verb+document:+ scheme qualifier. Depending on the client
side implementation the call can throw an exception if the repository is not
existent or the entitlements of the server prohibit such access. Some implementations
wrap the document does not exist failure into returning a null value from this
API call.

In the \Reflex scripting language a caller can use the \emph{pull} operator (\verb+<--+)
to retrieve a document from \Rapture and convert its contents from JSON to a \Reflex map
of maps. This call will fail if the document is not in JSON format.



\rule{12cm}{2pt}
\section{PutDoc}
\index{PutDoc}
\label{Api:PutDoc}
\begin{lstlisting}[style=nonumbers]
   String putDoc (
           String    docUri
           String    content
   )
\end{lstlisting}
\begin{Verbatim}[formatcom=\color{Maroon}]
  Entitlement: /data/write/$f(docUri)
\end{Verbatim}
%\begin{lstlisting}[language=reflex]
%ret = #doc.putDoc(docUri,content);
%\end{lstlisting}
The \verb+putDoc+ call writes data into a \Rapture document repository. The
data in a document repository is a simple string but this is usually formatted
as a JSON document. If the repository is versioned this call will write a new
version to the repository and set the latest version to be this version. If it
is not versioned this call will simply overwrite the current version.

If there is no document at this location the call will create a new document.

If the repository has an id generator attached to it (see the id API) the uri
can end in the special keyword \verb+#id+ a new unique id will be generated when
this document is written. The return value of this API call will then reflect
the actual location that was used to store this content.



\rule{12cm}{2pt}
\section{PutDocWithVersion}
\index{PutDocWithVersion}
\label{Api:PutDocWithVersion}
\begin{lstlisting}[style=nonumbers]
   boolean putDocWithVersion (
           String    docUri
           String    content
           int    currentVersion
   )
\end{lstlisting}
\begin{Verbatim}[formatcom=\color{Maroon}]
  Entitlement: /data/write/$f(docUri)
\end{Verbatim}
%\begin{lstlisting}[language=reflex]
%ret = #doc.putDocWithVersion(docUri,content,currentVersion);
%\end{lstlisting}
The \verb+putDocWithVersion+ can be used to implement an optimistic locking
strategy for updating data into a repository. The approach is for an application
to load a document with its version using the \verb+getDocAndMeta+ call. The
application can then modify that document and use this call to store the
update, passing in the last known version. The call will succeed if and only if
no other program has modified the document -- in which case the version number
of the document will be incremented and not match.

If the version check fails the call will fail with an exception that will be
returned to the caller in a manner appropriate to the calling language implementation.



\rule{12cm}{2pt}
\section{PutDocWithEventContext}
\index{PutDocWithEventContext}
\label{Api:PutDocWithEventContext}
\begin{lstlisting}[style=nonumbers]
   DocWriteHandle putDocWithEventContext (
           String    docUri
           String    content
           Map<String,String>    eventContext
   )
\end{lstlisting}
\begin{Verbatim}[formatcom=\color{Maroon}]
  Entitlement: /data/write/$f(docUri)
\end{Verbatim}
%\begin{lstlisting}[language=reflex]
%ret = #doc.putDocWithEventContext(docUri,content,eventContext);
%\end{lstlisting}
The \verb+putDocWithEventContext+ call is identical to \verb+putDoc+ except that
any events created by the act of storing this data will contain additional information
that is passed as the \verb+eventContext+ parameter. See \emph{Events} for more details
on how events are fired and attached to.



\rule{12cm}{2pt}
\section{DeleteDoc}
\index{DeleteDoc}
\label{Api:DeleteDoc}
\begin{lstlisting}[style=nonumbers]
   boolean deleteDoc (
           String    docUri
   )
\end{lstlisting}
\begin{Verbatim}[formatcom=\color{Maroon}]
  Entitlement: /data/write/$f(docUri)
\end{Verbatim}
%\begin{lstlisting}[language=reflex]
%ret = #doc.deleteDoc(docUri);
%\end{lstlisting}
The \verb+deleteDoc+ call is used to remove a document from a document repository.

In a versioned repository the history of the document is preserved -- the effect of
this call is to set the \emph{current} version of the document to a null version, so
that \verb+getDoc+ and \verb+existsDoc+ calls work as expected, but you can retrieve
old versions of the document.



\rule{12cm}{2pt}
\section{RenameDoc}
\index{RenameDoc}
\label{Api:RenameDoc}
\begin{lstlisting}[style=nonumbers]
   String renameDoc (
           String    fromDocUri
           String    toDocUri
   )
\end{lstlisting}
\begin{Verbatim}[formatcom=\color{Maroon}]
  Entitlement: /data/write/$f(fromDocUri)
\end{Verbatim}
%\begin{lstlisting}[language=reflex]
%ret = #doc.renameDoc(fromDocUri,toDocUri);
%\end{lstlisting}
The \verb+renameDoc+ call is an atomic wrapper around the \verb+deleteDoc+ and \verb+putDoc+.

The implementation simply retrieves the original document contents using
\verb+getDoc+ and then stores the document using \verb+putDoc+. Finally
the original document is deleted using \verb+deleteDoc+.

The entitlements check for this call also checks the entitlements for the
three calls it makes.



\rule{12cm}{2pt}
\section{GetDocs}
\index{GetDocs}
\label{Api:GetDocs}
\begin{lstlisting}[style=nonumbers]
   Map<String,String> getDocs (
           List<String>    docUris
   )
\end{lstlisting}
\begin{Verbatim}[formatcom=\color{Maroon}]
  Entitlement: /data/batch
\end{Verbatim}
%\begin{lstlisting}[language=reflex]
%ret = #doc.getDocs(docUris);
%\end{lstlisting}
The \verb+getDocs+ call takes a list of document uris and performs a \verb+getDoc+ call
on each uri. If the call passes the entitlement check for \verb+getDoc+ with that
uri the content is loaded and the added to the return map, keyed by the uri.



\rule{12cm}{2pt}
\section{GetDocAndMetas}
\index{GetDocAndMetas}
\label{Api:GetDocAndMetas}
\begin{lstlisting}[style=nonumbers]
   List<DocumentWithMeta> getDocAndMetas (
           List<String>    docUris
   )
\end{lstlisting}
\begin{Verbatim}[formatcom=\color{Maroon}]
  Entitlement: /data/batch
\end{Verbatim}
%\begin{lstlisting}[language=reflex]
%ret = #doc.getDocAndMetas(docUris);
%\end{lstlisting}
The \verb+getDocAndMetas+ call is a batch form of \verb+getDocAndMeta+. The return
value is the result of calling that function for each uri in the list passed to
the batch call and adding it to the return list. If the entitlement check fails
for the individual call a null is placed in the list at that point.



\rule{12cm}{2pt}
\section{DocsExist}
\index{DocsExist}
\label{Api:DocsExist}
\begin{lstlisting}[style=nonumbers]
   List<boolean> docsExist (
           List<String>    docUris
   )
\end{lstlisting}
\begin{Verbatim}[formatcom=\color{Maroon}]
  Entitlement: /data/batch
\end{Verbatim}
%\begin{lstlisting}[language=reflex]
%ret = #doc.docsExist(docUris);
%\end{lstlisting}
The \verb+docExists+ call is a batch form of the \verb+docExists+ call.

For each uri in the parameter to the call the \verb+docExists+ call is made
after an entitlement check. The return value of the call is added to the list
returned by this function. If the entitlement check fails a false is placed
in its position in the return list.



\rule{12cm}{2pt}
\section{PutDocs}
\index{PutDocs}
\label{Api:PutDocs}
\begin{lstlisting}[style=nonumbers]
   List<Object> putDocs (
           List<String>    docUris
           List<String>    contents
   )
\end{lstlisting}
\begin{Verbatim}[formatcom=\color{Maroon}]
  Entitlement: /data/batch
\end{Verbatim}
%\begin{lstlisting}[language=reflex]
%ret = #doc.putDocs(docUris,contents);
%\end{lstlisting}
The \verb+putDocs+ call is a wrapper around a loop to the \verb+putDoc+ call. For each
uri/content pair the entitlements for a given store are checked. The return value
is a list of the resultant uris created by the store call -- which can contain autogenerated
ids as per the \verb+putDoc+ implementation.



\rule{12cm}{2pt}
\section{RenameDocs}
\index{RenameDocs}
\label{Api:RenameDocs}
\begin{lstlisting}[style=nonumbers]
   List<String> renameDocs (
           String    authority
           String    comment
           List<String>    fromDocUris
           List<String>    toDocUris
   )
\end{lstlisting}
\begin{Verbatim}[formatcom=\color{Maroon}]
  Entitlement: /data/batch
\end{Verbatim}
%\begin{lstlisting}[language=reflex]
%ret = #doc.renameDocs(authority,comment,fromDocUris,toDocUris);
%\end{lstlisting}
The \verb+renameDocs+ call is a batch version of the \verb+renameDoc+ call.

The entitlement check for \verb+renameDoc+ is made for each uri in the batch. The
return value for \verb+renameDoc+ is added to the return list of this call. If
the entitlement check fails for a uri a null will occupy its place in the list.



\rule{12cm}{2pt}
\section{DeleteDocsByUriPrefix}
\index{DeleteDocsByUriPrefix}
\label{Api:DeleteDocsByUriPrefix}
\begin{lstlisting}[style=nonumbers]
   List<String> deleteDocsByUriPrefix (
           String    docUri
   )
\end{lstlisting}
\begin{Verbatim}[formatcom=\color{Maroon}]
  Entitlement: /data/write/$f(docUri)
\end{Verbatim}
%\begin{lstlisting}[language=reflex]
%ret = #doc.deleteDocsByUriPrefix(docUri);
%\end{lstlisting}
Delete Docs By URI Prefix




\rule{12cm}{2pt}
\section{ListDocsByUriPrefix}
\index{ListDocsByUriPrefix}
\label{Api:ListDocsByUriPrefix}
\begin{lstlisting}[style=nonumbers]
   Map<String,RaptureFolderInfo> listDocsByUriPrefix (
           String    docUri
           int    depth
   )
\end{lstlisting}
\begin{Verbatim}[formatcom=\color{Maroon}]
  Entitlement: /data/read/$f(docUri)
\end{Verbatim}
%\begin{lstlisting}[language=reflex]
%ret = #doc.listDocsByUriPrefix(docUri,depth);
%\end{lstlisting}
The \verb+listDocsByUriPrefix+ call is normally used by user interfaces that wish
to present a browser type interface on a document repository. The call returns all documents
and "sub folders" (to a given depth) below a given point in the hierarchy implied
by the naming conventions used in uris. Typically an interface will use \verb+/+ as
the initial prefix and then append onto that prefix the names of either documents
or folders for further \verb+listDocs+ type calls or \verb+getDoc+ if the location
maps to a real document.

The \verb+RaptureFolderInfo+ structure returned by this call is described below:

\begin{table}[ht]
\begin{center}
\begin{tabular}{r l p{8cm}}
  Field & Type & Description \\
  \hline
  name & String & The name of this element. \\
  folder & Boolean & Whether the name refers to a document or a sub-folder \\
\end{tabular}
\end{center}
\end{table}



\rule{12cm}{2pt}
\section{SetDocAttribute}
\index{SetDocAttribute}
\label{Api:SetDocAttribute}
\begin{lstlisting}[style=nonumbers]
   boolean setDocAttribute (
           String    attributeUri
           String    value
   )
\end{lstlisting}
\begin{Verbatim}[formatcom=\color{Maroon}]
  Entitlement: /data/write/$f(attributeUri)
\end{Verbatim}
%\begin{lstlisting}[language=reflex]
%ret = #doc.setDocAttribute(attributeUri,value);
%\end{lstlisting}
The \verb+setDocAttribute+ call is used to add an attribute to a document in a repository.

Attributes are metadata that can be used to place any arbitrary value on a document independent of its
content.

In the call an attribute uri is the combination of a document path (as seen in many other calls) with a
bookmark symbol (\verb+#+) followed by the name of an attribute. The value is a string value that has
meaning to the calling application.

The return value is true if the attribute was set by the repository.



\rule{12cm}{2pt}
\section{SetDocAttributes}
\index{SetDocAttributes}
\label{Api:SetDocAttributes}
\begin{lstlisting}[style=nonumbers]
   Map<String,boolean> setDocAttributes (
           String    attributeUri
           List<String>    keys
           List<String>    values
   )
\end{lstlisting}
\begin{Verbatim}[formatcom=\color{Maroon}]
  Entitlement: /data/write/$f(attributeUri)
\end{Verbatim}
%\begin{lstlisting}[language=reflex]
%ret = #doc.setDocAttributes(attributeUri,keys,values);
%\end{lstlisting}
The \verb+setDocAttributes+ call takes an attribute uri prefix and then loops through
each key and value in the passed parameters and calls the \verb+setDocAttribute+ call for each.

For each invocation the \verb+attributeUri+ to the \verb+setDocAttribute+ call is formed using
the following production:

\begin{verbatim}
   attributeUriForCall = attributeUri + '/' + key
\end{verbatim}

An entitlement check is made for each individual call.



\rule{12cm}{2pt}
\section{GetDocAttribute}
\index{GetDocAttribute}
\label{Api:GetDocAttribute}
\begin{lstlisting}[style=nonumbers]
   XferDocumentAttribute getDocAttribute (
           String    attributeUri
   )
\end{lstlisting}
\begin{Verbatim}[formatcom=\color{Maroon}]
  Entitlement: /data/read/$f(attributeUri)
\end{Verbatim}
%\begin{lstlisting}[language=reflex]
%ret = #doc.getDocAttribute(attributeUri);
%\end{lstlisting}
The \verb+getDocAttribute+ call mirrors the \verb+setDocAttribute+ call. A full attributeUri is passed
and the return value contains both the key and the value associated with the uri.



\rule{12cm}{2pt}
\section{GetDocAttributes}
\index{GetDocAttributes}
\label{Api:GetDocAttributes}
\begin{lstlisting}[style=nonumbers]
   List<XferDocumentAttribute> getDocAttributes (
           String    attributeUri
   )
\end{lstlisting}
\begin{Verbatim}[formatcom=\color{Maroon}]
  Entitlement: /data/read/$f(attributeUri)
\end{Verbatim}
%\begin{lstlisting}[language=reflex]
%ret = #doc.getDocAttributes(attributeUri);
%\end{lstlisting}
The \verb+getDocAttributes+ call is invoked on a uri pointing at a document in \Rapture. The return value is
a list of all of the attributes associated with that uri.



\rule{12cm}{2pt}
\section{DeleteDocAttribute}
\index{DeleteDocAttribute}
\label{Api:DeleteDocAttribute}
\begin{lstlisting}[style=nonumbers]
   boolean deleteDocAttribute (
           String    attributeUri
   )
\end{lstlisting}
\begin{Verbatim}[formatcom=\color{Maroon}]
  Entitlement: /data/write/$f(attributeUri)
\end{Verbatim}
%\begin{lstlisting}[language=reflex]
%ret = #doc.deleteDocAttribute(attributeUri);
%\end{lstlisting}
The \verb+deleteDocAttribute+ call is used to remove an attribute previously added using
\verb+setDocAttribute+. It returns true if the attribute was deleted (usually always the case unless
the attribute did not exist).



\rule{12cm}{2pt}
\section{GetDocRepoIdGenUri}
\index{GetDocRepoIdGenUri}
\label{Api:GetDocRepoIdGenUri}
\begin{lstlisting}[style=nonumbers]
   String getDocRepoIdGenUri (
           String    docRepoUri
   )
\end{lstlisting}
\begin{Verbatim}[formatcom=\color{Maroon}]
  Entitlement: /admin/idgen
\end{Verbatim}
%\begin{lstlisting}[language=reflex]
%ret = #doc.getDocRepoIdGenUri(docRepoUri);
%\end{lstlisting}
Document repositories can have an id generator attached to them. If a document is added to
a repository and it contains a "\verb+#id+" suffix the id generator will replace that keyword with
a newly generated unique id.

The \verb+getDocRepoIdGenUri+ returns the id generator internal uri that is associated with
this repository if it exists. Id generators can be independent of repositories (they can be used for
more than repository unique ids).



\rule{12cm}{2pt}
\section{SetDocRepoIdGenConfig}
\index{SetDocRepoIdGenConfig}
\label{Api:SetDocRepoIdGenConfig}
\begin{lstlisting}[style=nonumbers]
   DocumentRepoConfig setDocRepoIdGenConfig (
           String    docRepoUri
           String    idGenConfig
   )
\end{lstlisting}
\begin{Verbatim}[formatcom=\color{Maroon}]
  Entitlement: /admin/idgen
\end{Verbatim}
%\begin{lstlisting}[language=reflex]
%ret = #doc.setDocRepoIdGenConfig(docRepoUri,idGenConfig);
%\end{lstlisting}
Set Doc Repo Id Gen Config




\rule{12cm}{2pt}
\section{GetDocRepoIdGenConfig}
\index{GetDocRepoIdGenConfig}
\label{Api:GetDocRepoIdGenConfig}
\begin{lstlisting}[style=nonumbers]
   RaptureIdGenConfig getDocRepoIdGenConfig (
           String    docRepoUri
   )
\end{lstlisting}
\begin{Verbatim}[formatcom=\color{Maroon}]
  Entitlement: /admin/idgen
\end{Verbatim}
%\begin{lstlisting}[language=reflex]
%ret = #doc.getDocRepoIdGenConfig(docRepoUri);
%\end{lstlisting}
Document repositories can have an id generator attached to them. If a document is added to
a repository and it contains a "\verb+#id+" suffix the id generator will replace that keyword with
a newly generated unique id.

The \verb+getDocRepoIdGenConfig+ call retrieves the configuration of the associated id generator
that was previously set with the \verb+setDocRepoIdGenConfig+ call.



\rule{12cm}{2pt}

\chapter{Blob API}
\index{Blob API}

The document API for \Rapture is often abbreviated to \emph{Doc}. The API is used
to manipulate the presence and the content of document repositories in \Rapture.

In the abstract a document repository in \Rapture is a key/value store with optional
enhancements. The key in \Rapture corresponds to a URI for the document and where the
context is not obvious the scheme of the uri is \verb+document://+. In all document
API calls this scheme may be omitted.

Document repositories in \Rapture are backed by concrete data storage systems. When
you define a repository in \Rapture you provide a configuration string that is used
by \Rapture to route your request to a low level driver that interacts with the
underlying system. The format of this configuration string will be described in
the API call for creating a repository.

Document repositories can also be versioned. When you update a document in a
versioned repository the previous history of that document is preserved. In fact you can
qualify the URI of a document with the @ symbol and a version number to retrieve
previous versions of a document. Omitting the @ symbol will always retrieve the
latest version of a document.

Documents in repositories can also have metadata associated with them. \Rapture
automatically maintains some of this metadata - the time the document was created, the
user that created it. But a developer can use metadata update calls to add their
own attributes to documents in \Rapture.

The URI of a document in a repository implies a folder-like structure with the
forward slash delineating these folders. There are document API calls to treat a
document repository like a file system -- these are useful when constructing
browsable user interfaces to a repository.

\subsection{Methods}

\section{CreateBlobRepo}
\index{CreateBlobRepo}
\label{Api:CreateBlobRepo}
\begin{lstlisting}[style=nonumbers]
   void createBlobRepo (
           String    blobRepoUri
           String    config
           String    metaConfig
   )
\end{lstlisting}
\begin{Verbatim}[formatcom=\color{Maroon}]
  Entitlement: /repo/write
\end{Verbatim}
%\begin{lstlisting}[language=reflex]
%ret = #blob.createBlobRepo(blobRepoUri,config,metaConfig);
%\end{lstlisting}
The \verb+createBlobRepo+ is used to create a new blob repository in \Rapture. The two configuration strings
act in a similar way to the Document API's \verb+createDocRepo+ call, in fact the \verb+metaConfig+ parameter
is used to define an internal document repository that will be used to store and manage the meta data for the
repository. The documentation for \verb+createDocRepo+ shows the possible options for that configuration.

The main configuration string is a similar domain specific language that is used to pass configuration
options to an underlying implementation. The current options are simply described with an example.

The typical configuration string for a blob repository backed by MongoDB is reproduced below:

\begin{verbatim}
BLOB {} USING MONGODB { prefix = 'testBlob' }
\end{verbatim}

The general form of the configuration is:

\begin{verbatim}
BLOB { [ blob repo config] }
     USING [underlying implementation] { [ config ]}
     [ ON [ instance] ]
\end{verbatim}

The blob repo config part of the configuration string is currently blank for all blob repo types.

The second part of the configuration string defines the underlying implementation and its configuration. In
most cases the configuration associated with the implementation has a \verb+prefix+ parameter that is used to
define a table or a collection or a prefix for such entities in the underlying storage. The underlying implementation
defines what lower level software is used to host the data managed by \Rapture. The following table shows the current
implementations:

\begin{table}[h]
  \small
\begin{center}
\begin{tabular}{r l p{7cm}}
  Keyword & Underlying & Configuration \\
  \hline
  MONGODB & MongoDb & The prefix parameter defines the name of the collections used by this repository. To avoid
  conflict this is usually a function of the name of the \Rapture repository. \\
  CASSANDRA & Cassandra & The prefix parameter defines the name of the collections used by this repository. To avoid
  conflict this is usually a function of the name of the \Rapture repository. \\
\end{tabular}
\end{center}
\end{table}

There are some additional directives that can be used in a repository configuration definition.

The \verb+ON+ directive defines which configuration will be used to connect to the underlying store. If
not present the \verb+DEFAULT+ configuration will be used. These keywords are used by the underlying
implementation to load a system specific configuration file, environment variables or property set.

For example the default configuration for MongoDb (\verb+ON DEFAULT+) instructs the MongoDB implementation
to look in three places for a connection string to a MongoDB server -

\begin{itemize}
\item{The environment variable RAPTUREMONGODB-DEFAULT.}
\item{The java property RAPTUREMONGODB-DEFAULT.}
\item{The line beginning default= in the file RaptureMONGODB.cfg on the classpath of the application.}
\end{itemize}

In most cases the implementation will read the value from the file associated with the application server.

Using this technique multiple underlying servers can be used and repositories attached to them using the
\verb+ON+ keyword.



\rule{12cm}{2pt}
\section{GetBlobRepoConfig}
\index{GetBlobRepoConfig}
\label{Api:GetBlobRepoConfig}
\begin{lstlisting}[style=nonumbers]
   BlobRepoConfig getBlobRepoConfig (
           String    blobRepoUri
   )
\end{lstlisting}
\begin{Verbatim}[formatcom=\color{Maroon}]
  Entitlement: /repo/read
\end{Verbatim}
%\begin{lstlisting}[language=reflex]
%ret = #blob.getBlobRepoConfig(blobRepoUri);
%\end{lstlisting}
The \verb+getBlobRepoConfig+ call is used to retrieve the configuration of a blob
repository that was previously set with \verb+createBlobRepo+. It is primarily used in
administration user interfaces.



\rule{12cm}{2pt}
\section{GetBlobRepoConfigs}
\index{GetBlobRepoConfigs}
\label{Api:GetBlobRepoConfigs}
\begin{lstlisting}[style=nonumbers]
   List<BlobRepoConfig> getBlobRepoConfigs (
   )
\end{lstlisting}
\begin{Verbatim}[formatcom=\color{Maroon}]
  Entitlement: /repo/read
\end{Verbatim}
%\begin{lstlisting}[language=reflex]
%ret = #blob.getBlobRepoConfigs();
%\end{lstlisting}
\input{api/blob/getBlobRepoConfigs}


\rule{12cm}{2pt}
\section{DeleteBlobRepo}
\index{DeleteBlobRepo}
\label{Api:DeleteBlobRepo}
\begin{lstlisting}[style=nonumbers]
   void deleteBlobRepo (
           String    repoUri
   )
\end{lstlisting}
\begin{Verbatim}[formatcom=\color{Maroon}]
  Entitlement: /repo/write
\end{Verbatim}
%\begin{lstlisting}[language=reflex]
%ret = #blob.deleteBlobRepo(repoUri);
%\end{lstlisting}
The \verb+deleteBlobRepo+ call removes a blob repository from a \Rapture system. The underlying
implementation of a repository driver determines whether the data associated with this repository
is also removed from the system.

Currently both the MongoDB and Cassandra implementations of blob repository simply remove the configuration
and not the data.



\rule{12cm}{2pt}
\section{BlobRepoExists}
\index{BlobRepoExists}
\label{Api:BlobRepoExists}
\begin{lstlisting}[style=nonumbers]
   boolean blobRepoExists (
           String    repoUri
   )
\end{lstlisting}
\begin{Verbatim}[formatcom=\color{Maroon}]
  Entitlement: /repo/list
\end{Verbatim}
%\begin{lstlisting}[language=reflex]
%ret = #blob.blobRepoExists(repoUri);
%\end{lstlisting}
The \verb+blobRepoExists+ call checks that a blob repo is present in a \Rapture environment.



\rule{12cm}{2pt}
\section{BlobExists}
\index{BlobExists}
\label{Api:BlobExists}
\begin{lstlisting}[style=nonumbers]
   boolean blobExists (
           String    blobUri
   )
\end{lstlisting}
\begin{Verbatim}[formatcom=\color{Maroon}]
  Entitlement: /data/list/$f(blobUri)
\end{Verbatim}
%\begin{lstlisting}[language=reflex]
%ret = #blob.blobExists(blobUri);
%\end{lstlisting}
The \verb+blobExists+ call is used to test for the existence of a blob. The call is much
more efficient than attempting to retrieve the contents of a blob and testing for the failure
of that call.



\rule{12cm}{2pt}
\section{AddBlobContent}
\index{AddBlobContent}
\label{Api:AddBlobContent}
\begin{lstlisting}[style=nonumbers]
   void addBlobContent (
           String    blobUri
           byte[]    content
   )
\end{lstlisting}
\begin{Verbatim}[formatcom=\color{Maroon}]
  Entitlement: /data/write/$f(blobUri)
\end{Verbatim}
%\begin{lstlisting}[language=reflex]
%ret = #blob.addBlobContent(blobUri,content);
%\end{lstlisting}
The \verb+addBlobContent+ call is used to add content to a blob repository. The call
takes the uri of the blob to create or update and an array of bytes that constitute the blob
content. If a blob already exists this call will append to the existing blob. If the desire is to
overwrite any existing blob the \verb+putBlob+ call should be used instead.

Note that \Rapture servers that have the BlobContentServlet installed will support a RESTful interface
to adding content. A "file POST" to the url formed by prepending the uri of the blob with the prefix
\verb+/blob/+ will result in a more efficient client side experience.



\rule{12cm}{2pt}
\section{PutBlob}
\index{PutBlob}
\label{Api:PutBlob}
\begin{lstlisting}[style=nonumbers]
   void putBlob (
           String    blobUri
           byte[]    content
           String    contentType
   )
\end{lstlisting}
\begin{Verbatim}[formatcom=\color{Maroon}]
  Entitlement: /data/write/$f(blobUri)
\end{Verbatim}
%\begin{lstlisting}[language=reflex]
%ret = #blob.putBlob(blobUri,content,contentType);
%\end{lstlisting}
The \verb+putBlob+ call is used to add content to a blob repository. The call
takes the uri of the blob to create or update and an array of bytes that constitute the blob
content. A mime-type of the content is also supplied.

Note that \Rapture servers that have the BlobContentServlet installed will support a RESTful interface
to adding content. A "file POST" to the url formed by prepending the uri of the blob with the prefix
\verb+/blob/+ will result in a more efficient client side experience.



\rule{12cm}{2pt}
\section{GetBlob}
\index{GetBlob}
\label{Api:GetBlob}
\begin{lstlisting}[style=nonumbers]
   BlobContainer getBlob (
           String    blobUri
   )
\end{lstlisting}
\begin{Verbatim}[formatcom=\color{Maroon}]
  Entitlement: /data/read/$f(blobUri)
\end{Verbatim}
%\begin{lstlisting}[language=reflex]
%ret = #blob.getBlob(blobUri);
%\end{lstlisting}
The \verb+getBlob+ call is used to retrieve a blob from a repository. The \verb+BlobContainer+ structure
returned is complex - it contains both the content of the blob and any \emph{headers} associated with the
blob as provided by the underlying implementation. The keys to these headers will follow the HTTP protocol format and
include mime type and content size.

Note that \Rapture servers that have the BlobContentServlet installed will support a RESTful interface
to adding content. A "GET" to the url formed by prepending the uri of the blob with the prefix
\verb+/blob/+ will result in a more efficient client side experience, with the mime-type and content size
set in the headers of the return.



\rule{12cm}{2pt}
\section{DeleteBlob}
\index{DeleteBlob}
\label{Api:DeleteBlob}
\begin{lstlisting}[style=nonumbers]
   void deleteBlob (
           String    blobUri
   )
\end{lstlisting}
\begin{Verbatim}[formatcom=\color{Maroon}]
  Entitlement: /data/write/$f(blobUri)
\end{Verbatim}
%\begin{lstlisting}[language=reflex]
%ret = #blob.deleteBlob(blobUri);
%\end{lstlisting}
The \verb+deleteBlob+ call removes a blob from the repository.



\rule{12cm}{2pt}
\section{GetBlobSize}
\index{GetBlobSize}
\label{Api:GetBlobSize}
\begin{lstlisting}[style=nonumbers]
   Long getBlobSize (
           String    blobUri
   )
\end{lstlisting}
\begin{Verbatim}[formatcom=\color{Maroon}]
  Entitlement: /data/list/$f(blobUri)
\end{Verbatim}
%\begin{lstlisting}[language=reflex]
%ret = #blob.getBlobSize(blobUri);
%\end{lstlisting}
The \verb+getBlobSize+ call is a helper function that looks at the meta data for a blob and extracts the
size of the blob content from that data.



\rule{12cm}{2pt}
\section{GetBlobMetaData}
\index{GetBlobMetaData}
\label{Api:GetBlobMetaData}
\begin{lstlisting}[style=nonumbers]
   Map<String,String> getBlobMetaData (
           String    blobUri
   )
\end{lstlisting}
\begin{Verbatim}[formatcom=\color{Maroon}]
  Entitlement: /data/list/$f(blobUri)
\end{Verbatim}
%\begin{lstlisting}[language=reflex]
%ret = #blob.getBlobMetaData(blobUri);
%\end{lstlisting}
The \verb+getBlobMetaData+ call is used to retrieve meta data information about a blob, which normally
includes its mime type and size as well as information about who created the data and when.



\rule{12cm}{2pt}
\section{ListBlobsByUriPrefix}
\index{ListBlobsByUriPrefix}
\label{Api:ListBlobsByUriPrefix}
\begin{lstlisting}[style=nonumbers]
   Map<String,RaptureFolderInfo> listBlobsByUriPrefix (
           String    blobUri
           int    depth
   )
\end{lstlisting}
\begin{Verbatim}[formatcom=\color{Maroon}]
  Entitlement: /data/list/$f(blobUri)
\end{Verbatim}
%\begin{lstlisting}[language=reflex]
%ret = #blob.listBlobsByUriPrefix(blobUri,depth);
%\end{lstlisting}
The \verb+listBlobsByUriPrefix+ call is normally used by user interfaces that wish
to present a browser type interface on a blob repository. The call returns all blobs
and "sub folders" (to a given depth) below a given point in the hierarchy implied
by the naming conventions used in uris. Typically an interface will use \verb+/+ as
the initial prefix and then append onto that prefix the names of either documents
or folders for further \verb+listBlobs+ type calls or \verb+getBlob+ if the location
maps to a real blob.

The \verb+RaptureFolderInfo+ structure returned by this call is described below:

\begin{table}[H]
  \small
\begin{center}
\begin{tabular}{r l p{8cm}}
  Field & Type & Description \\
  \hline
  name & String & The name of this element. \\
  folder & Boolean & Whether the name refers to a blob or a sub-folder \\
\end{tabular}
\end{center}
\end{table}



\rule{12cm}{2pt}
\section{DeleteBlobsByUriPrefix}
\index{DeleteBlobsByUriPrefix}
\label{Api:DeleteBlobsByUriPrefix}
\begin{lstlisting}[style=nonumbers]
   List<String> deleteBlobsByUriPrefix (
           String    blobUri
   )
\end{lstlisting}
\begin{Verbatim}[formatcom=\color{Maroon}]
  Entitlement: /data/write/$f(blobUri)
\end{Verbatim}
%\begin{lstlisting}[language=reflex]
%ret = #blob.deleteBlobsByUriPrefix(blobUri);
%\end{lstlisting}
The \verb+deleteBlobsByUriPrefix+ call is used to remove all blobs below a certain
point in the hierarchy implied by the uri naming scheme. In conceptual terms it is
the equivalent of "removing the folder" from a repository.



\rule{12cm}{2pt}

\chapter{Series API}
\index{Series API}

The document API for \Rapture is often abbreviated to \emph{Doc}. The API is used
to manipulate the presence and the content of document repositories in \Rapture.

In the abstract a document repository in \Rapture is a key/value store with optional
enhancements. The key in \Rapture corresponds to a URI for the document and where the
context is not obvious the scheme of the uri is \verb+document://+. In all document
API calls this scheme may be omitted.

Document repositories in \Rapture are backed by concrete data storage systems. When
you define a repository in \Rapture you provide a configuration string that is used
by \Rapture to route your request to a low level driver that interacts with the
underlying system. The format of this configuration string will be described in
the API call for creating a repository.

Document repositories can also be versioned. When you update a document in a
versioned repository the previous history of that document is preserved. In fact you can
qualify the URI of a document with the @ symbol and a version number to retrieve
previous versions of a document. Omitting the @ symbol will always retrieve the
latest version of a document.

Documents in repositories can also have metadata associated with them. \Rapture
automatically maintains some of this metadata - the time the document was created, the
user that created it. But a developer can use metadata update calls to add their
own attributes to documents in \Rapture.

The URI of a document in a repository implies a folder-like structure with the
forward slash delineating these folders. There are document API calls to treat a
document repository like a file system -- these are useful when constructing
browsable user interfaces to a repository.

\subsection{Methods}

\section{CreateSeriesRepo}
\index{CreateSeriesRepo}
\label{Api:CreateSeriesRepo}
\begin{lstlisting}[style=nonumbers]
   void createSeriesRepo (
           String    seriesRepoUri
           String    config
   )
\end{lstlisting}
\begin{Verbatim}[formatcom=\color{Maroon}]
  Entitlement: /repo/write
\end{Verbatim}
%\begin{lstlisting}[language=reflex]
%ret = #series.createSeriesRepo(seriesRepoUri,config);
%\end{lstlisting}
The \verb+createSeriesRepo+ is used to create a new series repository in \Rapture. The parameters to the call
look straightforward -- simply the name of the new repository and a configuration string. The configuration string is
in fact a complex instruction written in a repository domain specific language (DSL) that is used to define the
capabilities and underlying implementation of the repository.

The typical configuration string for a series repository backed by MongoDB is reproduced below:

\begin{verbatim}
SREP {} USING MONGODB { prefix = 'test' }
\end{verbatim}

The general form of the configuration is:

\begin{verbatim}
SREP { [ series repo config] }
     USING [underlying implementation] { [ config ]}
     [ ON [ instance] ]
\end{verbatim}

The series repo config part of the configuration string is currently blank for all document repo types.

The second part of the configuration string defines the underlying implementation and its configuration. In
most cases the configuration associated with the implementation has a \verb+prefix+ parameter that is used to
define a table or a collection or a prefix for such entities in the underlying storage. The underlying implementation
defines what lower level software is used to host the data managed by \Rapture. The following table shows the current
implementations:

\begin{table}[h]
\begin{center}
\begin{tabular}{r l p{8cm}}
  Keyword & Underlying & Configuration \\
  \hline
  MONGODB & MongoDb & The prefix parameter defines the name of the collections used by this repository. To avoid
  conflict this is usually a function of the name of the \Rapture repository. \\
  CASSANDRA & Cassandra & The prefix parameter defines the name of the collections used by this repository. To avoid
  conflict this is usually a function of the name of the \Rapture repository. \\
\end{tabular}
\end{center}
\end{table}

The \verb+ON+ directive defines which configuration will be used to connect to the underlying store. If
not present the \verb+DEFAULT+ configuration will be used. These keywords are used by the underlying
implementation to load a system specific configuration file, environment variables or property set.

For example the default configuration for MongoDb (\verb+ON DEFAULT+) instructs the MongoDB implementation
to look in three places for a connection string to a MongoDB server -

\begin{itemize}
\item{The environment variable RAPTUREMONGODB-DEFAULT.}
\item{The java property RAPTUREMONGODB-DEFAULT.}
\item{The line beginning default= in the file RaptureMONGODB.cfg on the classpath of the application.}
\end{itemize}

In most cases the implementation will read the value from the file associated with the application server.

Using this technique multiple underlying servers can be used and repositories attached to them using the
\verb+ON+ keyword.



\rule{12cm}{2pt}
\section{CreateSeries}
\index{CreateSeries}
\label{Api:CreateSeries}
\begin{lstlisting}[style=nonumbers]
   void createSeries (
           String    seriesUri
   )
\end{lstlisting}
\begin{Verbatim}[formatcom=\color{Maroon}]
  Entitlement: /repo/write
\end{Verbatim}
%\begin{lstlisting}[language=reflex]
%ret = #series.createSeries(seriesUri);
%\end{lstlisting}
The \verb+createSeries+ call is used to simply create a series with no points. Normally to create
a series you would have to add a point to it -- this creates a placeholder for a series that points
can be added to.



\rule{12cm}{2pt}
\section{SeriesRepoExists}
\index{SeriesRepoExists}
\label{Api:SeriesRepoExists}
\begin{lstlisting}[style=nonumbers]
   boolean seriesRepoExists (
           String    seriesRepoUri
   )
\end{lstlisting}
\begin{Verbatim}[formatcom=\color{Maroon}]
  Entitlement: /repo/list
\end{Verbatim}
%\begin{lstlisting}[language=reflex]
%ret = #series.seriesRepoExists(seriesRepoUri);
%\end{lstlisting}
The \verb+seriesRepoExists+ call is an efficient way to test the existence of a series repository.



\rule{12cm}{2pt}
\section{SeriesExists}
\index{SeriesExists}
\label{Api:SeriesExists}
\begin{lstlisting}[style=nonumbers]
   boolean seriesExists (
           String    seriesUri
   )
\end{lstlisting}
\begin{Verbatim}[formatcom=\color{Maroon}]
  Entitlement: /repo/list
\end{Verbatim}
%\begin{lstlisting}[language=reflex]
%ret = #series.seriesExists(seriesUri);
%\end{lstlisting}
The \verb+seriesExist+ call is an efficient way to test the existence of a series in a repository.



\rule{12cm}{2pt}
\section{GetSeriesRepoConfig}
\index{GetSeriesRepoConfig}
\label{Api:GetSeriesRepoConfig}
\begin{lstlisting}[style=nonumbers]
   SeriesRepoConfig getSeriesRepoConfig (
           String    seriesRepoUri
   )
\end{lstlisting}
\begin{Verbatim}[formatcom=\color{Maroon}]
  Entitlement: /repo/read
\end{Verbatim}
%\begin{lstlisting}[language=reflex]
%ret = #series.getSeriesRepoConfig(seriesRepoUri);
%\end{lstlisting}
The \verb+getDocRepoConfig+ api call returns the underlying structure of a series repository. The return
value is a complex type that has the following fields:

\begin{table}[h]
\begin{center}
\begin{tabular}{r l p{8cm}}
  Field & Type & Description \\
  \hline
  description & String & The description of this repository. \\
  config & String & The configuration passed to the createDocRepo call. \\
  authority & String & No longer used. \\
  seriesName & String & The name of this series repository \\
  sampleColumn & String & A typical column format for this repository \\
\end{tabular}
\end{center}
\end{table}



\rule{12cm}{2pt}
\section{GetSeriesRepoConfigs}
\index{GetSeriesRepoConfigs}
\label{Api:GetSeriesRepoConfigs}
\begin{lstlisting}[style=nonumbers]
   List<SeriesRepoConfig> getSeriesRepoConfigs (
   )
\end{lstlisting}
\begin{Verbatim}[formatcom=\color{Maroon}]
  Entitlement: /repo/read
\end{Verbatim}
%\begin{lstlisting}[language=reflex]
%ret = #series.getSeriesRepoConfigs();
%\end{lstlisting}
The \verb+getSeriesRepoConfigs+ call returns all of the series repository configurations in use
on a \Rapture server. This call is often used in management user interfaces to provide a high level
starting point for navigating a \Rapture environment.

The object returned in a list from this function is described as part of the \verb+getSeriesRepoConfig+ call.



\rule{12cm}{2pt}
\section{DeleteSeriesRepo}
\index{DeleteSeriesRepo}
\label{Api:DeleteSeriesRepo}
\begin{lstlisting}[style=nonumbers]
   void deleteSeriesRepo (
           String    seriesRepoUri
   )
\end{lstlisting}
\begin{Verbatim}[formatcom=\color{Maroon}]
  Entitlement: /repo/write
\end{Verbatim}
%\begin{lstlisting}[language=reflex]
%ret = #series.deleteSeriesRepo(seriesRepoUri);
%\end{lstlisting}
The \verb+deleteSeriesRepo+ removes a series repository from \Rapture. Whether the underlying
data is removed is implementation specific. Both the Cassandra and MongoDB implementations
also remove the underlying data.



\rule{12cm}{2pt}
\section{DeleteSeries}
\index{DeleteSeries}
\label{Api:DeleteSeries}
\begin{lstlisting}[style=nonumbers]
   void deleteSeries (
           String    seriesUri
   )
\end{lstlisting}
\begin{Verbatim}[formatcom=\color{Maroon}]
  Entitlement: /data/write/$f(seriesUri)
\end{Verbatim}
%\begin{lstlisting}[language=reflex]
%ret = #series.deleteSeries(seriesUri);
%\end{lstlisting}
The \verb+deleteSeries+ call is used to remove the data behind a series and its
reference in the repository.



\rule{12cm}{2pt}
\section{DeleteSeriesByUriPrefix}
\index{DeleteSeriesByUriPrefix}
\label{Api:DeleteSeriesByUriPrefix}
\begin{lstlisting}[style=nonumbers]
   List<String> deleteSeriesByUriPrefix (
           String    seriesUri
   )
\end{lstlisting}
\begin{Verbatim}[formatcom=\color{Maroon}]
  Entitlement: /data/write/$f(seriesUri)
\end{Verbatim}
%\begin{lstlisting}[language=reflex]
%ret = #series.deleteSeriesByUriPrefix(seriesUri);
%\end{lstlisting}
The \verb+deleteSeriesByUriPrefix+ call is used to remove all series below a certain
point in the hierarchy implied by the uri naming scheme. In conceptual terms it is
the equivalent of "removing the folder" from a repository.



\rule{12cm}{2pt}
\section{AddDoubleToSeries}
\index{AddDoubleToSeries}
\label{Api:AddDoubleToSeries}
\begin{lstlisting}[style=nonumbers]
   void addDoubleToSeries (
           String    seriesUri
           String    pointKey
           double    pointValue
   )
\end{lstlisting}
\begin{Verbatim}[formatcom=\color{Maroon}]
  Entitlement: /data/write/$f(seriesUri)
\end{Verbatim}
%\begin{lstlisting}[language=reflex]
%ret = #series.addDoubleToSeries(seriesUri,pointKey,pointValue);
%\end{lstlisting}
\input{series/addDoubleToSeries}


\rule{12cm}{2pt}
\section{AddLongToSeries}
\index{AddLongToSeries}
\label{Api:AddLongToSeries}
\begin{lstlisting}[style=nonumbers]
   void addLongToSeries (
           String    seriesUri
           String    pointKey
           Long    pointValue
   )
\end{lstlisting}
\begin{Verbatim}[formatcom=\color{Maroon}]
  Entitlement: /data/write/$f(seriesUri)
\end{Verbatim}
%\begin{lstlisting}[language=reflex]
%ret = #series.addLongToSeries(seriesUri,pointKey,pointValue);
%\end{lstlisting}
The \verb+addLongToSeries+ call adds one data point to a series -- given the name of a series (its uri),
the column name to be added and its value.

If there is more than one point to be added it is often more efficient to call the \verb+addLongsToSeries+ call
instead.



\rule{12cm}{2pt}
\section{AddStringToSeries}
\index{AddStringToSeries}
\label{Api:AddStringToSeries}
\begin{lstlisting}[style=nonumbers]
   void addStringToSeries (
           String    seriesUri
           String    pointKey
           String    pointValue
   )
\end{lstlisting}
\begin{Verbatim}[formatcom=\color{Maroon}]
  Entitlement: /data/write/$f(seriesUri)
\end{Verbatim}
%\begin{lstlisting}[language=reflex]
%ret = #series.addStringToSeries(seriesUri,pointKey,pointValue);
%\end{lstlisting}
\input{series/addStringToSeries}


\rule{12cm}{2pt}
\section{AddStructureToSeries}
\index{AddStructureToSeries}
\label{Api:AddStructureToSeries}
\begin{lstlisting}[style=nonumbers]
   void addStructureToSeries (
           String    seriesUri
           String    pointKey
           String    pointValue
   )
\end{lstlisting}
\begin{Verbatim}[formatcom=\color{Maroon}]
  Entitlement: /data/write/$f(seriesUri)
\end{Verbatim}
%\begin{lstlisting}[language=reflex]
%ret = #series.addStructureToSeries(seriesUri,pointKey,pointValue);
%\end{lstlisting}
The \verb+addStructureToSeries+ is syntactically equivalent to \verb+addStringToSeries+ and most
implementations are also identical at present. The call is present to allow for differentiation of
storage mechanisms in underlying drivers.



\rule{12cm}{2pt}
\section{AddDoublesToSeries}
\index{AddDoublesToSeries}
\label{Api:AddDoublesToSeries}
\begin{lstlisting}[style=nonumbers]
   void addDoublesToSeries (
           String    seriesUri
           List<String>    pointKeys
           List<double>    pointValues
   )
\end{lstlisting}
\begin{Verbatim}[formatcom=\color{Maroon}]
  Entitlement: /data/write/$f(seriesUri)
\end{Verbatim}
%\begin{lstlisting}[language=reflex]
%ret = #series.addDoublesToSeries(seriesUri,pointKeys,pointValues);
%\end{lstlisting}
The \verb+addDoublesToSeries+ call takes a the uri to a series, a list of column names (or keys) and an equivalent
list (same size) of double values. The call adds the key/value pairs to the series referenced in the first parameter.



\rule{12cm}{2pt}
\section{AddLongsToSeries}
\index{AddLongsToSeries}
\label{Api:AddLongsToSeries}
\begin{lstlisting}[style=nonumbers]
   void addLongsToSeries (
           String    seriesUri
           List<String>    pointKeys
           List<Long>    pointValues
   )
\end{lstlisting}
\begin{Verbatim}[formatcom=\color{Maroon}]
  Entitlement: /data/write/$f(seriesUri)
\end{Verbatim}
%\begin{lstlisting}[language=reflex]
%ret = #series.addLongsToSeries(seriesUri,pointKeys,pointValues);
%\end{lstlisting}
The \verb+addLongsToSeries+ call takes a the uri to a series, a list of column names (or keys) and an equivalent
list (same size) of long values. The call adds the key/value pairs to the series referenced in the first parameter.



\rule{12cm}{2pt}
\section{AddStringsToSeries}
\index{AddStringsToSeries}
\label{Api:AddStringsToSeries}
\begin{lstlisting}[style=nonumbers]
   void addStringsToSeries (
           String    seriesUri
           List<String>    pointKeys
           List<String>    pointValues
   )
\end{lstlisting}
\begin{Verbatim}[formatcom=\color{Maroon}]
  Entitlement: /data/write/$f(seriesUri)
\end{Verbatim}
%\begin{lstlisting}[language=reflex]
%ret = #series.addStringsToSeries(seriesUri,pointKeys,pointValues);
%\end{lstlisting}
The \verb+addStringsToSeries+ call takes a the uri to a series, a list of column names (or keys) and an equivalent
list (same size) of string values. The call adds the key/value pairs to the series referenced in the first parameter.



\rule{12cm}{2pt}
\section{AddStructuresToSeries}
\index{AddStructuresToSeries}
\label{Api:AddStructuresToSeries}
\begin{lstlisting}[style=nonumbers]
   void addStructuresToSeries (
           String    seriesUri
           List<String>    pointKeys
           List<String>    pointValues
   )
\end{lstlisting}
\begin{Verbatim}[formatcom=\color{Maroon}]
  Entitlement: /data/write/$f(seriesUri)
\end{Verbatim}
%\begin{lstlisting}[language=reflex]
%ret = #series.addStructuresToSeries(seriesUri,pointKeys,pointValues);
%\end{lstlisting}
The \verb+addStructuresToSeries+ is syntactically equivalent to \verb+addStringsToSeries+ and most
implementations are also identical at present. The call is present to allow for differentiation of
storage mechanisms in underlying drivers.



\rule{12cm}{2pt}
\section{DeletePointsFromSeriesByPointKey}
\index{DeletePointsFromSeriesByPointKey}
\label{Api:DeletePointsFromSeriesByPointKey}
\begin{lstlisting}[style=nonumbers]
   void deletePointsFromSeriesByPointKey (
           String    seriesUri
           List<String>    pointKeys
   )
\end{lstlisting}
\begin{Verbatim}[formatcom=\color{Maroon}]
  Entitlement: /data/write/$f(seriesUri)
\end{Verbatim}
%\begin{lstlisting}[language=reflex]
%ret = #series.deletePointsFromSeriesByPointKey(seriesUri,pointKeys);
%\end{lstlisting}
The \verb+deletePointsFromSeriesByPointKey+ is used to remove a subset of a the data points
in a series by referencing the keys of each point individually.



\rule{12cm}{2pt}
\section{DeletePointsFromSeries}
\index{DeletePointsFromSeries}
\label{Api:DeletePointsFromSeries}
\begin{lstlisting}[style=nonumbers]
   void deletePointsFromSeries (
           String    seriesUri
   )
\end{lstlisting}
\begin{Verbatim}[formatcom=\color{Maroon}]
  Entitlement: /data/write/$f(seriesUri)
\end{Verbatim}
%\begin{lstlisting}[language=reflex]
%ret = #series.deletePointsFromSeries(seriesUri);
%\end{lstlisting}
The \verb+deletePointsFromSeries+ call is used to wipe out the data behind a series
without removing the series reference itself. To remove both the data and the series
reference use the \verb+deleteSeries+ call.



\rule{12cm}{2pt}
\section{GetLastPoint}
\index{GetLastPoint}
\label{Api:GetLastPoint}
\begin{lstlisting}[style=nonumbers]
   SeriesPoint getLastPoint (
           String    seriesUri
   )
\end{lstlisting}
\begin{Verbatim}[formatcom=\color{Maroon}]
  Entitlement: /data/read/$f(seriesUri)
\end{Verbatim}
%\begin{lstlisting}[language=reflex]
%ret = #series.getLastPoint(seriesUri);
%\end{lstlisting}
The \verb+getLastPoint+ call retrieves the last point in a series. The last point is determined by a
lexical sort of the column names in a series.

The \verb+SeriesPoint+ return value used in this call (and other point calls) is simply a compound
structure containing the column name and value (as a string).



\rule{12cm}{2pt}
\section{GetPoints}
\index{GetPoints}
\label{Api:GetPoints}
\begin{lstlisting}[style=nonumbers]
   List<SeriesPoint> getPoints (
           String    seriesUri
   )
\end{lstlisting}
\begin{Verbatim}[formatcom=\color{Maroon}]
  Entitlement: /data/read/$f(seriesUri)
\end{Verbatim}
%\begin{lstlisting}[language=reflex]
%ret = #series.getPoints(seriesUri);
%\end{lstlisting}
The \verb+getPoints+ call retrieves all of the points in a series, returning a list of
\verb+SeriesPoint+ objects.

The \verb+SeriesPoint+ return value used in this call (and other point calls) is simply a compound
structure containing the column name and value (as a string).



\rule{12cm}{2pt}
\section{GetPointsAfter}
\index{GetPointsAfter}
\label{Api:GetPointsAfter}
\begin{lstlisting}[style=nonumbers]
   List<SeriesPoint> getPointsAfter (
           String    seriesUri
           String    startColumn
           int    maxNumber
   )
\end{lstlisting}
\begin{Verbatim}[formatcom=\color{Maroon}]
  Entitlement: /data/read/$f(seriesUri)
\end{Verbatim}
%\begin{lstlisting}[language=reflex]
%ret = #series.getPointsAfter(seriesUri,startColumn,maxNumber);
%\end{lstlisting}
The most common way to page through points in a series is to use the \verb+getPointsAfter+ calls. In this call
a start column is specified (a zero length string implies the starting point in this case). A max number parameter
determines how many points could be returned.

If the return value contains "maxNumber" entries than the column name of the last point in that list can be
used as the starting point for the next "page" call.



\rule{12cm}{2pt}
\section{GetPointsAfterReverse}
\index{GetPointsAfterReverse}
\label{Api:GetPointsAfterReverse}
\begin{lstlisting}[style=nonumbers]
   List<SeriesPoint> getPointsAfterReverse (
           String    seriesUri
           String    startColumn
           int    maxNumber
   )
\end{lstlisting}
\begin{Verbatim}[formatcom=\color{Maroon}]
  Entitlement: /data/read/$f(seriesUri)
\end{Verbatim}
%\begin{lstlisting}[language=reflex]
%ret = #series.getPointsAfterReverse(seriesUri,startColumn,maxNumber);
%\end{lstlisting}
The most common way to page through points in a series is to use the \verb+getPointsAfter+ calls. In this call
the last column is specified (a zero length string implies the starting point in this case). A max number parameter
determines how many points could be returned. The call returns up to maxNumber values, starting from that last column
and then going in reverse from that point.

If the return value contains "maxNumber" entries than the column name of the last point in that list can be
used as the starting point for the next "page" call.



\rule{12cm}{2pt}
\section{GetPointsInRange}
\index{GetPointsInRange}
\label{Api:GetPointsInRange}
\begin{lstlisting}[style=nonumbers]
   List<SeriesPoint> getPointsInRange (
           String    seriesUri
           String    startColumn
           String    endColumn
           int    maxNumber
   )
\end{lstlisting}
\begin{Verbatim}[formatcom=\color{Maroon}]
  Entitlement: /data/read/$f(seriesUri)
\end{Verbatim}
%\begin{lstlisting}[language=reflex]
%ret = #series.getPointsInRange(seriesUri,startColumn,endColumn,maxNumber);
%\end{lstlisting}
\input{series/getPointsInRange}


\rule{12cm}{2pt}
\section{GetPointsAsDoubles}
\index{GetPointsAsDoubles}
\label{Api:GetPointsAsDoubles}
\begin{lstlisting}[style=nonumbers]
   List<SeriesDouble> getPointsAsDoubles (
           String    seriesUri
   )
\end{lstlisting}
\begin{Verbatim}[formatcom=\color{Maroon}]
  Entitlement: /data/read/$f(seriesUri)
\end{Verbatim}
%\begin{lstlisting}[language=reflex]
%ret = #series.getPointsAsDoubles(seriesUri);
%\end{lstlisting}
The \verb+getPointsAsDoubles+ call retrieves all of the points in a series, returning a list of
\verb+SeriesDouble+ objects.

The \verb+SeriesDouble+ return value used in this call (and other point calls) is simply a compound
structure containing the column name and value (as a double).



\rule{12cm}{2pt}
\section{GetPointsAfterAsDoubles}
\index{GetPointsAfterAsDoubles}
\label{Api:GetPointsAfterAsDoubles}
\begin{lstlisting}[style=nonumbers]
   List<SeriesDouble> getPointsAfterAsDoubles (
           String    seriesUri
           String    startColumn
           int    maxNumber
   )
\end{lstlisting}
\begin{Verbatim}[formatcom=\color{Maroon}]
  Entitlement: /data/read/$f(seriesUri)
\end{Verbatim}
%\begin{lstlisting}[language=reflex]
%ret = #series.getPointsAfterAsDoubles(seriesUri,startColumn,maxNumber);
%\end{lstlisting}
The most common way to page through points in a series is to use the \verb+getPointsAfter+ calls. In this call
a start column is specified (a zero length string implies the starting point in this case). A max number parameter
determines how many points could be returned.

If the return value contains "maxNumber" entries than the column name of the last point in that list can be
used as the starting point for the next "page" call.

The return value from this call contains a list of objects. Each object contains the column name and a value represented
as a double value.



\rule{12cm}{2pt}
\section{GetPointsInRangeAsDoubles}
\index{GetPointsInRangeAsDoubles}
\label{Api:GetPointsInRangeAsDoubles}
\begin{lstlisting}[style=nonumbers]
   List<SeriesDouble> getPointsInRangeAsDoubles (
           String    seriesUri
           String    startColumn
           String    endColumn
           int    maxNumber
   )
\end{lstlisting}
\begin{Verbatim}[formatcom=\color{Maroon}]
  Entitlement: /data/read/$f(seriesUri)
\end{Verbatim}
%\begin{lstlisting}[language=reflex]
%ret = #series.getPointsInRangeAsDoubles(seriesUri,startColumn,endColumn,maxNumber);
%\end{lstlisting}
\input{series/getPointsInRangeAsDoubles}


\rule{12cm}{2pt}
\section{GetPointsAsStrings}
\index{GetPointsAsStrings}
\label{Api:GetPointsAsStrings}
\begin{lstlisting}[style=nonumbers]
   List<SeriesString> getPointsAsStrings (
           String    seriesUri
   )
\end{lstlisting}
\begin{Verbatim}[formatcom=\color{Maroon}]
  Entitlement: /data/read/$f(seriesUri)
\end{Verbatim}
%\begin{lstlisting}[language=reflex]
%ret = #series.getPointsAsStrings(seriesUri);
%\end{lstlisting}
The \verb+getPointsAsStrings+ call retrieves all of the points in a series, returning a list of
\verb+SeriesString+ objects.

The \verb+SeriesString+ return value used in this call (and other point calls) is simply a compound
structure containing the column name and value (as a string).



\rule{12cm}{2pt}
\section{GetPointsAfterAsStrings}
\index{GetPointsAfterAsStrings}
\label{Api:GetPointsAfterAsStrings}
\begin{lstlisting}[style=nonumbers]
   List<SeriesString> getPointsAfterAsStrings (
           String    seriesUri
           String    startColumn
           int    maxNumber
   )
\end{lstlisting}
\begin{Verbatim}[formatcom=\color{Maroon}]
  Entitlement: /data/read/$f(seriesUri)
\end{Verbatim}
%\begin{lstlisting}[language=reflex]
%ret = #series.getPointsAfterAsStrings(seriesUri,startColumn,maxNumber);
%\end{lstlisting}
\input{series/getPointsAfterAsStrings}


\rule{12cm}{2pt}
\section{GetPointsInRangeAsStrings}
\index{GetPointsInRangeAsStrings}
\label{Api:GetPointsInRangeAsStrings}
\begin{lstlisting}[style=nonumbers]
   List<SeriesString> getPointsInRangeAsStrings (
           String    seriesUri
           String    startColumn
           String    endColumn
           int    maxNumber
   )
\end{lstlisting}
\begin{Verbatim}[formatcom=\color{Maroon}]
  Entitlement: /data/read/$f(seriesUri)
\end{Verbatim}
%\begin{lstlisting}[language=reflex]
%ret = #series.getPointsInRangeAsStrings(seriesUri,startColumn,endColumn,maxNumber);
%\end{lstlisting}
The \verb+getPointsInRangeAsStrings+ call returns up to maxNumber values that span the two end points provided as the
\verb+startColumn+ and \verb+endColumn+ parameters.

The return value from this call contains a list of objects. Each object contains the column name and a value represented
as a string value.



\rule{12cm}{2pt}
\section{RunSeriesScript}
\index{RunSeriesScript}
\label{Api:RunSeriesScript}
\begin{lstlisting}[style=nonumbers]
   List<SeriesPoint> runSeriesScript (
           String    scriptContent
           List<String>    arguments
   )
\end{lstlisting}
\begin{Verbatim}[formatcom=\color{Maroon}]
  Entitlement: /data/user
\end{Verbatim}
%\begin{lstlisting}[language=reflex]
%ret = #series.runSeriesScript(scriptContent,arguments);
%\end{lstlisting}
The \verb+runSeriesScript+ is a beta facility whereupon a script can be run against series data to return
a new series. Its scope is outside this manual.



\rule{12cm}{2pt}
\section{RunSeriesScriptQuiet}
\index{RunSeriesScriptQuiet}
\label{Api:RunSeriesScriptQuiet}
\begin{lstlisting}[style=nonumbers]
   void runSeriesScriptQuiet (
           String    scriptContent
           List<String>    arguments
   )
\end{lstlisting}
\begin{Verbatim}[formatcom=\color{Maroon}]
  Entitlement: /data/user
\end{Verbatim}
%\begin{lstlisting}[language=reflex]
%ret = #series.runSeriesScriptQuiet(scriptContent,arguments);
%\end{lstlisting}
The \verb+runSeriesScriptQuiet+ is a beta facility whereupon a script can be run against series data to return
a new series. Its scope is outside this manual.



\rule{12cm}{2pt}
\section{ListSeriesByUriPrefix}
\index{ListSeriesByUriPrefix}
\label{Api:ListSeriesByUriPrefix}
\begin{lstlisting}[style=nonumbers]
   Map<String,RaptureFolderInfo> listSeriesByUriPrefix (
           String    seriesUri
           int    depth
   )
\end{lstlisting}
\begin{Verbatim}[formatcom=\color{Maroon}]
  Entitlement: /data/read/$f(seriesUri)
\end{Verbatim}
%\begin{lstlisting}[language=reflex]
%ret = #series.listSeriesByUriPrefix(seriesUri,depth);
%\end{lstlisting}
\input{series/listSeriesByUriPrefix}


\rule{12cm}{2pt}

\section{Script API}

The document API for \Rapture is often abbreviated to \emph{Doc}. The API is used
to manipulate the presence and the content of document repositories in \Rapture.

In the abstract a document repository in \Rapture is a key/value store with optional
enhancements. The key in \Rapture corresponds to a URI for the document and where the
context is not obvious the scheme of the uri is \verb+document://+. In all document
API calls this scheme may be omitted.

Document repositories in \Rapture are backed by concrete data storage systems. When
you define a repository in \Rapture you provide a configuration string that is used
by \Rapture to route your request to a low level driver that interacts with the
underlying system. The format of this configuration string will be described in
the API call for creating a repository.

Document repositories can also be versioned. When you update a document in a
versioned repository the previous history of that document is preserved. In fact you can
qualify the URI of a document with the @ symbol and a version number to retrieve
previous versions of a document. Omitting the @ symbol will always retrieve the
latest version of a document.

Documents in repositories can also have metadata associated with them. \Rapture
automatically maintains some of this metadata - the time the document was created, the
user that created it. But a developer can use metadata update calls to add their
own attributes to documents in \Rapture.

The URI of a document in a repository implies a folder-like structure with the
forward slash delineating these folders. There are document API calls to treat a
document repository like a file system -- these are useful when constructing
browsable user interfaces to a repository.

\subsection{Methods}

\subsubsection{CreateScript}
\label{Api:CreateScript}
\begin{verbatim}
   RaptureScript createScript (
           String    scriptURI
           RaptureScriptLanguage    language
           RaptureScriptPurpose    purpose
           String    script
   )
\end{verbatim}
\begin{lstlisting}[language=reflex]
// Reflex use
ret = #script.createScript(scriptURI,language,purpose,script);
\end{lstlisting}
The \verb+createScript+ call is used to define a script in \Rapture for the first time. If the
script already exists and is to be updated you should use the \verb+putScript+ call instead.

The language and purpose fields primarily exist for extensibility. You should use "REFLEX" and "PROGRAM"
for these fields.



\rule{15cm}{2pt}
\subsubsection{CreateScriptLink}
\label{Api:CreateScriptLink}
\begin{verbatim}
   void createScriptLink (
           String    fromScriptURI
           String    toScriptURI
   )
\end{verbatim}
\begin{lstlisting}[language=reflex]
// Reflex use
ret = #script.createScriptLink(fromScriptURI,toScriptURI);
\end{lstlisting}
The \verb+createScriptLink+ creates a soft reference between two scripts in \Rapture. A script link
points to 2nd script and whenever a script link is used to run a script the target script is
executed instead. It is effectively creating an \emph{alias} to that script.

Script links are removed using the \verb+removeScriptLink+ call.



\rule{15cm}{2pt}
\subsubsection{RemoveScriptLink}
\label{Api:RemoveScriptLink}
\begin{verbatim}
   void removeScriptLink (
           String    fromScriptURI
   )
\end{verbatim}
\begin{lstlisting}[language=reflex]
// Reflex use
ret = #script.removeScriptLink(fromScriptURI);
\end{lstlisting}
The \verb+removeScriptLink+ reverses the changes made by the \verb+createScriptLink+ call.



\rule{15cm}{2pt}
\subsubsection{SetScriptParameters}
\label{Api:SetScriptParameters}
\begin{verbatim}
   RaptureScript setScriptParameters (
           String    scriptURI
           List<RaptureParameter>    parameters
   )
\end{verbatim}
\begin{lstlisting}[language=reflex]
// Reflex use
ret = #script.setScriptParameters(scriptURI,parameters);
\end{lstlisting}
The \verb+setScriptParameters+ can be used to define the parameters that this script should accept. It
is simply a recommendation -- no internal checks are made to see whether the script can actually accept these
parameters.

The definition of \verb+RaptureParameter+ is shown in the \verb+getScript+ call.



\rule{15cm}{2pt}
\subsubsection{DoesScriptExist}
\label{Api:DoesScriptExist}
\begin{verbatim}
   boolean doesScriptExist (
           String    scriptURI
   )
\end{verbatim}
\begin{lstlisting}[language=reflex]
// Reflex use
ret = #script.doesScriptExist(scriptURI);
\end{lstlisting}
The \verb+doesScriptExist+ call simply checks for the existence of a script in \Rapture.



\rule{15cm}{2pt}
\subsubsection{DeleteScript}
\label{Api:DeleteScript}
\begin{verbatim}
   void deleteScript (
           String    scriptURI
   )
\end{verbatim}
\begin{lstlisting}[language=reflex]
// Reflex use
ret = #script.deleteScript(scriptURI);
\end{lstlisting}
Scripts are removed from \Rapture by invoking the \verb+deleteScript+ api call. Note that currently no
reference checks are made as to the usage of the script -- if a workflow or an event is linked to this
script it will no longer execute correctly.



\rule{15cm}{2pt}
\subsubsection{GetScriptNames}
\label{Api:GetScriptNames}
\begin{verbatim}
   List<String> getScriptNames (
           String    scriptURI
   )
\end{verbatim}
\begin{lstlisting}[language=reflex]
// Reflex use
ret = #script.getScriptNames(scriptURI);
\end{lstlisting}
The \verb+getScriptNames+ call is deprecated. You should use the \verb+listScriptsByUriPrefix+ call instead.



\rule{15cm}{2pt}
\subsubsection{GetScript}
\label{Api:GetScript}
\begin{verbatim}
   RaptureScript getScript (
           String    scriptURI
   )
\end{verbatim}
\begin{lstlisting}[language=reflex]
// Reflex use
ret = #script.getScript(scriptURI);
\end{lstlisting}
The \verb+getScript+ retrieves a script previously created by \verb+createScript+.

The return value is complex and contains the following fields:

\begin{table}[h]
\begin{center}
\begin{tabular}{r l p{8cm}}
  Field & Type & Description \\
  \hline
  name & string & The name of this script.\\
  script & string & The body of the script.\\
  language & RaptureScriptLanguage & In all cases REFLEX \\
  purpose & RaptureScriptPurpose & In most cases PROGRAM.\\
  authority & string &  Unused.\\
  parameters & List(RaptureParameter) & The parameters this script expects.\\
\end{tabular}
\end{center}
\end{table}

The \verb+RaptureParameter+ type, used to define a parameter that the script could expect
and for the use in UI builders contains the following fields:

\begin{table}[h]
\begin{center}
\begin{tabular}{r l p{6cm}}
  Field & Type & Description \\
  \hline
  name & string & The name of this parameter.\\
  parameterType & RaptureParameterType & The expected type of this parameter (see below).\\
\end{tabular}
\end{center}
\end{table}

The RaptureParameterType can have the following values.

\begin{table}[h]
\begin{center}
\begin{tabular}{r p{10cm}}
  Type & Meaning \\
  \hline
  STRING & A string of characters.\\
  NUMBER & A number.\\
  DATE & Something (usually a string) that can be parsed to a date/time value.\\
  MAP & A key/value map.\\
  LIST & A list of values.\\
\end{tabular}
\end{center}
\end{table}



\rule{15cm}{2pt}
\subsubsection{PutScript}
\label{Api:PutScript}
\begin{verbatim}
   RaptureScript putScript (
           String    scriptURI
           RaptureScript    script
   )
\end{verbatim}
\begin{lstlisting}[language=reflex]
// Reflex use
ret = #script.putScript(scriptURI,script);
\end{lstlisting}
The \verb+putScript+ call is used to update the content (the actual script program) for a script.



\rule{15cm}{2pt}
\subsubsection{PutRawScript}
\label{Api:PutRawScript}
\begin{verbatim}
   RaptureScript putRawScript (
           String    scriptURI
           String    content
           String    language
           String    purpose
           List<String>    param_types
           List<String>    param_names
   )
\end{verbatim}
\begin{lstlisting}[language=reflex]
// Reflex use
ret = #script.putRawScript(scriptURI,content,language,purpose,param_types,param_names);
\end{lstlisting}
The \verb+putRawScript+ call can be used to create a script when all of the
information associated with the script is known at call time.



\rule{15cm}{2pt}
\subsubsection{RunScript}
\label{Api:RunScript}
\begin{verbatim}
   String runScript (
           String    scriptURI
           Map<String,String>    parameters
   )
\end{verbatim}
\begin{lstlisting}[language=reflex]
// Reflex use
ret = #script.runScript(scriptURI,parameters);
\end{lstlisting}
The \verb+runScript+ executes a script on the \Rapture server. The parameters for the execution, passed
by this call, are simply transferred to the context of the executing script. It is up to the script itself to
interpret those parameters as it sees fit. The \verb+RaptureParameter+ entries set up with \verb+setScriptParameters+ or
\verb+putRawScript+ are simply guidelines for a UI.

The return value of this call is the return value of the execution of the script -- whatever it \verb+returns+.



\rule{15cm}{2pt}
\subsubsection{RunScriptExtended}
\label{Api:RunScriptExtended}
\begin{verbatim}
   ScriptResult runScriptExtended (
           String    scriptURI
           Map<String,String>    parameters
   )
\end{verbatim}
\begin{lstlisting}[language=reflex]
// Reflex use
ret = #script.runScriptExtended(scriptURI,parameters);
\end{lstlisting}
The \verb+runScriptExtended+ works in a similar way to \verb+runScript+ except that the standard output from the script execution
is captured and returned to the caller. The \verb+ScriptResult+ return value contains two fields - the
\verb+returnValue+ as per the \verb+runScript+ call and a list of (string) lines in the \verb+output+ field.



\rule{15cm}{2pt}
\subsubsection{CheckScript}
\label{Api:CheckScript}
\begin{verbatim}
   String checkScript (
           String    scriptURI
   )
\end{verbatim}
\begin{lstlisting}[language=reflex]
// Reflex use
ret = #script.checkScript(scriptURI);
\end{lstlisting}
The \verb+checkScript+ call takes a script that has been saved to \Rapture and validates it for
syntax and structure.

If the script validates correctly the return value will be a zero length string. If incorrect the return
value will be a message indicating the parsing error which can be displayed to a user.



\rule{15cm}{2pt}
\subsubsection{CreateREPLSession}
\label{Api:CreateREPLSession}
\begin{verbatim}
   String createREPLSession (
   )
\end{verbatim}
\begin{lstlisting}[language=reflex]
// Reflex use
ret = #script.createREPLSession();
\end{lstlisting}
The \verb+createREPLSession+ (and associated REPL calls) are used to manage a "Read Evaluate Print Loop" for
the \Reflex scripting language. The idea is that this call returns a string handle that can be
associated with a user -- this handle stores the variable scope and any procedure declarations. As
users type in their \Reflex snippets in a UI the \verb+evaluateREPL+ call is used to update the context for this
session and to return some text which can be displayed on a terminal.



\rule{15cm}{2pt}
\subsubsection{DestroyREPLSession}
\label{Api:DestroyREPLSession}
\begin{verbatim}
   void destroyREPLSession (
           String    sessionId
   )
\end{verbatim}
\begin{lstlisting}[language=reflex]
// Reflex use
ret = #script.destroyREPLSession(sessionId);
\end{lstlisting}
The \verb+destroyREPLSession+ removes a session previously created by \verb+createREPLSession+.
Sessions can also be removed using the \verb+archiveOldREPLSessions+ call.



\rule{15cm}{2pt}
\subsubsection{EvaluateREPL}
\label{Api:EvaluateREPL}
\begin{verbatim}
   String evaluateREPL (
           String    sessionId
           String    line
   )
\end{verbatim}
\begin{lstlisting}[language=reflex]
// Reflex use
ret = #script.evaluateREPL(sessionId,line);
\end{lstlisting}
The \verb+evaluateREPL+ call sends a small piece of \Reflex code to an active REPL session (
previously created by \verb+createREPLSession+) and returns the result of evaluating that code in that
session.



\rule{15cm}{2pt}
\subsubsection{ArchiveOldREPLSessions}
\label{Api:ArchiveOldREPLSessions}
\begin{verbatim}
   void archiveOldREPLSessions (
           Long    ageInMinutes
   )
\end{verbatim}
\begin{lstlisting}[language=reflex]
// Reflex use
ret = #script.archiveOldREPLSessions(ageInMinutes);
\end{lstlisting}
The \verb+archiveOldREPLSessions+ call is an admin/operations API call that is used to remove
old \Reflex REPL sessions from \Rapture's ephemeral storage. In an active system where REPL is used
this should be called periodically to purge old sessions.



\rule{15cm}{2pt}
\subsubsection{CreateSnippet}
\label{Api:CreateSnippet}
\begin{verbatim}
   RaptureSnippet createSnippet (
           String    snippetURI
           String    snippet
   )
\end{verbatim}
\begin{lstlisting}[language=reflex]
// Reflex use
ret = #script.createSnippet(snippetURI,snippet);
\end{lstlisting}
A snippet is a small piece of script that could be used in a UI to paste into an actual
\Reflex script. They can be treated in a similar manner to "pasteable text".

The \verb+createSnippet+ call creates a snippet.



\rule{15cm}{2pt}
\subsubsection{GetSnippetChildren}
\label{Api:GetSnippetChildren}
\begin{verbatim}
   List<RaptureFolderInfo> getSnippetChildren (
           String    prefix
   )
\end{verbatim}
\begin{lstlisting}[language=reflex]
// Reflex use
ret = #script.getSnippetChildren(prefix);
\end{lstlisting}
The \verb+getSnippetChildren+ returns the snippets and folders defined in \Rapture below a specified
point in the naming heirarchy.



\rule{15cm}{2pt}
\subsubsection{DeleteSnippet}
\label{Api:DeleteSnippet}
\begin{verbatim}
   void deleteSnippet (
           String    snippetURI
   )
\end{verbatim}
\begin{lstlisting}[language=reflex]
// Reflex use
ret = #script.deleteSnippet(snippetURI);
\end{lstlisting}
The \verb+deleteSnippet+ call removes a script previously created by the \verb+createSnippet+ call.



\rule{15cm}{2pt}
\subsubsection{GetSnippet}
\label{Api:GetSnippet}
\begin{verbatim}
   RaptureSnippet getSnippet (
           String    snippetURI
   )
\end{verbatim}
\begin{lstlisting}[language=reflex]
// Reflex use
ret = #script.getSnippet(snippetURI);
\end{lstlisting}
The \verb+getSnippet+ call retrieves a snippet saved with the \verb+createSnippetCall+.



\rule{15cm}{2pt}
\subsubsection{ListScriptsByUriPrefix}
\label{Api:ListScriptsByUriPrefix}
\begin{verbatim}
   Map<String,RaptureFolderInfo> listScriptsByUriPrefix (
           String    uriPrefix
           int    depth
   )
\end{verbatim}
\begin{lstlisting}[language=reflex]
// Reflex use
ret = #script.listScriptsByUriPrefix(uriPrefix,depth);
\end{lstlisting}
\input{script/listScriptsByUriPrefix}


\rule{15cm}{2pt}
\subsubsection{DeleteScriptsByUriPrefix}
\label{Api:DeleteScriptsByUriPrefix}
\begin{verbatim}
   List<String> deleteScriptsByUriPrefix (
           String    uriPrefix
   )
\end{verbatim}
\begin{lstlisting}[language=reflex]
// Reflex use
ret = #script.deleteScriptsByUriPrefix(uriPrefix);
\end{lstlisting}
The \verb+deleteScriptsByUriPrefix+ call is used to remove all scripts below a certain point
in the hierarchy formed by the script naming convention.



\rule{15cm}{2pt}

\chapter{Decision API}
\index{Decision API}

The document API for \Rapture is often abbreviated to \emph{Doc}. The API is used
to manipulate the presence and the content of document repositories in \Rapture.

In the abstract a document repository in \Rapture is a key/value store with optional
enhancements. The key in \Rapture corresponds to a URI for the document and where the
context is not obvious the scheme of the uri is \verb+document://+. In all document
API calls this scheme may be omitted.

Document repositories in \Rapture are backed by concrete data storage systems. When
you define a repository in \Rapture you provide a configuration string that is used
by \Rapture to route your request to a low level driver that interacts with the
underlying system. The format of this configuration string will be described in
the API call for creating a repository.

Document repositories can also be versioned. When you update a document in a
versioned repository the previous history of that document is preserved. In fact you can
qualify the URI of a document with the @ symbol and a version number to retrieve
previous versions of a document. Omitting the @ symbol will always retrieve the
latest version of a document.

Documents in repositories can also have metadata associated with them. \Rapture
automatically maintains some of this metadata - the time the document was created, the
user that created it. But a developer can use metadata update calls to add their
own attributes to documents in \Rapture.

The URI of a document in a repository implies a folder-like structure with the
forward slash delineating these folders. There are document API calls to treat a
document repository like a file system -- these are useful when constructing
browsable user interfaces to a repository.

\subsection{Methods}

\subsection{GetAllWorkflows}
\index{GetAllWorkflows}
\label{Api:GetAllWorkflows}
\begin{verbatim}
   List<Workflow> getAllWorkflows (
   )
\end{verbatim}
\begin{Verbatim}[fontsize=\small, formatcom=\color{Maroon}]
  Entitlement: /decision/read
\end{Verbatim}
%\begin{lstlisting}[language=reflex]
%ret = #decision.getAllWorkflows();
%\end{lstlisting}
\input{decision/getAllWorkflows}


\rule{12cm}{2pt}
\subsection{GetWorkflowChildren}
\index{GetWorkflowChildren}
\label{Api:GetWorkflowChildren}
\begin{verbatim}
   List<RaptureFolderInfo> getWorkflowChildren (
           String    workflowURI
   )
\end{verbatim}
\begin{Verbatim}[fontsize=\small, formatcom=\color{Maroon}]
  Entitlement: /decision/read
\end{Verbatim}
%\begin{lstlisting}[language=reflex]
%ret = #decision.getWorkflowChildren(workflowURI);
%\end{lstlisting}
\input{decision/getWorkflowChildren}


\rule{12cm}{2pt}
\subsection{GetWorkOrderChildren}
\index{GetWorkOrderChildren}
\label{Api:GetWorkOrderChildren}
\begin{verbatim}
   List<RaptureFolderInfo> getWorkOrderChildren (
           String    parentPath
   )
\end{verbatim}
\begin{Verbatim}[fontsize=\small, formatcom=\color{Maroon}]
  Entitlement: /decision/read
\end{Verbatim}
%\begin{lstlisting}[language=reflex]
%ret = #decision.getWorkOrderChildren(parentPath);
%\end{lstlisting}
\input{decision/getWorkOrderChildren}


\rule{12cm}{2pt}
\subsection{PutWorkflow}
\index{PutWorkflow}
\label{Api:PutWorkflow}
\begin{verbatim}
   void putWorkflow (
           Workflow    workflow
   )
\end{verbatim}
\begin{Verbatim}[fontsize=\small, formatcom=\color{Maroon}]
  Entitlement: /decision/write
\end{Verbatim}
%\begin{lstlisting}[language=reflex]
%ret = #decision.putWorkflow(workflow);
%\end{lstlisting}
\input{decision/putWorkflow}


\rule{12cm}{2pt}
\subsection{GetWorkflow}
\index{GetWorkflow}
\label{Api:GetWorkflow}
\begin{verbatim}
   Workflow getWorkflow (
           String    workflowURI
   )
\end{verbatim}
\begin{Verbatim}[fontsize=\small, formatcom=\color{Maroon}]
  Entitlement: /decision/read/$f(workflowURI)
\end{Verbatim}
%\begin{lstlisting}[language=reflex]
%ret = #decision.getWorkflow(workflowURI);
%\end{lstlisting}
\input{decision/getWorkflow}


\rule{12cm}{2pt}
\subsection{GetWorkflowStep}
\index{GetWorkflowStep}
\label{Api:GetWorkflowStep}
\begin{verbatim}
   Step getWorkflowStep (
           String    stepURI
   )
\end{verbatim}
\begin{Verbatim}[fontsize=\small, formatcom=\color{Maroon}]
  Entitlement: /decision/read/$f(stepURI)
\end{Verbatim}
%\begin{lstlisting}[language=reflex]
%ret = #decision.getWorkflowStep(stepURI);
%\end{lstlisting}
\input{decision/getWorkflowStep}


\rule{12cm}{2pt}
\subsection{GetStepCategory}
\index{GetStepCategory}
\label{Api:GetStepCategory}
\begin{verbatim}
   String getStepCategory (
           String    stepURI
   )
\end{verbatim}
\begin{Verbatim}[fontsize=\small, formatcom=\color{Maroon}]
  Entitlement: /decision/read/$f(stepURI)
\end{Verbatim}
%\begin{lstlisting}[language=reflex]
%ret = #decision.getStepCategory(stepURI);
%\end{lstlisting}
\input{decision/getStepCategory}


\rule{12cm}{2pt}
\subsection{AddStep}
\index{AddStep}
\label{Api:AddStep}
\begin{verbatim}
   void addStep (
           String    workflowURI
           Step    step
   )
\end{verbatim}
\begin{Verbatim}[fontsize=\small, formatcom=\color{Maroon}]
  Entitlement: /decision/write/$f(workflowURI)
\end{Verbatim}
%\begin{lstlisting}[language=reflex]
%ret = #decision.addStep(workflowURI,step);
%\end{lstlisting}
The \verb+addStep+ call appends the passed in step to the workflow definition. This is
a call that is often used when defining a workflow for the first time, along with \verb+addTransition+.

The call \verb+putWorkflow+ is used to create the initial skeleton for the workflow.

The \verb+Step+ parameter is a complex structure that has the following fields.




\begin{table}[h]
  \small
\begin{center}
\begin{tabular}{r l p{7cm}}
  Field & Type & Description \\
  \hline
  name & string & The name of this step in the workflow. Used in transition references and the "startStep" field of a workflow.\\
  description & string & A description of this workflow. Used in user interfaces.\\
  executable & string & The task that will be executed as part of this step. (See below) \\
  view & map(string, string) & The view for this step -- the variables associated with this step. (See below).\\
  transitions & list(transitions) &  The transitions for this step. (See below).\\
  categoryOverride & string & If the step can only be executed on a certain group of \Rapture servers, this defines that group.\\
  softTimeout & integer & An indiciation as to how long (in seconds) the step is expected to run.\\
\end{tabular}
\end{center}
\end{table}

\subsubsection{Step Executable}
The step executable field of the \verb+Step+ type defines the underlying program that will be run
when this step is executed. It is a uri where the scheme can be one of the folllowing:

\begin{table}[h]
  \small
\begin{center}
\begin{tabular}{l p{6cm}}
  Scheme + Example & Description \\
  \hline
  \verb+script://test/one+ & Runs the \Reflex script referenced with the variables in the table below set. \\
  \verb+dp_java_invocable://myclass+ & Instantiate the java class \verb+rapture.dp.invocable.myclass+ which implements the \verb+AbstractInvocable+ interface. \\
  \verb+workflow://other/two+ & Spawns and switches to the attached workflow. The return value from the workflow is the return value of the step.\\
\end{tabular}
\end{center}
\end{table}

\begin{table}[h]
  \small
\begin{center}
\begin{tabular}{r p{10cm}}
  Variable Name & Meaning \\
  \hline
  \verb+$DP_WORK_ORDER_URI+ & The uri of the work order \\
  \verb+$DP_WORKER_URI+ & The uri of the worker \\
  \verb+$DP_WORKER_ID+ & The worker id \\
  \verb+$DP_STEP_NAME+ & The step name being executed \\
  \verb+$DP_STEP_START_TIME+ & The step start time \\
\end{tabular}
\end{center}
\end{table}



\rule{12cm}{2pt}
\subsection{RemoveStep}
\index{RemoveStep}
\label{Api:RemoveStep}
\begin{verbatim}
   void removeStep (
           String    workflowURI
           String    stepName
   )
\end{verbatim}
\begin{Verbatim}[fontsize=\small, formatcom=\color{Maroon}]
  Entitlement: /decision/write/$f(workflowURI)
\end{Verbatim}
%\begin{lstlisting}[language=reflex]
%ret = #decision.removeStep(workflowURI,stepName);
%\end{lstlisting}
\input{decision/removeStep}


\rule{12cm}{2pt}
\subsection{AddTransition}
\index{AddTransition}
\label{Api:AddTransition}
\begin{verbatim}
   void addTransition (
           String    workflowURI
           String    stepName
           Transition    transition
   )
\end{verbatim}
\begin{Verbatim}[fontsize=\small, formatcom=\color{Maroon}]
  Entitlement: /decision/write/$f(workflowURI)
\end{Verbatim}
%\begin{lstlisting}[language=reflex]
%ret = #decision.addTransition(workflowURI,stepName,transition);
%\end{lstlisting}
The \verb+addTransition+ call is used to add a transition to an existing workflow definition.

A transition is a link from one step to another step that will be traversed by a workflow if
the return value upon execution of a step matches the transition name.

As an example, imagine a step (implementing using a \Reflex script) that could return two results --
"filesProcessed" and "error". For this step we would add two transitions. The first would map the result
"filesProcessed" to a step that would perhaps continue the execution of the workflow. The second would map the
result "error" to a terminal step or maybe an error handler step.

The transition parameter to this call is a complex type which has fields defined as below.

\begin{table}[h]
  \small
\begin{center}
\begin{tabular}{r l p{8cm}}
  Field & Type & Description \\
  \hline
  name & string & The name of this transition in this step.\\
  targetStep & string & The step that this transition points to.\\
\end{tabular}
\end{center}
\end{table}

The \verb+targetStep+ field of a transition can contain special directives that control
special behavior in the workflow engine. These directives are described below.

\begin{table}[h]
  \small
\begin{center}
\begin{tabular}{r p{8cm}}
  Directive & Meaning \\
  \hline
  \verb+$RETURN+ & This directive instructs the workflow to exit and return a value to
     the caller. The return value string is supplied after the \verb+$RETURN+ field like so:
     \verb+$RETURN:ok+. \\
  \verb+$JOIN+ & What what \\
  \verb+$FORK+ & What what \\
  \verb+$SPLIT+ & What what \\

\end{tabular}
\end{center}
\end{table}



\rule{12cm}{2pt}
\subsection{RemoveTransition}
\index{RemoveTransition}
\label{Api:RemoveTransition}
\begin{verbatim}
   void removeTransition (
           String    workflowURI
           String    stepName
           String    transitionName
   )
\end{verbatim}
\begin{Verbatim}[fontsize=\small, formatcom=\color{Maroon}]
  Entitlement: /decision/write/$f(workflowURI)
\end{Verbatim}
%\begin{lstlisting}[language=reflex]
%ret = #decision.removeTransition(workflowURI,stepName,transitionName);
%\end{lstlisting}
\input{decision/removeTransition}


\rule{12cm}{2pt}
\subsection{DeleteWorkflow}
\index{DeleteWorkflow}
\label{Api:DeleteWorkflow}
\begin{verbatim}
   void deleteWorkflow (
           String    workflowURI
   )
\end{verbatim}
\begin{Verbatim}[fontsize=\small, formatcom=\color{Maroon}]
  Entitlement: /decision/write/$f(workflowURI)
\end{Verbatim}
%\begin{lstlisting}[language=reflex]
%ret = #decision.deleteWorkflow(workflowURI);
%\end{lstlisting}
\input{decision/deleteWorkflow}


\rule{12cm}{2pt}
\subsection{CreateWorkOrder}
\index{CreateWorkOrder}
\label{Api:CreateWorkOrder}
\begin{verbatim}
   String createWorkOrder (
           String    workflowURI
           Map<String,String>    argsMap
   )
\end{verbatim}
\begin{Verbatim}[fontsize=\small, formatcom=\color{Maroon}]
  Entitlement: /decision/execute/$f(workflowURI)
\end{Verbatim}
%\begin{lstlisting}[language=reflex]
%ret = #decision.createWorkOrder(workflowURI,argsMap);
%\end{lstlisting}
\input{decision/createWorkOrder}


\rule{12cm}{2pt}
\subsection{CreateWorkOrderP}
\index{CreateWorkOrderP}
\label{Api:CreateWorkOrderP}
\begin{verbatim}
   CreateResponse createWorkOrderP (
           String    workflowURI
           Map<String,String>    argsMap
           String    appStatusUriPattern
   )
\end{verbatim}
\begin{Verbatim}[fontsize=\small, formatcom=\color{Maroon}]
  Entitlement: /decision/execute/$f(workflowURI)
\end{Verbatim}
%\begin{lstlisting}[language=reflex]
%ret = #decision.createWorkOrderP(workflowURI,argsMap,appStatusUriPattern);
%\end{lstlisting}
\input{decision/createWorkOrderP}


\rule{12cm}{2pt}
\subsection{ReleaseWorkOrderLock}
\index{ReleaseWorkOrderLock}
\label{Api:ReleaseWorkOrderLock}
\begin{verbatim}
   void releaseWorkOrderLock (
           String    workOrderURI
   )
\end{verbatim}
\begin{Verbatim}[fontsize=\small, formatcom=\color{Maroon}]
  Entitlement: /decision/admin
\end{Verbatim}
%\begin{lstlisting}[language=reflex]
%ret = #decision.releaseWorkOrderLock(workOrderURI);
%\end{lstlisting}
\input{decision/releaseWorkOrderLock}


\rule{12cm}{2pt}
\subsection{GetWorkOrderStatus}
\index{GetWorkOrderStatus}
\label{Api:GetWorkOrderStatus}
\begin{verbatim}
   WorkOrderStatus getWorkOrderStatus (
           String    workOrderURI
   )
\end{verbatim}
\begin{Verbatim}[fontsize=\small, formatcom=\color{Maroon}]
  Entitlement: /decision/read/$f(workOrderURI)
\end{Verbatim}
%\begin{lstlisting}[language=reflex]
%ret = #decision.getWorkOrderStatus(workOrderURI);
%\end{lstlisting}
\input{decision/getWorkOrderStatus}


\rule{12cm}{2pt}
\subsection{WriteWorkflowAuditEntry}
\index{WriteWorkflowAuditEntry}
\label{Api:WriteWorkflowAuditEntry}
\begin{verbatim}
   void writeWorkflowAuditEntry (
           String    workOrderURI
           String    message
           boolean    error
   )
\end{verbatim}
\begin{Verbatim}[fontsize=\small, formatcom=\color{Maroon}]
  Entitlement: /decision/write/$f(workOrderURI)
\end{Verbatim}
%\begin{lstlisting}[language=reflex]
%ret = #decision.writeWorkflowAuditEntry(workOrderURI,message,error);
%\end{lstlisting}
\input{decision/writeWorkflowAuditEntry}


\rule{12cm}{2pt}
\subsection{GetWorkOrdersByDay}
\index{GetWorkOrdersByDay}
\label{Api:GetWorkOrdersByDay}
\begin{verbatim}
   List<WorkOrder> getWorkOrdersByDay (
           Long    startTimeInstant
   )
\end{verbatim}
\begin{Verbatim}[fontsize=\small, formatcom=\color{Maroon}]
  Entitlement: /decision/read
\end{Verbatim}
%\begin{lstlisting}[language=reflex]
%ret = #decision.getWorkOrdersByDay(startTimeInstant);
%\end{lstlisting}
\input{decision/getWorkOrdersByDay}


\rule{12cm}{2pt}
\subsection{GetWorkOrder}
\index{GetWorkOrder}
\label{Api:GetWorkOrder}
\begin{verbatim}
   WorkOrder getWorkOrder (
           String    workOrderURI
   )
\end{verbatim}
\begin{Verbatim}[fontsize=\small, formatcom=\color{Maroon}]
  Entitlement: /decision/read/$f(workOrderURI)
\end{Verbatim}
%\begin{lstlisting}[language=reflex]
%ret = #decision.getWorkOrder(workOrderURI);
%\end{lstlisting}
\input{decision/getWorkOrder}


\rule{12cm}{2pt}
\subsection{GetWorker}
\index{GetWorker}
\label{Api:GetWorker}
\begin{verbatim}
   Worker getWorker (
           String    workOrderURI
           String    workerId
   )
\end{verbatim}
\begin{Verbatim}[fontsize=\small, formatcom=\color{Maroon}]
  Entitlement: /decision/read/$f(workOrderURI)
\end{Verbatim}
%\begin{lstlisting}[language=reflex]
%ret = #decision.getWorker(workOrderURI,workerId);
%\end{lstlisting}
\input{decision/getWorker}


\rule{12cm}{2pt}
\subsection{CancelWorkOrder}
\index{CancelWorkOrder}
\label{Api:CancelWorkOrder}
\begin{verbatim}
   void cancelWorkOrder (
           String    workOrderURI
   )
\end{verbatim}
\begin{Verbatim}[fontsize=\small, formatcom=\color{Maroon}]
  Entitlement: /decision/write/$f(workOrderURI)
\end{Verbatim}
%\begin{lstlisting}[language=reflex]
%ret = #decision.cancelWorkOrder(workOrderURI);
%\end{lstlisting}
\input{decision/cancelWorkOrder}


\rule{12cm}{2pt}
\subsection{ResumeWorkOrder}
\index{ResumeWorkOrder}
\label{Api:ResumeWorkOrder}
\begin{verbatim}
   CreateResponse resumeWorkOrder (
           String    workOrderURI
           String    resumeStepURI
   )
\end{verbatim}
\begin{Verbatim}[fontsize=\small, formatcom=\color{Maroon}]
  Entitlement: /decision/write/$f(workOrderURI)
\end{Verbatim}
%\begin{lstlisting}[language=reflex]
%ret = #decision.resumeWorkOrder(workOrderURI,resumeStepURI);
%\end{lstlisting}
\input{decision/resumeWorkOrder}


\rule{12cm}{2pt}
\subsection{WasCancelCalled}
\index{WasCancelCalled}
\label{Api:WasCancelCalled}
\begin{verbatim}
   boolean wasCancelCalled (
           String    workOrderURI
   )
\end{verbatim}
\begin{Verbatim}[fontsize=\small, formatcom=\color{Maroon}]
  Entitlement: /decision/read/$f(workOrderURI)
\end{Verbatim}
%\begin{lstlisting}[language=reflex]
%ret = #decision.wasCancelCalled(workOrderURI);
%\end{lstlisting}
\input{decision/wasCancelCalled}


\rule{12cm}{2pt}
\subsection{GetCancellationDetails}
\index{GetCancellationDetails}
\label{Api:GetCancellationDetails}
\begin{verbatim}
   WorkOrderCancellation getCancellationDetails (
           String    workOrderURI
   )
\end{verbatim}
\begin{Verbatim}[fontsize=\small, formatcom=\color{Maroon}]
  Entitlement: /decision/read/$f(workOrderURI)
\end{Verbatim}
%\begin{lstlisting}[language=reflex]
%ret = #decision.getCancellationDetails(workOrderURI);
%\end{lstlisting}
\input{decision/getCancellationDetails}


\rule{12cm}{2pt}
\subsection{GetWorkOrderDebug}
\index{GetWorkOrderDebug}
\label{Api:GetWorkOrderDebug}
\begin{verbatim}
   WorkOrderDebug getWorkOrderDebug (
           String    workOrderURI
   )
\end{verbatim}
\begin{Verbatim}[fontsize=\small, formatcom=\color{Maroon}]
  Entitlement: /decision/debug/$f(workOrderURI)
\end{Verbatim}
%\begin{lstlisting}[language=reflex]
%ret = #decision.getWorkOrderDebug(workOrderURI);
%\end{lstlisting}
\input{decision/getWorkOrderDebug}


\rule{12cm}{2pt}
\subsection{SetWorkOrderIdGenConfig}
\index{SetWorkOrderIdGenConfig}
\label{Api:SetWorkOrderIdGenConfig}
\begin{verbatim}
   void setWorkOrderIdGenConfig (
           String    config
           boolean    force
   )
\end{verbatim}
\begin{Verbatim}[fontsize=\small, formatcom=\color{Maroon}]
  Entitlement: /decision/admin
\end{Verbatim}
%\begin{lstlisting}[language=reflex]
%ret = #decision.setWorkOrderIdGenConfig(config,force);
%\end{lstlisting}
\input{decision/setWorkOrderIdGenConfig}


\rule{12cm}{2pt}
\subsection{SetContextLiteral}
\index{SetContextLiteral}
\label{Api:SetContextLiteral}
\begin{verbatim}
   void setContextLiteral (
           String    workerURI
           String    varAlias
           String    literalValue
   )
\end{verbatim}
\begin{Verbatim}[fontsize=\small, formatcom=\color{Maroon}]
  Entitlement: /decision/write
\end{Verbatim}
%\begin{lstlisting}[language=reflex]
%ret = #decision.setContextLiteral(workerURI,varAlias,literalValue);
%\end{lstlisting}
\input{decision/setContextLiteral}


\rule{12cm}{2pt}
\subsection{SetContextLink}
\index{SetContextLink}
\label{Api:SetContextLink}
\begin{verbatim}
   void setContextLink (
           String    workerURI
           String    varAlias
           String    expressionValue
   )
\end{verbatim}
\begin{Verbatim}[fontsize=\small, formatcom=\color{Maroon}]
  Entitlement: /decision/write
\end{Verbatim}
%\begin{lstlisting}[language=reflex]
%ret = #decision.setContextLink(workerURI,varAlias,expressionValue);
%\end{lstlisting}
\input{decision/setContextLink}


\rule{12cm}{2pt}
\subsection{GetContextValue}
\index{GetContextValue}
\label{Api:GetContextValue}
\begin{verbatim}
   String getContextValue (
           String    workerURI
           String    varAlias
   )
\end{verbatim}
\begin{Verbatim}[fontsize=\small, formatcom=\color{Maroon}]
  Entitlement: /decision/read
\end{Verbatim}
%\begin{lstlisting}[language=reflex]
%ret = #decision.getContextValue(workerURI,varAlias);
%\end{lstlisting}
\input{decision/getContextValue}


\rule{12cm}{2pt}
\subsection{AddErrorToContext}
\index{AddErrorToContext}
\label{Api:AddErrorToContext}
\begin{verbatim}
   void addErrorToContext (
           String    workerURI
           ErrorWrapper    errorWrapper
   )
\end{verbatim}
\begin{Verbatim}[fontsize=\small, formatcom=\color{Maroon}]
  Entitlement: /decision/write
\end{Verbatim}
%\begin{lstlisting}[language=reflex]
%ret = #decision.addErrorToContext(workerURI,errorWrapper);
%\end{lstlisting}
The \verb+addErrorToContext+ call is used to take any errors associated with the workflow
and add it to the variable set that can be accessed from steps in the workflow.

The errors are added as a list with the keyword \verb+errorList+.

This call is often made during step execution when processing a failure path in a workflow.



\rule{12cm}{2pt}
\subsection{GetErrorsFromContext}
\index{GetErrorsFromContext}
\label{Api:GetErrorsFromContext}
\begin{verbatim}
   List<ErrorWrapper> getErrorsFromContext (
           String    workerURI
   )
\end{verbatim}
\begin{Verbatim}[fontsize=\small, formatcom=\color{Maroon}]
  Entitlement: /decision/read
\end{Verbatim}
%\begin{lstlisting}[language=reflex]
%ret = #decision.getErrorsFromContext(workerURI);
%\end{lstlisting}
\input{decision/getErrorsFromContext}


\rule{12cm}{2pt}
\subsection{GetExceptionInfo}
\index{GetExceptionInfo}
\label{Api:GetExceptionInfo}
\begin{verbatim}
   List<ErrorWrapper> getExceptionInfo (
           String    workOrderURI
   )
\end{verbatim}
\begin{Verbatim}[fontsize=\small, formatcom=\color{Maroon}]
  Entitlement: /decision/read
\end{Verbatim}
%\begin{lstlisting}[language=reflex]
%ret = #decision.getExceptionInfo(workOrderURI);
%\end{lstlisting}
\input{decision/getExceptionInfo}


\rule{12cm}{2pt}
\subsection{ReportStepProgress}
\index{ReportStepProgress}
\label{Api:ReportStepProgress}
\begin{verbatim}
   void reportStepProgress (
           String    workerURI
           Long    stepStartTime
           String    message
           Long    progress
           Long    max
   )
\end{verbatim}
\begin{Verbatim}[fontsize=\small, formatcom=\color{Maroon}]
  Entitlement: /decision/read
\end{Verbatim}
%\begin{lstlisting}[language=reflex]
%ret = #decision.reportStepProgress(workerURI,stepStartTime,message,progress,max);
%\end{lstlisting}
\input{decision/reportStepProgress}


\rule{12cm}{2pt}
\subsection{GetAppStatuses}
\index{GetAppStatuses}
\label{Api:GetAppStatuses}
\begin{verbatim}
   List<AppStatus> getAppStatuses (
           String    prefix
   )
\end{verbatim}
\begin{Verbatim}[fontsize=\small, formatcom=\color{Maroon}]
  Entitlement: /decision/read
\end{Verbatim}
%\begin{lstlisting}[language=reflex]
%ret = #decision.getAppStatuses(prefix);
%\end{lstlisting}
\input{decision/getAppStatuses}


\rule{12cm}{2pt}
\subsection{GetAppStatusDetails}
\index{GetAppStatusDetails}
\label{Api:GetAppStatusDetails}
\begin{verbatim}
   List<AppStatusDetails> getAppStatusDetails (
           String    prefix
           List<String>    extraContextValues
   )
\end{verbatim}
\begin{Verbatim}[fontsize=\small, formatcom=\color{Maroon}]
  Entitlement: /decision/read
\end{Verbatim}
%\begin{lstlisting}[language=reflex]
%ret = #decision.getAppStatusDetails(prefix,extraContextValues);
%\end{lstlisting}
\input{decision/getAppStatusDetails}


\rule{12cm}{2pt}
\subsection{GetMonthlyMetrics}
\index{GetMonthlyMetrics}
\label{Api:GetMonthlyMetrics}
\begin{verbatim}
   WorkflowHistoricalMetrics getMonthlyMetrics (
           String    workflowURI
           String    jobURI
           String    argsHashValue
           String    state
   )
\end{verbatim}
\begin{Verbatim}[fontsize=\small, formatcom=\color{Maroon}]
  Entitlement: /decision/read
\end{Verbatim}
%\begin{lstlisting}[language=reflex]
%ret = #decision.getMonthlyMetrics(workflowURI,jobURI,argsHashValue,state);
%\end{lstlisting}
\input{decision/getMonthlyMetrics}


\rule{12cm}{2pt}
\subsection{QueryLogs}
\index{QueryLogs}
\label{Api:QueryLogs}
\begin{verbatim}
   LogQueryResponse queryLogs (
           String    workOrderURI
           Long    startTime
           Long    endTime
           Long    keepAlive
           Long    bufferSize
           String    nextBatchId
           String    stepName
           String    stepStartTime
   )
\end{verbatim}
\begin{Verbatim}[fontsize=\small, formatcom=\color{Maroon}]
  Entitlement: /decision/read
\end{Verbatim}
%\begin{lstlisting}[language=reflex]
%ret = #decision.queryLogs(workOrderURI,startTime,endTime,keepAlive,bufferSize,nextBatchId,stepName,stepStartTime);
%\end{lstlisting}
\input{decision/queryLogs}


\rule{12cm}{2pt}

\chapter{Admin API}
\index{Admin API}

The document API for \Rapture is often abbreviated to \emph{Doc}. The API is used
to manipulate the presence and the content of document repositories in \Rapture.

In the abstract a document repository in \Rapture is a key/value store with optional
enhancements. The key in \Rapture corresponds to a URI for the document and where the
context is not obvious the scheme of the uri is \verb+document://+. In all document
API calls this scheme may be omitted.

Document repositories in \Rapture are backed by concrete data storage systems. When
you define a repository in \Rapture you provide a configuration string that is used
by \Rapture to route your request to a low level driver that interacts with the
underlying system. The format of this configuration string will be described in
the API call for creating a repository.

Document repositories can also be versioned. When you update a document in a
versioned repository the previous history of that document is preserved. In fact you can
qualify the URI of a document with the @ symbol and a version number to retrieve
previous versions of a document. Omitting the @ symbol will always retrieve the
latest version of a document.

Documents in repositories can also have metadata associated with them. \Rapture
automatically maintains some of this metadata - the time the document was created, the
user that created it. But a developer can use metadata update calls to add their
own attributes to documents in \Rapture.

The URI of a document in a repository implies a folder-like structure with the
forward slash delineating these folders. There are document API calls to treat a
document repository like a file system -- these are useful when constructing
browsable user interfaces to a repository.

\subsection{Methods}

\subsection{GetSystemProperties}
\index{GetSystemProperties}
\label{Api:GetSystemProperties}
\begin{verbatim}
   Map<String,String> getSystemProperties (
           List<String>    keys
   )
\end{verbatim}
%\begin{lstlisting}[language=reflex]
%ret = #admin.getSystemProperties(keys);
%\end{lstlisting}
The \verb+getSystemProperties+ call retrieves the set of system properties (java properties)
present on a the \Rapture server that responds to this call. The properties in a well run environment
will typically be identical across the environment.



\rule{15cm}{2pt}
\subsection{GetRepoConfig}
\index{GetRepoConfig}
\label{Api:GetRepoConfig}
\begin{verbatim}
   List<RepoConfig> getRepoConfig (
   )
\end{verbatim}
%\begin{lstlisting}[language=reflex]
%ret = #admin.getRepoConfig();
%\end{lstlisting}
\Rapture is a hierarchical set of repositories, and this method returns the config of the top most level - that used
for general config and temporary (transient) values such as sessions. In clustered mode these configs would
be referencing shared storage, and in test mode they would normally refer to in-memory versions of the config.
The manipulation of the top level config can be performed through the Bootstrap API.



\rule{15cm}{2pt}
\subsection{GetSessionsForUser}
\index{GetSessionsForUser}
\label{Api:GetSessionsForUser}
\begin{verbatim}
   List<CallingContext> getSessionsForUser (
           String    user
   )
\end{verbatim}
%\begin{lstlisting}[language=reflex]
%ret = #admin.getSessionsForUser(user);
%\end{lstlisting}
When a user logs into \Rapture they create a transient session and this method is a way of retrieving all of the sessions
for a given user. The \verb+CallingContext+ is a common object passed around \Rapture api calls at the lower level.



\rule{15cm}{2pt}
\subsection{DeleteUser}
\index{DeleteUser}
\label{Api:DeleteUser}
\begin{verbatim}
   void deleteUser (
           String    userName
   )
\end{verbatim}
%\begin{lstlisting}[language=reflex]
%ret = #admin.deleteUser(userName);
%\end{lstlisting}
This method removes a user from this Rapture system. The user is removed from all entitlement groups also. The actual
user definition is retained and marked as inactive (so the user cannot login). This is because the user may still be
referenced in audit trails and the change history in type repositories.



\rule{15cm}{2pt}
\subsection{DestroyUser}
\index{DestroyUser}
\label{Api:DestroyUser}
\begin{verbatim}
   void destroyUser (
           String    userName
   )
\end{verbatim}
%\begin{lstlisting}[language=reflex]
%ret = #admin.destroyUser(userName);
%\end{lstlisting}
This method destroys a user record. The user must have been previously disabled using \verb+deleteUser+ before this method 
may be called. This is a severe method that should only be used in non-production machines or to correct an administrative
error in creating an account with the wrong name before that account has been used. Reference to the missing user may still
exist, and may not display properly in some UIs



\rule{15cm}{2pt}
\subsection{RestoreUser}
\index{RestoreUser}
\label{Api:RestoreUser}
\begin{verbatim}
   void restoreUser (
           String    userName
   )
\end{verbatim}
%\begin{lstlisting}[language=reflex]
%ret = #admin.restoreUser(userName);
%\end{lstlisting}
This method restores a user that has been deleted.



\rule{15cm}{2pt}
\subsection{AddUser}
\index{AddUser}
\label{Api:AddUser}
\begin{verbatim}
   void addUser (
           String    userName
           String    description
           String    hashPassword
           String    email
   )
\end{verbatim}
%\begin{lstlisting}[language=reflex]
%ret = #admin.addUser(userName,description,hashPassword,email);
%\end{lstlisting}
This method adds a user to the \Rapture environment. The user will be in no entitlement groups by default. The password
field passed is actually the MD5 hash of the password - or at least the same hash function that will be applied when
logging in to the system (the password is hashed, and then hashed again with the salt returned during the login protocol).



\rule{15cm}{2pt}
\subsection{VerifyUser}
\index{VerifyUser}
\label{Api:VerifyUser}
\begin{verbatim}
   boolean verifyUser (
           String    userName
           String    token
   )
\end{verbatim}
%\begin{lstlisting}[language=reflex]
%ret = #admin.verifyUser(userName,token);
%\end{lstlisting}
The \verb+verifyUser+ call is used to check that a user is who they say they are -- a token can be generated through
the \verb+createRegistrationToken+ call and then sent to the user via a email. When they click on a link the glue code
between that endpoint and \Rapture can call this function to verify the user.



\rule{15cm}{2pt}
\subsection{DoesUserExist}
\index{DoesUserExist}
\label{Api:DoesUserExist}
\begin{verbatim}
   boolean doesUserExist (
           String    userName
   )
\end{verbatim}
%\begin{lstlisting}[language=reflex]
%ret = #admin.doesUserExist(userName);
%\end{lstlisting}
The \verb+doesUserExist+ call can be used to see if the user name passed is in use or not.



\rule{15cm}{2pt}
\subsection{GetUser}
\index{GetUser}
\label{Api:GetUser}
\begin{verbatim}
   RaptureUser getUser (
           String    userName
   )
\end{verbatim}
%\begin{lstlisting}[language=reflex]
%ret = #admin.getUser(userName);
%\end{lstlisting}
The \verb+getUser+ call returns a user record that contains such information as their full name and email address.



\rule{15cm}{2pt}
\subsection{GenerateApiUser}
\index{GenerateApiUser}
\label{Api:GenerateApiUser}
\begin{verbatim}
   RaptureUser generateApiUser (
           String    prefix
           String    description
   )
\end{verbatim}
%\begin{lstlisting}[language=reflex]
%ret = #admin.generateApiUser(prefix,description);
%\end{lstlisting}
The \verb+generateApiUser+ call is used to generate a special user that can be used by services. The user id is a UUID and
can be used with no password, so care must be taken in the use of api users. The fact that a user is an 'api user' is reflected
in the user record.



\rule{15cm}{2pt}
\subsection{ResetUserPassword}
\index{ResetUserPassword}
\label{Api:ResetUserPassword}
\begin{verbatim}
   void resetUserPassword (
           String    userName
           String    newHashPassword
   )
\end{verbatim}
%\begin{lstlisting}[language=reflex]
%ret = #admin.resetUserPassword(userName,newHashPassword);
%\end{lstlisting}
The \verb+resetUserPassword+ call is used to change the password for a given user. Users will use the equivalent call in the
User API, this call is for administrators.



\rule{15cm}{2pt}
\subsection{CreatePasswordResetToken}
\index{CreatePasswordResetToken}
\label{Api:CreatePasswordResetToken}
\begin{verbatim}
   String createPasswordResetToken (
           String    username
   )
\end{verbatim}
%\begin{lstlisting}[language=reflex]
%ret = #admin.createPasswordResetToken(username);
%\end{lstlisting}
If a user wishes to reset their password (perhaps because they have forgotten it) an administrator (or an admin style function) cancelPasswordResetToken
use this call to setup a new reset password token for that user. The returned token can then be sent to the user in some way -- perhaps by email.



\rule{15cm}{2pt}
\subsection{CreateRegistrationToken}
\index{CreateRegistrationToken}
\label{Api:CreateRegistrationToken}
\begin{verbatim}
   String createRegistrationToken (
           String    username
   )
\end{verbatim}
%\begin{lstlisting}[language=reflex]
%ret = #admin.createRegistrationToken(username);
%\end{lstlisting}
The \verb+createRegistrationToken+ creates a token that can be used to verify a newly created user - in particular that
the user's email address is valid. The token returned from this call can be used in the \verb+verifyUser+ call.



\rule{15cm}{2pt}
\subsection{CancelPasswordResetToken}
\index{CancelPasswordResetToken}
\label{Api:CancelPasswordResetToken}
\begin{verbatim}
   void cancelPasswordResetToken (
           String    username
   )
\end{verbatim}
%\begin{lstlisting}[language=reflex]
%ret = #admin.cancelPasswordResetToken(username);
%\end{lstlisting}
The \verb+cancelPasswordResetToken+ call is used to remove the request to change a password -- if a call comes in with that
token it will not be allowed to create a new password for this user.



\rule{15cm}{2pt}
\subsection{EmailUser}
\index{EmailUser}
\label{Api:EmailUser}
\begin{verbatim}
   void emailUser (
           String    userName
           String    emailTemplate
           Map<String,Object>    templateValues
   )
\end{verbatim}
%\begin{lstlisting}[language=reflex]
%ret = #admin.emailUser(userName,emailTemplate,templateValues);
%\end{lstlisting}
The \verb+emailUser+ call is used to send an email to a given user. The email address for the user is used for the
destination of the email. The template is defined by the \verb+addTemplate+ call and can contain references that are substituted by
the \verb+templateValues+ passed here.



\rule{15cm}{2pt}
\subsection{UpdateUserEmail}
\index{UpdateUserEmail}
\label{Api:UpdateUserEmail}
\begin{verbatim}
   void updateUserEmail (
           String    userName
           String    newEmail
   )
\end{verbatim}
%\begin{lstlisting}[language=reflex]
%ret = #admin.updateUserEmail(userName,newEmail);
%\end{lstlisting}
The \verb+updateUserEmail+ is used to refresh the email address associated with a user account.



\rule{15cm}{2pt}
\subsection{AddTemplate}
\index{AddTemplate}
\label{Api:AddTemplate}
\begin{verbatim}
   void addTemplate (
           String    name
           String    template
           boolean    overwrite
   )
\end{verbatim}
%\begin{lstlisting}[language=reflex]
%ret = #admin.addTemplate(name,template,overwrite);
%\end{lstlisting}
This function adds a template to the Rapture system. A template is a simple way of registering predefined configs that can be used to automatically generate configs for repositories, queues, and the like.
Templates use the popular StringTemplate library for merging values into a text template.



\rule{15cm}{2pt}
\subsection{RunTemplate}
\index{RunTemplate}
\label{Api:RunTemplate}
\begin{verbatim}
   String runTemplate (
           String    name
           String    parameters
   )
\end{verbatim}
%\begin{lstlisting}[language=reflex]
%ret = #admin.runTemplate(name,parameters);
%\end{lstlisting}
This method executes a template, replacing parts of the template with the passed parameters to create a new string.



\rule{15cm}{2pt}
\subsection{GetTemplate}
\index{GetTemplate}
\label{Api:GetTemplate}
\begin{verbatim}
   String getTemplate (
           String    name
   )
\end{verbatim}
%\begin{lstlisting}[language=reflex]
%ret = #admin.getTemplate(name);
%\end{lstlisting}
This method returns the definition of a template.



\rule{15cm}{2pt}
\subsection{CopyDocumentRepo}
\index{CopyDocumentRepo}
\label{Api:CopyDocumentRepo}
\begin{verbatim}
   void copyDocumentRepo (
           String    srcAuthority
           String    targAuthority
           boolean    wipe
   )
\end{verbatim}
%\begin{lstlisting}[language=reflex]
%ret = #admin.copyDocumentRepo(srcAuthority,targAuthority,wipe);
%\end{lstlisting}
Copies the data from one DocumentRepo to another. The target repository is wiped out before hand if 'wipe' is set to true. The target must already exist when this method is called.



\rule{15cm}{2pt}
\subsection{AddIPToWhiteList}
\index{AddIPToWhiteList}
\label{Api:AddIPToWhiteList}
\begin{verbatim}
   void addIPToWhiteList (
           String    ipAddress
   )
\end{verbatim}
%\begin{lstlisting}[language=reflex]
%ret = #admin.addIPToWhiteList(ipAddress);
%\end{lstlisting}
Use this method to add an IP address to a white list of allowed IP addresses that can
log in to this Rapture environment. Once set only IP addresses in this ipAddress list
can access Rapture. By default there are no whitelist IP addresses defined which actually
means that all IP addresses are allowed.



\rule{15cm}{2pt}
\subsection{RemoveIPFromWhiteList}
\index{RemoveIPFromWhiteList}
\label{Api:RemoveIPFromWhiteList}
\begin{verbatim}
   void removeIPFromWhiteList (
           String    ipAddress
   )
\end{verbatim}
%\begin{lstlisting}[language=reflex]
%ret = #admin.removeIPFromWhiteList(ipAddress);
%\end{lstlisting}
Use this method to remove an IP address from a white list.



\rule{15cm}{2pt}
\subsection{GetIPWhiteList}
\index{GetIPWhiteList}
\label{Api:GetIPWhiteList}
\begin{verbatim}
   List<String> getIPWhiteList (
   )
\end{verbatim}
%\begin{lstlisting}[language=reflex]
%ret = #admin.getIPWhiteList();
%\end{lstlisting}
Use this method to return the IP white list.



\rule{15cm}{2pt}
\subsection{GetAllUsers}
\index{GetAllUsers}
\label{Api:GetAllUsers}
\begin{verbatim}
   List<RaptureUser> getAllUsers (
   )
\end{verbatim}
%\begin{lstlisting}[language=reflex]
%ret = #admin.getAllUsers();
%\end{lstlisting}
The \verb+getAllUsers+ call is used to retrieve all of the registered users in the \Rapture system.



\rule{15cm}{2pt}
\subsection{InitiateTypeConversion}
\index{InitiateTypeConversion}
\label{Api:InitiateTypeConversion}
\begin{verbatim}
   void initiateTypeConversion (
           String    raptureURI
           String    newConfig
           int    versionsToKeep
   )
\end{verbatim}
%\begin{lstlisting}[language=reflex]
%ret = #admin.initiateTypeConversion(raptureURI,newConfig,versionsToKeep);
%\end{lstlisting}
This method kicks off a process that will migrate a DocumentRepo to an alternate config. A temporary type will be created with the new config,
 the old type will be locked for modifications and then all of the documents in the existing type will be copied to the new type, with the metadata intact.
 Optionally a number of historical versions will be kept if the source repository (and target) support it. Once all of the data has been copied the config
 attached to each type will be swapped and the type released for access. The temporary type will then be dropped.



\rule{15cm}{2pt}
\subsection{PutArchiveConfig}
\index{PutArchiveConfig}
\label{Api:PutArchiveConfig}
\begin{verbatim}
   void putArchiveConfig (
           String    raptureURI
           TypeArchiveConfig    config
   )
\end{verbatim}
%\begin{lstlisting}[language=reflex]
%ret = #admin.putArchiveConfig(raptureURI,config);
%\end{lstlisting}
The \verb+putArchiveConfig+ call is used to set the archive configuration for a given repository. This configuration
is used as guidelines when archiving data within repositories.



\rule{15cm}{2pt}
\subsection{GetArchiveConfig}
\index{GetArchiveConfig}
\label{Api:GetArchiveConfig}
\begin{verbatim}
   TypeArchiveConfig getArchiveConfig (
           String    raptureURI
   )
\end{verbatim}
%\begin{lstlisting}[language=reflex]
%ret = #admin.getArchiveConfig(raptureURI);
%\end{lstlisting}
The \verb+getArchiveConfig+ is used to retrieve an archive configuration previously setup with \verb+putArchiveConfig+.



\rule{15cm}{2pt}
\subsection{DeleteArchiveConfig}
\index{DeleteArchiveConfig}
\label{Api:DeleteArchiveConfig}
\begin{verbatim}
   void deleteArchiveConfig (
           String    raptureURI
   )
\end{verbatim}
%\begin{lstlisting}[language=reflex]
%ret = #admin.deleteArchiveConfig(raptureURI);
%\end{lstlisting}
The \verb+deleteArchiveConfig+ is used to remove an archive configuration previously setup with \verb+putArchiveConfig+.



\rule{15cm}{2pt}
\subsection{Ping}
\index{Ping}
\label{Api:Ping}
\begin{verbatim}
   boolean ping (
   )
\end{verbatim}
%\begin{lstlisting}[language=reflex]
%ret = #admin.ping();
%\end{lstlisting}
The \verb+ping+ call simply exercises the connection to \Rapture and can be used if the route of that connection
can be reset upon an inactivity timeout.



\rule{15cm}{2pt}
\subsection{AddMetadata}
\index{AddMetadata}
\label{Api:AddMetadata}
\begin{verbatim}
   void addMetadata (
           Map<String,String>    values
           boolean    overwrite
   )
\end{verbatim}
%\begin{lstlisting}[language=reflex]
%ret = #admin.addMetadata(values,overwrite);
%\end{lstlisting}
This function adds values to the metadata field of the \verb+CallingContext+. It's used to hold values specific to this connection. 
Since it's set by the caller the values cannot be considered entirely trusted, and private or secure data such as passwords shouldn't be stored in here.
If overwrite is false and an entry already exists then an exception should be thrown



\rule{15cm}{2pt}
\subsection{SetMOTD}
\index{SetMOTD}
\label{Api:SetMOTD}
\begin{verbatim}
   void setMOTD (
           String    message
   )
\end{verbatim}
%\begin{lstlisting}[language=reflex]
%ret = #admin.setMOTD(message);
%\end{lstlisting}
Set the MOTD (message of the day) for this environment. Setting to a zero length string implies that there is no message of the day.



\rule{15cm}{2pt}
\subsection{GetMOTD}
\index{GetMOTD}
\label{Api:GetMOTD}
\begin{verbatim}
   String getMOTD (
   )
\end{verbatim}
%\begin{lstlisting}[language=reflex]
%ret = #admin.getMOTD();
%\end{lstlisting}
The \verb+getMOTD+ retrieves the MOTD (Message of the Day) previously set by the \verb+setMOTD+ call.



\rule{15cm}{2pt}
\subsection{SetEnvironmentName}
\index{SetEnvironmentName}
\label{Api:SetEnvironmentName}
\begin{verbatim}
   void setEnvironmentName (
           String    name
   )
\end{verbatim}
%\begin{lstlisting}[language=reflex]
%ret = #admin.setEnvironmentName(name);
%\end{lstlisting}
The \verb+setEnvironmentName+ call simply sets the name of this environment -- something that can be used post login in
a user interface.



\rule{15cm}{2pt}
\subsection{SetEnvironmentProperties}
\index{SetEnvironmentProperties}
\label{Api:SetEnvironmentProperties}
\begin{verbatim}
   void setEnvironmentProperties (
           Map<String,String>    properties
   )
\end{verbatim}
%\begin{lstlisting}[language=reflex]
%ret = #admin.setEnvironmentProperties(properties);
%\end{lstlisting}
The \verb+setEnvironmentProperties+ is used to define environment properties that can be retrieved using the
\verb+getEnvironmentProperties+.



\rule{15cm}{2pt}
\subsection{GetEnvironmentName}
\index{GetEnvironmentName}
\label{Api:GetEnvironmentName}
\begin{verbatim}
   String getEnvironmentName (
   )
\end{verbatim}
%\begin{lstlisting}[language=reflex]
%ret = #admin.getEnvironmentName();
%\end{lstlisting}
The \verb+getEnvironmentName+ call is used to retrieve the name of an environment previously set by \verb+setEnvironmentName+.



\rule{15cm}{2pt}
\subsection{GetEnvironmentProperties}
\index{GetEnvironmentProperties}
\label{Api:GetEnvironmentProperties}
\begin{verbatim}
   Map<String,String> getEnvironmentProperties (
   )
\end{verbatim}
%\begin{lstlisting}[language=reflex]
%ret = #admin.getEnvironmentProperties();
%\end{lstlisting}
The \verb+getEnvironmentProperties+ call retrieves properties set by the \verb+setEnvironmentProperties+ call.



\rule{15cm}{2pt}
\subsection{Encode}
\index{Encode}
\label{Api:Encode}
\begin{verbatim}
   String encode (
           String    toEncode
   )
\end{verbatim}
%\begin{lstlisting}[language=reflex]
%ret = #admin.encode(toEncode);
%\end{lstlisting}
Encode a String using the default encoding mechanism.



\rule{15cm}{2pt}
\subsection{CreateURI}
\index{CreateURI}
\label{Api:CreateURI}
\begin{verbatim}
   String createURI (
           String    path
           String    leaf
   )
\end{verbatim}
%\begin{lstlisting}[language=reflex]
%ret = #admin.createURI(path,leaf);
%\end{lstlisting}
Create a URI with proper encoding given a path and a leaf. Normal URI characters such as : or / in the path will not be encoded.



\rule{15cm}{2pt}
\subsection{CreateMultipartURI}
\index{CreateMultipartURI}
\label{Api:CreateMultipartURI}
\begin{verbatim}
   String createMultipartURI (
           List<String>    elements
   )
\end{verbatim}
%\begin{lstlisting}[language=reflex]
%ret = #admin.createMultipartURI(elements);
%\end{lstlisting}
Create a URI with proper encoding given a list of elements. The return value will begin with // Each element will be encoded
(including all punctuation characters) and the elements joined together separated by /.



\rule{15cm}{2pt}
\subsection{Decode}
\index{Decode}
\label{Api:Decode}
\begin{verbatim}
   String decode (
           String    encoded
   )
\end{verbatim}
%\begin{lstlisting}[language=reflex]
%ret = #admin.decode(encoded);
%\end{lstlisting}
Decode the supplied String according to the URI encoding/decoding rules.



\rule{15cm}{2pt}
\subsection{FindGroupNamesByUser}
\index{FindGroupNamesByUser}
\label{Api:FindGroupNamesByUser}
\begin{verbatim}
   List<String> findGroupNamesByUser (
           String    username
   )
\end{verbatim}
%\begin{lstlisting}[language=reflex]
%ret = #admin.findGroupNamesByUser(username);
%\end{lstlisting}
Find the groups for a given user and return just the names.



\rule{15cm}{2pt}

\chapter{User API}

The document API for \Rapture is often abbreviated to \emph{Doc}. The API is used
to manipulate the presence and the content of document repositories in \Rapture.

In the abstract a document repository in \Rapture is a key/value store with optional
enhancements. The key in \Rapture corresponds to a URI for the document and where the
context is not obvious the scheme of the uri is \verb+document://+. In all document
API calls this scheme may be omitted.

Document repositories in \Rapture are backed by concrete data storage systems. When
you define a repository in \Rapture you provide a configuration string that is used
by \Rapture to route your request to a low level driver that interacts with the
underlying system. The format of this configuration string will be described in
the API call for creating a repository.

Document repositories can also be versioned. When you update a document in a
versioned repository the previous history of that document is preserved. In fact you can
qualify the URI of a document with the @ symbol and a version number to retrieve
previous versions of a document. Omitting the @ symbol will always retrieve the
latest version of a document.

Documents in repositories can also have metadata associated with them. \Rapture
automatically maintains some of this metadata - the time the document was created, the
user that created it. But a developer can use metadata update calls to add their
own attributes to documents in \Rapture.

The URI of a document in a repository implies a folder-like structure with the
forward slash delineating these folders. There are document API calls to treat a
document repository like a file system -- these are useful when constructing
browsable user interfaces to a repository.

\subsection{Methods}

\subsection{GetWhoAmI}
\label{Api:GetWhoAmI}
\begin{verbatim}
   RaptureUser getWhoAmI (
   )
\end{verbatim}
%\begin{lstlisting}[language=reflex]
%ret = #user.getWhoAmI();
%\end{lstlisting}
The \verb+getWhoAmi+ call returns the user record for the logged in user.



\rule{15cm}{2pt}
\subsection{LogoutUser}
\label{Api:LogoutUser}
\begin{verbatim}
   void logoutUser (
   )
\end{verbatim}
%\begin{lstlisting}[language=reflex]
%ret = #user.logoutUser();
%\end{lstlisting}
The \verb+logoutUser+ call removes the calling context (session) from the active session list, effectively logging out
every session using that context.



\rule{15cm}{2pt}
\subsection{StorePreference}
\label{Api:StorePreference}
\begin{verbatim}
   void storePreference (
           String    category
           String    name
           String    content
   )
\end{verbatim}
%\begin{lstlisting}[language=reflex]
%ret = #user.storePreference(category,name,content);
%\end{lstlisting}
Preferences are user specific configuration strings that can be associated with a user by applications. The category
name should be chosen carefully to prevent conflict with other applications.



\rule{15cm}{2pt}
\subsection{GetPreference}
\label{Api:GetPreference}
\begin{verbatim}
   String getPreference (
           String    category
           String    name
   )
\end{verbatim}
%\begin{lstlisting}[language=reflex]
%ret = #user.getPreference(category,name);
%\end{lstlisting}
The \verb+getPreference+ retrieves a configuration previously stored with \verb+storePreference+.



\rule{15cm}{2pt}
\subsection{RemovePreference}
\label{Api:RemovePreference}
\begin{verbatim}
   void removePreference (
           String    category
           String    name
   )
\end{verbatim}
%\begin{lstlisting}[language=reflex]
%ret = #user.removePreference(category,name);
%\end{lstlisting}
The \verb+removePreference+ call removes preferences previously set by \verb+storePreference+.



\rule{15cm}{2pt}
\subsection{GetPreferenceCategories}
\label{Api:GetPreferenceCategories}
\begin{verbatim}
   List<String> getPreferenceCategories (
   )
\end{verbatim}
%\begin{lstlisting}[language=reflex]
%ret = #user.getPreferenceCategories();
%\end{lstlisting}
The \verb+getPreferenceCategories+ call is used to retrieve all of the high level preference categories associated with
this user, and is primarily used by user interfaces that which to display these preferences in some way.



\rule{15cm}{2pt}
\subsection{GetPreferencesInCategory}
\label{Api:GetPreferencesInCategory}
\begin{verbatim}
   List<String> getPreferencesInCategory (
           String    category
   )
\end{verbatim}
%\begin{lstlisting}[language=reflex]
%ret = #user.getPreferencesInCategory(category);
%\end{lstlisting}
The \verb+getPreferencesInCategory+ call is used to retrieve all of the preference names in a given preference category.
 



\rule{15cm}{2pt}
\subsection{UpdateMyDescription}
\label{Api:UpdateMyDescription}
\begin{verbatim}
   RaptureUser updateMyDescription (
           String    description
   )
\end{verbatim}
%\begin{lstlisting}[language=reflex]
%ret = #user.updateMyDescription(description);
%\end{lstlisting}
The \verb+updateMyDescription+ call is used to update the description field for the current logged in user.



\rule{15cm}{2pt}
\subsection{ChangeMyPassword}
\label{Api:ChangeMyPassword}
\begin{verbatim}
   RaptureUser changeMyPassword (
           String    oldHashPassword
           String    newHashPassword
   )
\end{verbatim}
%\begin{lstlisting}[language=reflex]
%ret = #user.changeMyPassword(oldHashPassword,newHashPassword);
%\end{lstlisting}
The \verb+changeMyPassword+ call is used to change a user's password, given the current password.



\rule{15cm}{2pt}
\subsection{ChangeMyEmail}
\label{Api:ChangeMyEmail}
\begin{verbatim}
   RaptureUser changeMyEmail (
           String    newAddress
   )
\end{verbatim}
%\begin{lstlisting}[language=reflex]
%ret = #user.changeMyEmail(newAddress);
%\end{lstlisting}
The \verb+changeMyEmail+ call is used to update the email.



\rule{15cm}{2pt}
\subsection{GetServerApiVersion}
\label{Api:GetServerApiVersion}
\begin{verbatim}
   ApiVersion getServerApiVersion (
   )
\end{verbatim}
%\begin{lstlisting}[language=reflex]
%ret = #user.getServerApiVersion();
%\end{lstlisting}
The \verb+getServerAPIVersion+ call is used to return the version of the API in use at the
server and is for informational use only. If the client and server API versions are not compatible
you would not be able to make this call!



\rule{15cm}{2pt}
\subsection{RunFilterCubeView}
\label{Api:RunFilterCubeView}
\begin{verbatim}
   RaptureCubeResult runFilterCubeView (
           String    typeURI
           String    scriptURI
           String    filterParams
           String    groupFields
           String    columnFields
   )
\end{verbatim}
%\begin{lstlisting}[language=reflex]
%ret = #user.runFilterCubeView(typeURI,scriptURI,filterParams,groupFields,columnFields);
%\end{lstlisting}
\input{user/runFilterCubeView}


\rule{15cm}{2pt}
\subsection{RunNativeQuery}
\label{Api:RunNativeQuery}
\begin{verbatim}
   RaptureQueryResult runNativeQuery (
           String    typeURI
           String    repoType
           List<String>    queryParams
   )
\end{verbatim}
%\begin{lstlisting}[language=reflex]
%ret = #user.runNativeQuery(typeURI,repoType,queryParams);
%\end{lstlisting}
\input{user/runNativeQuery}


\rule{15cm}{2pt}
\subsection{RunNativeFilterCubeView}
\label{Api:RunNativeFilterCubeView}
\begin{verbatim}
   RaptureCubeResult runNativeFilterCubeView (
           String    typeURI
           String    repoType
           List<String>    queryParams
           String    groupFields
           String    columnFields
   )
\end{verbatim}
%\begin{lstlisting}[language=reflex]
%ret = #user.runNativeFilterCubeView(typeURI,repoType,queryParams,groupFields,columnFields);
%\end{lstlisting}
\input{user/runNativeFilterCubeView}


\rule{15cm}{2pt}
\subsection{IsPermitted}
\label{Api:IsPermitted}
\begin{verbatim}
   boolean isPermitted (
           String    apiCall
           String    callParam
   )
\end{verbatim}
%\begin{lstlisting}[language=reflex]
%ret = #user.isPermitted(apiCall,callParam);
%\end{lstlisting}
The \verb+isPermitted+ call tests whether a particular API call would be allowed for the current user
given the context parameter that would be passed to the call.



\rule{15cm}{2pt}

\chapter{Index API}
\index{Index API}

The document API for \Rapture is often abbreviated to \emph{Doc}. The API is used
to manipulate the presence and the content of document repositories in \Rapture.

In the abstract a document repository in \Rapture is a key/value store with optional
enhancements. The key in \Rapture corresponds to a URI for the document and where the
context is not obvious the scheme of the uri is \verb+document://+. In all document
API calls this scheme may be omitted.

Document repositories in \Rapture are backed by concrete data storage systems. When
you define a repository in \Rapture you provide a configuration string that is used
by \Rapture to route your request to a low level driver that interacts with the
underlying system. The format of this configuration string will be described in
the API call for creating a repository.

Document repositories can also be versioned. When you update a document in a
versioned repository the previous history of that document is preserved. In fact you can
qualify the URI of a document with the @ symbol and a version number to retrieve
previous versions of a document. Omitting the @ symbol will always retrieve the
latest version of a document.

Documents in repositories can also have metadata associated with them. \Rapture
automatically maintains some of this metadata - the time the document was created, the
user that created it. But a developer can use metadata update calls to add their
own attributes to documents in \Rapture.

The URI of a document in a repository implies a folder-like structure with the
forward slash delineating these folders. There are document API calls to treat a
document repository like a file system -- these are useful when constructing
browsable user interfaces to a repository.

\subsection{Methods}

\section{CreateIndex}
\index{CreateIndex}
\label{Api:CreateIndex}
\begin{lstlisting}[style=nonumbers]
   IndexConfig createIndex (
           String    indexUri
           String    config
   )
\end{lstlisting}
\begin{Verbatim}[formatcom=\color{Maroon}]
  Entitlement: /admin/index/write
\end{Verbatim}
%\begin{lstlisting}[language=reflex]
%ret = #index.createIndex(indexUri,config);
%\end{lstlisting}
The \verb+createIndex+ creates a database index on a table. The \verb+columnNames+ parameter indicates
the columns that this index should be based upon.



\rule{12cm}{2pt}
\section{GetIndex}
\index{GetIndex}
\label{Api:GetIndex}
\begin{lstlisting}[style=nonumbers]
   IndexConfig getIndex (
           String    indexUri
   )
\end{lstlisting}
\begin{Verbatim}[formatcom=\color{Maroon}]
  Entitlement: /admin/index/read
\end{Verbatim}
%\begin{lstlisting}[language=reflex]
%ret = #index.getIndex(indexUri);
%\end{lstlisting}
The \verb+getIndex+ call returns the configuration of the index, primarily for display purposes in
an operator console.



\rule{12cm}{2pt}
\section{DeleteIndex}
\index{DeleteIndex}
\label{Api:DeleteIndex}
\begin{lstlisting}[style=nonumbers]
   void deleteIndex (
           String    indexUri
   )
\end{lstlisting}
\begin{Verbatim}[formatcom=\color{Maroon}]
  Entitlement: /admin/index/write
\end{Verbatim}
%\begin{lstlisting}[language=reflex]
%ret = #index.deleteIndex(indexUri);
%\end{lstlisting}
The \verb+deleteIndex+ removes an index from \Rapture.



\rule{12cm}{2pt}
\section{CreateTable}
\index{CreateTable}
\label{Api:CreateTable}
\begin{lstlisting}[style=nonumbers]
   TableConfig createTable (
           String    tableUri
           String    config
   )
\end{lstlisting}
\begin{Verbatim}[formatcom=\color{Maroon}]
  Entitlement: /admin/index/write
\end{Verbatim}
%\begin{lstlisting}[language=reflex]
%ret = #index.createTable(tableUri,config);
%\end{lstlisting}
The \verb+createTable+ call is used to create a table on a database. The table is defined
by specifying a map of column names to types, and the table URI is hosted on the structured
database defined by the first part of the uri.

As an example, consider the following sample \Reflex code that creates a table on
the structured store referenced by the uri \verb+//test+. (This structured store reference
is created using createStructuredRepo)

\begin{Verbatim}
  columns = {};
  columns.id = 'int';
  columns.firstname = 'varchar(255)';
  columns.lastname = 'varchar(255)';
  columns.age = 'int';
  #structured.createTable('//test/table', columns);
\end{Verbatim}



\rule{12cm}{2pt}
\section{DeleteTable}
\index{DeleteTable}
\label{Api:DeleteTable}
\begin{lstlisting}[style=nonumbers]
   void deleteTable (
           String    tableUri
   )
\end{lstlisting}
\begin{Verbatim}[formatcom=\color{Maroon}]
  Entitlement: /admin/index/write
\end{Verbatim}
%\begin{lstlisting}[language=reflex]
%ret = #index.deleteTable(tableUri);
%\end{lstlisting}
The \verb+deleteTable+ call is used to remove a table (and its data) created with \verb+createTable+.



\rule{12cm}{2pt}
\section{GetTable}
\index{GetTable}
\label{Api:GetTable}
\begin{lstlisting}[style=nonumbers]
   TableConfig getTable (
           String    indexURI
   )
\end{lstlisting}
\begin{Verbatim}[formatcom=\color{Maroon}]
  Entitlement: /admin/index/read
\end{Verbatim}
%\begin{lstlisting}[language=reflex]
%ret = #index.getTable(indexURI);
%\end{lstlisting}
The \verb+getTable+ call returns the configuration of a given table.



\rule{12cm}{2pt}
\section{QueryTable}
\index{QueryTable}
\label{Api:QueryTable}
\begin{lstlisting}[style=nonumbers]
   List<TableRecord> queryTable (
           String    indexURI
           TableQuery    query
   )
\end{lstlisting}
\begin{Verbatim}[formatcom=\color{Maroon}]
  Entitlement: /admin/index/read
\end{Verbatim}
%\begin{lstlisting}[language=reflex]
%ret = #index.queryTable(indexURI,query);
%\end{lstlisting}
The \verb+queryTable+ call works in a very similar way to \verb+findIndex+ except that it is
invoked directly on the underlying table and not an index.



\rule{12cm}{2pt}
\section{FindIndex}
\index{FindIndex}
\label{Api:FindIndex}
\begin{lstlisting}[style=nonumbers]
   TableQueryResult findIndex (
           String    indexUri
           String    query
   )
\end{lstlisting}
\begin{Verbatim}[formatcom=\color{Maroon}]
  Entitlement: /admin/index/read
\end{Verbatim}
%\begin{lstlisting}[language=reflex]
%ret = #index.findIndex(indexUri,query);
%\end{lstlisting}
The \verb+findIndex+ call is used to find data in a \Rapture index. The query string is a SQL like
string that is best illustrated with an example:

\begin{Verbatim}
  SELECT endTime,
         startTime,
         priority,
         workOrderURI WHERE workOrderURI="ABC123"
\end{Verbatim}

As can be seen it is similar to SQL except that the "table" is not specified - it is provided
as the indexURI to the API call.

The return value is a result set structure.



\rule{12cm}{2pt}

\chapter{Event API}
\index{Event API}

The document API for \Rapture is often abbreviated to \emph{Doc}. The API is used
to manipulate the presence and the content of document repositories in \Rapture.

In the abstract a document repository in \Rapture is a key/value store with optional
enhancements. The key in \Rapture corresponds to a URI for the document and where the
context is not obvious the scheme of the uri is \verb+document://+. In all document
API calls this scheme may be omitted.

Document repositories in \Rapture are backed by concrete data storage systems. When
you define a repository in \Rapture you provide a configuration string that is used
by \Rapture to route your request to a low level driver that interacts with the
underlying system. The format of this configuration string will be described in
the API call for creating a repository.

Document repositories can also be versioned. When you update a document in a
versioned repository the previous history of that document is preserved. In fact you can
qualify the URI of a document with the @ symbol and a version number to retrieve
previous versions of a document. Omitting the @ symbol will always retrieve the
latest version of a document.

Documents in repositories can also have metadata associated with them. \Rapture
automatically maintains some of this metadata - the time the document was created, the
user that created it. But a developer can use metadata update calls to add their
own attributes to documents in \Rapture.

The URI of a document in a repository implies a folder-like structure with the
forward slash delineating these folders. There are document API calls to treat a
document repository like a file system -- these are useful when constructing
browsable user interfaces to a repository.

\subsection{Methods}

\section{GetEvent}
\index{GetEvent}
\label{Api:GetEvent}
\begin{lstlisting}[style=nonumbers]
   RaptureEvent getEvent (
           String    eventUri
   )
\end{lstlisting}
\begin{Verbatim}[formatcom=\color{Maroon}]
  Entitlement: /event/admin
\end{Verbatim}
%\begin{lstlisting}[language=reflex]
%ret = #event.getEvent(eventUri);
%\end{lstlisting}
The \verb+getEvent+ call returns all of the associations attached to the event.

The return value is complex structure that contains sets of the associations by category:

\begin{table}[h]
\begin{center}
\begin{tabular}{r l p{6cm}}
  Field & Type & Description \\
  \hline
  uriFullPath & string & The full uri of this event.\\
  scripts & set(RaptureEventScript) & all of the scripts attached to this event.\\
  messages & set(RaptureEventMessage) & all of the message attached to this event.\\
  notifications & set(RaptureEventNotification) & all of the notifications attached to this event.\\
  workflows & set(RaptureEventWorkflow) & all of the workflows attached to this event.\\  
\end{tabular}
\end{center}
\end{table}



\rule{12cm}{2pt}
\section{PutEvent}
\index{PutEvent}
\label{Api:PutEvent}
\begin{lstlisting}[style=nonumbers]
   void putEvent (
           RaptureEvent    event
   )
\end{lstlisting}
\begin{Verbatim}[formatcom=\color{Maroon}]
  Entitlement: /event/admin
\end{Verbatim}
%\begin{lstlisting}[language=reflex]
%ret = #event.putEvent(event);
%\end{lstlisting}
The \verb+putEvent+ call is used with the data structure returned from \verb+getEvent+. A caller can modify
that returned value and update the event manually with this method. The preferred technique is to use the \verb+add+ and
\verb+delete+ calls to manage this structure.



\rule{12cm}{2pt}
\section{DeleteEvent}
\index{DeleteEvent}
\label{Api:DeleteEvent}
\begin{lstlisting}[style=nonumbers]
   void deleteEvent (
           String    eventUri
   )
\end{lstlisting}
\begin{Verbatim}[formatcom=\color{Maroon}]
  Entitlement: /event/admin
\end{Verbatim}
%\begin{lstlisting}[language=reflex]
%ret = #event.deleteEvent(eventUri);
%\end{lstlisting}
The \verb+deleteEvent+ call removes an event as a coordination point.



\rule{12cm}{2pt}
\section{ListEventsByUriPrefix}
\index{ListEventsByUriPrefix}
\label{Api:ListEventsByUriPrefix}
\begin{lstlisting}[style=nonumbers]
   List<RaptureFolderInfo> listEventsByUriPrefix (
           String    eventUriPrefix
   )
\end{lstlisting}
\begin{Verbatim}[formatcom=\color{Maroon}]
  Entitlement: /user/get
\end{Verbatim}
%\begin{lstlisting}[language=reflex]
%ret = #event.listEventsByUriPrefix(eventUriPrefix);
%\end{lstlisting}
The \verb+listEventsByUriPrefix+ call is normally used by user interfaces that wish
to present a browser type interface on the event repository. The call returns all events
and "sub folders" (to a given depth) below a given point in the hierarchy implied
by the naming conventions used in uris. Typically an interface will use \verb+/+ as
the initial prefix and then append onto that prefix the names of either events
or folders for further \verb+listEvents+ type calls or \verb+getEvent+ if the location
maps to a real event.

The \verb+RaptureFolderInfo+ structure returned by this call is described below:

\begin{table}[ht]
  \small
\begin{center}
\begin{tabular}{r l p{8cm}}
  Field & Type & Description \\
  \hline
  name & String & The name of this element. \\
  folder & Boolean & Whether the name refers to a document or a sub-folder \\
\end{tabular}
\end{center}
\end{table}



\rule{12cm}{2pt}
\section{AddEventScript}
\index{AddEventScript}
\label{Api:AddEventScript}
\begin{lstlisting}[style=nonumbers]
   void addEventScript (
           String    eventUri
           String    scriptUri
           boolean    performOnce
   )
\end{lstlisting}
\begin{Verbatim}[formatcom=\color{Maroon}]
  Entitlement: /event/admin
\end{Verbatim}
%\begin{lstlisting}[language=reflex]
%ret = #event.addEventScript(eventUri,scriptUri,performOnce);
%\end{lstlisting}
The \verb+addEventScript+ call is used to add a \Reflex script to be executed when an event fires.

When the event is fired the script execution is placed on the internal pipeline queue so that the script
execution is scheduled alongside any other tasks running as part of workflows. The category of the task is set to
ALPHA which implies that any "general" \Rapture server running the default configuration could pick up this script and
execute it.



\rule{12cm}{2pt}
\section{DeleteEventScript}
\index{DeleteEventScript}
\label{Api:DeleteEventScript}
\begin{lstlisting}[style=nonumbers]
   void deleteEventScript (
           String    eventUri
           String    scriptUri
   )
\end{lstlisting}
\begin{Verbatim}[formatcom=\color{Maroon}]
  Entitlement: /event/admin
\end{Verbatim}
%\begin{lstlisting}[language=reflex]
%ret = #event.deleteEventScript(eventUri,scriptUri);
%\end{lstlisting}
The \verb+deleteEventScript+ removes an event association previously added using \verb+addEventScript+.



\rule{12cm}{2pt}
\section{AddEventMessage}
\index{AddEventMessage}
\label{Api:AddEventMessage}
\begin{lstlisting}[style=nonumbers]
   void addEventMessage (
           String    eventUri
           String    name
           String    pipeline
           Map<String,String>    params
   )
\end{lstlisting}
\begin{Verbatim}[formatcom=\color{Maroon}]
  Entitlement: /event/admin
\end{Verbatim}
%\begin{lstlisting}[language=reflex]
%ret = #event.addEventMessage(eventUri,name,pipeline,params);
%\end{lstlisting}
The \verb+addEventMessage+ call adds the act of sending a message down a pipeline when the event is fired. The name parameter is a unique name within this
event which can be used to remove this association by the \verb+deleteEventMessage+ call.

The act of firing an event message involves publishing a message that contains two map fields --
the \verb+context+ is the passed context from the event firing (e.g. the associatedURI of the event) and
\verb+params+ which is the settings added by this api call.



\rule{12cm}{2pt}
\section{DeleteEventMessage}
\index{DeleteEventMessage}
\label{Api:DeleteEventMessage}
\begin{lstlisting}[style=nonumbers]
   void deleteEventMessage (
           String    eventUri
           String    name
   )
\end{lstlisting}
\begin{Verbatim}[formatcom=\color{Maroon}]
  Entitlement: /event/admin
\end{Verbatim}
%\begin{lstlisting}[language=reflex]
%ret = #event.deleteEventMessage(eventUri,name);
%\end{lstlisting}
The \verb+deleteEventMessage+ removes an event association previously added using \verb+addEventMessage+.



\rule{12cm}{2pt}
\section{AddEventNotification}
\index{AddEventNotification}
\label{Api:AddEventNotification}
\begin{lstlisting}[style=nonumbers]
   void addEventNotification (
           String    eventUri
           String    name
           String    notification
           Map<String,String>    params
   )
\end{lstlisting}
\begin{Verbatim}[formatcom=\color{Maroon}]
  Entitlement: /event/admin
\end{Verbatim}
%\begin{lstlisting}[language=reflex]
%ret = #event.addEventNotification(eventUri,name,notification,params);
%\end{lstlisting}
The \verb+addEventNotification+ call is used to add a hook to publish a \Rapture notification when an event is fired.

The implementation of the event firing simply calls the Notification API \verb+publishNotification+ call with the message being a json
representation of the event message. The event message is described as part of the \verb+addEventMessage+ call.



\rule{12cm}{2pt}
\section{DeleteEventNotification}
\index{DeleteEventNotification}
\label{Api:DeleteEventNotification}
\begin{lstlisting}[style=nonumbers]
   void deleteEventNotification (
           String    eventUri
           String    name
   )
\end{lstlisting}
\begin{Verbatim}[formatcom=\color{Maroon}]
  Entitlement: /event/admin
\end{Verbatim}
%\begin{lstlisting}[language=reflex]
%ret = #event.deleteEventNotification(eventUri,name);
%\end{lstlisting}
The \verb+deleteEventNotification+ removes an event association previously added using \verb+addEventNotification+.



\rule{12cm}{2pt}
\section{AddEventWorkflow}
\index{AddEventWorkflow}
\label{Api:AddEventWorkflow}
\begin{lstlisting}[style=nonumbers]
   void addEventWorkflow (
           String    eventUri
           String    name
           String    workflowUri
           Map<String,String>    params
   )
\end{lstlisting}
\begin{Verbatim}[formatcom=\color{Maroon}]
  Entitlement: /event/admin
\end{Verbatim}
%\begin{lstlisting}[language=reflex]
%ret = #event.addEventWorkflow(eventUri,name,workflowUri,params);
%\end{lstlisting}
The \verb+addEventWorkflow+ call is used to attach a workflow to the firing of an event.

When an event is fired this workflow will be started using the Decision API \verb+runWorkflow+ call. The parameters to
the workflow will be a combination of the event context parameters and the parameters to this call.



\rule{12cm}{2pt}
\section{DeleteEventWorkflow}
\index{DeleteEventWorkflow}
\label{Api:DeleteEventWorkflow}
\begin{lstlisting}[style=nonumbers]
   void deleteEventWorkflow (
           String    eventUri
           String    name
   )
\end{lstlisting}
\begin{Verbatim}[formatcom=\color{Maroon}]
  Entitlement: /event/admin
\end{Verbatim}
%\begin{lstlisting}[language=reflex]
%ret = #event.deleteEventWorkflow(eventUri,name);
%\end{lstlisting}
The \verb+deleteEventWorkflow+ removes an event association previously added using \verb+addEventWorkflow+.



\rule{12cm}{2pt}
\section{RunEvent}
\index{RunEvent}
\label{Api:RunEvent}
\begin{lstlisting}[style=nonumbers]
   boolean runEvent (
           String    eventUri
           String    associatedUri
           String    eventContext
   )
\end{lstlisting}
\begin{Verbatim}[formatcom=\color{Maroon}]
  Entitlement: /event/admin
\end{Verbatim}
%\begin{lstlisting}[language=reflex]
%ret = #event.runEvent(eventUri,associatedUri,eventContext);
%\end{lstlisting}
The \verb+runEvent+ call fires an event and activates any associated scripts, workflows, messages and notifications.



\rule{12cm}{2pt}
\section{RunEventWithContext}
\index{RunEventWithContext}
\label{Api:RunEventWithContext}
\begin{lstlisting}[style=nonumbers]
   RunEventHandle runEventWithContext (
           String    eventUri
           String    associatedUri
           Map<String,String>    eventContextMap
   )
\end{lstlisting}
\begin{Verbatim}[formatcom=\color{Maroon}]
  Entitlement: /event/admin
\end{Verbatim}
%\begin{lstlisting}[language=reflex]
%ret = #event.runEventWithContext(eventUri,associatedUri,eventContextMap);
%\end{lstlisting}
The \verb+runEventWithContext+ call is identical to \verb+runEvent+ except that you can provide
a context that is given to any executions that happen as a result of this firing.



\rule{12cm}{2pt}
\section{EventExists}
\index{EventExists}
\label{Api:EventExists}
\begin{lstlisting}[style=nonumbers]
   boolean eventExists (
           String    eventUri
   )
\end{lstlisting}
\begin{Verbatim}[formatcom=\color{Maroon}]
  Entitlement: /event/admin
\end{Verbatim}
%\begin{lstlisting}[language=reflex]
%ret = #event.eventExists(eventUri);
%\end{lstlisting}
The \verb+eventExists+ call is a quick mechanism to determine whether a named event has any associations attached to it.



\rule{12cm}{2pt}
\section{DeleteEventsByUriPrefix}
\index{DeleteEventsByUriPrefix}
\label{Api:DeleteEventsByUriPrefix}
\begin{lstlisting}[style=nonumbers]
   List<String> deleteEventsByUriPrefix (
           String    uriPrefix
   )
\end{lstlisting}
\begin{Verbatim}[formatcom=\color{Maroon}]
  Entitlement: /data/write/$f(uriPrefix)
\end{Verbatim}
%\begin{lstlisting}[language=reflex]
%ret = #event.deleteEventsByUriPrefix(uriPrefix);
%\end{lstlisting}
The \verb+deleteEventsByUriPrefix+ call is used to remove all events below a certain
point in the hierarchy implied by the uri naming scheme. In conceptual terms it is
the equivalent of "removing the folder" from a repository.



\rule{12cm}{2pt}

\chapter{Entitlement API}
\index{Entitlement API}

The document API for \Rapture is often abbreviated to \emph{Doc}. The API is used
to manipulate the presence and the content of document repositories in \Rapture.

In the abstract a document repository in \Rapture is a key/value store with optional
enhancements. The key in \Rapture corresponds to a URI for the document and where the
context is not obvious the scheme of the uri is \verb+document://+. In all document
API calls this scheme may be omitted.

Document repositories in \Rapture are backed by concrete data storage systems. When
you define a repository in \Rapture you provide a configuration string that is used
by \Rapture to route your request to a low level driver that interacts with the
underlying system. The format of this configuration string will be described in
the API call for creating a repository.

Document repositories can also be versioned. When you update a document in a
versioned repository the previous history of that document is preserved. In fact you can
qualify the URI of a document with the @ symbol and a version number to retrieve
previous versions of a document. Omitting the @ symbol will always retrieve the
latest version of a document.

Documents in repositories can also have metadata associated with them. \Rapture
automatically maintains some of this metadata - the time the document was created, the
user that created it. But a developer can use metadata update calls to add their
own attributes to documents in \Rapture.

The URI of a document in a repository implies a folder-like structure with the
forward slash delineating these folders. There are document API calls to treat a
document repository like a file system -- these are useful when constructing
browsable user interfaces to a repository.

\subsection{Methods}

\section{GetEntitlements}
\index{GetEntitlements}
\label{Api:GetEntitlements}
\begin{lstlisting}[style=nonumbers]
   List<RaptureEntitlement> getEntitlements (
   )
\end{lstlisting}
\begin{Verbatim}[formatcom=\color{Maroon}]
  Entitlement: /admin/ent
\end{Verbatim}
%\begin{lstlisting}[language=reflex]
%ret = #entitlement.getEntitlements();
%\end{lstlisting}
This call retrieves all of the entitlements in the system.



\rule{12cm}{2pt}
\section{GetEntitlement}
\index{GetEntitlement}
\label{Api:GetEntitlement}
\begin{lstlisting}[style=nonumbers]
   RaptureEntitlement getEntitlement (
           String    entitlementName
   )
\end{lstlisting}
\begin{Verbatim}[formatcom=\color{Maroon}]
  Entitlement: /admin/ent
\end{Verbatim}
%\begin{lstlisting}[language=reflex]
%ret = #entitlement.getEntitlement(entitlementName);
%\end{lstlisting}
This call retrieves the definition of an entitlement.



\rule{12cm}{2pt}
\section{GetEntitlementByAddress}
\index{GetEntitlementByAddress}
\label{Api:GetEntitlementByAddress}
\begin{lstlisting}[style=nonumbers]
   RaptureEntitlement getEntitlementByAddress (
           String    entitlementURI
   )
\end{lstlisting}
\begin{Verbatim}[formatcom=\color{Maroon}]
  Entitlement: /admin/ent
\end{Verbatim}
%\begin{lstlisting}[language=reflex]
%ret = #entitlement.getEntitlementByAddress(entitlementURI);
%\end{lstlisting}
This call is functionally equivalent to \verb+getEntitlement+.



\rule{12cm}{2pt}
\section{GetEntitlementGroup}
\index{GetEntitlementGroup}
\label{Api:GetEntitlementGroup}
\begin{lstlisting}[style=nonumbers]
   RaptureEntitlementGroup getEntitlementGroup (
           String    groupName
   )
\end{lstlisting}
\begin{Verbatim}[formatcom=\color{Maroon}]
  Entitlement: /admin/ent
\end{Verbatim}
%\begin{lstlisting}[language=reflex]
%ret = #entitlement.getEntitlementGroup(groupName);
%\end{lstlisting}
This call retrieves an entitlement group definition (the user members).



\rule{12cm}{2pt}
\section{GetEntitlementGroupByAddress}
\index{GetEntitlementGroupByAddress}
\label{Api:GetEntitlementGroupByAddress}
\begin{lstlisting}[style=nonumbers]
   RaptureEntitlementGroup getEntitlementGroupByAddress (
           String    groupURI
   )
\end{lstlisting}
\begin{Verbatim}[formatcom=\color{Maroon}]
  Entitlement: /admin/ent
\end{Verbatim}
%\begin{lstlisting}[language=reflex]
%ret = #entitlement.getEntitlementGroupByAddress(groupURI);
%\end{lstlisting}
This call is functionally equivalent to \verb+getEntitlementGroup+.



\rule{12cm}{2pt}
\section{GetEntitlementGroups}
\index{GetEntitlementGroups}
\label{Api:GetEntitlementGroups}
\begin{lstlisting}[style=nonumbers]
   List<RaptureEntitlementGroup> getEntitlementGroups (
   )
\end{lstlisting}
\begin{Verbatim}[formatcom=\color{Maroon}]
  Entitlement: /admin/ent
\end{Verbatim}
%\begin{lstlisting}[language=reflex]
%ret = #entitlement.getEntitlementGroups();
%\end{lstlisting}
This call retrieves all of the entitlement groups defined.



\rule{12cm}{2pt}
\section{AddEntitlement}
\index{AddEntitlement}
\label{Api:AddEntitlement}
\begin{lstlisting}[style=nonumbers]
   RaptureEntitlement addEntitlement (
           String    entitlementName
           String    initialGroup
   )
\end{lstlisting}
\begin{Verbatim}[formatcom=\color{Maroon}]
  Entitlement: /admin/ent
\end{Verbatim}
%\begin{lstlisting}[language=reflex]
%ret = #entitlement.addEntitlement(entitlementName,initialGroup);
%\end{lstlisting}
The \verb+addEntitlement+ call adds an entitlement to a \Rapture system. On startup \Rapture will
create any missing "root" entitlements as defined in the api specification. This call can be used to
create specific "parameterized" entitlement definitions.



\rule{12cm}{2pt}
\section{AddGroupToEntitlement}
\index{AddGroupToEntitlement}
\label{Api:AddGroupToEntitlement}
\begin{lstlisting}[style=nonumbers]
   RaptureEntitlement addGroupToEntitlement (
           String    entitlementName
           String    groupName
   )
\end{lstlisting}
\begin{Verbatim}[formatcom=\color{Maroon}]
  Entitlement: /admin/ent
\end{Verbatim}
%\begin{lstlisting}[language=reflex]
%ret = #entitlement.addGroupToEntitlement(entitlementName,groupName);
%\end{lstlisting}
This api call adds a group to an entitlement, effectively giving all of the members of this
group that entitlement.



\rule{12cm}{2pt}
\section{RemoveGroupFromEntitlement}
\index{RemoveGroupFromEntitlement}
\label{Api:RemoveGroupFromEntitlement}
\begin{lstlisting}[style=nonumbers]
   RaptureEntitlement removeGroupFromEntitlement (
           String    entitlementName
           String    groupName
   )
\end{lstlisting}
\begin{Verbatim}[formatcom=\color{Maroon}]
  Entitlement: /admin/ent
\end{Verbatim}
%\begin{lstlisting}[language=reflex]
%ret = #entitlement.removeGroupFromEntitlement(entitlementName,groupName);
%\end{lstlisting}
This call removes a group from an entitlement, reversing an earlier \verb+addGroupToEntitlement+ call.



\rule{12cm}{2pt}
\section{DeleteEntitlement}
\index{DeleteEntitlement}
\label{Api:DeleteEntitlement}
\begin{lstlisting}[style=nonumbers]
   void deleteEntitlement (
           String    entitlementName
   )
\end{lstlisting}
\begin{Verbatim}[formatcom=\color{Maroon}]
  Entitlement: /admin/ent
\end{Verbatim}
%\begin{lstlisting}[language=reflex]
%ret = #entitlement.deleteEntitlement(entitlementName);
%\end{lstlisting}
This call removes an entitlement from the system.



\rule{12cm}{2pt}
\section{DeleteEntitlementGroup}
\index{DeleteEntitlementGroup}
\label{Api:DeleteEntitlementGroup}
\begin{lstlisting}[style=nonumbers]
   void deleteEntitlementGroup (
           String    groupName
   )
\end{lstlisting}
\begin{Verbatim}[formatcom=\color{Maroon}]
  Entitlement: /admin/ent
\end{Verbatim}
%\begin{lstlisting}[language=reflex]
%ret = #entitlement.deleteEntitlementGroup(groupName);
%\end{lstlisting}
This call removes an entitlement group from the system.



\rule{12cm}{2pt}
\section{AddEntitlementGroup}
\index{AddEntitlementGroup}
\label{Api:AddEntitlementGroup}
\begin{lstlisting}[style=nonumbers]
   RaptureEntitlementGroup addEntitlementGroup (
           String    groupName
   )
\end{lstlisting}
\begin{Verbatim}[formatcom=\color{Maroon}]
  Entitlement: /admin/ent
\end{Verbatim}
%\begin{lstlisting}[language=reflex]
%ret = #entitlement.addEntitlementGroup(groupName);
%\end{lstlisting}
The \verb+addEntitlementGroup+ call creates an initially empty entitlement group in the environment.



\rule{12cm}{2pt}
\section{AddUserToEntitlementGroup}
\index{AddUserToEntitlementGroup}
\label{Api:AddUserToEntitlementGroup}
\begin{lstlisting}[style=nonumbers]
   RaptureEntitlementGroup addUserToEntitlementGroup (
           String    groupName
           String    userName
   )
\end{lstlisting}
\begin{Verbatim}[formatcom=\color{Maroon}]
  Entitlement: /admin/ent
\end{Verbatim}
%\begin{lstlisting}[language=reflex]
%ret = #entitlement.addUserToEntitlementGroup(groupName,userName);
%\end{lstlisting}
This call adds the user to an entitlement group, giving that user any entitlements that this group
is a member of.



\rule{12cm}{2pt}
\section{RemoveUserFromEntitlementGroup}
\index{RemoveUserFromEntitlementGroup}
\label{Api:RemoveUserFromEntitlementGroup}
\begin{lstlisting}[style=nonumbers]
   RaptureEntitlementGroup removeUserFromEntitlementGroup (
           String    groupName
           String    userName
   )
\end{lstlisting}
\begin{Verbatim}[formatcom=\color{Maroon}]
  Entitlement: /admin/ent
\end{Verbatim}
%\begin{lstlisting}[language=reflex]
%ret = #entitlement.removeUserFromEntitlementGroup(groupName,userName);
%\end{lstlisting}
This call removes a user from an entitlement group, reversing an earlier \verb+addUserToEntitlementGroup+ call.



\rule{12cm}{2pt}
\section{FindEntitlementsByUser}
\index{FindEntitlementsByUser}
\label{Api:FindEntitlementsByUser}
\begin{lstlisting}[style=nonumbers]
   List<RaptureEntitlement> findEntitlementsByUser (
           String    username
   )
\end{lstlisting}
\begin{Verbatim}[formatcom=\color{Maroon}]
  Entitlement: /admin/ent
\end{Verbatim}
%\begin{lstlisting}[language=reflex]
%ret = #entitlement.findEntitlementsByUser(username);
%\end{lstlisting}
This call is identical to \verb+findEntitlementsBySelf+ except that you can provide a user to test.



\rule{12cm}{2pt}
\section{FindEntitlementsByGroup}
\index{FindEntitlementsByGroup}
\label{Api:FindEntitlementsByGroup}
\begin{lstlisting}[style=nonumbers]
   List<RaptureEntitlement> findEntitlementsByGroup (
           String    groupname
   )
\end{lstlisting}
\begin{Verbatim}[formatcom=\color{Maroon}]
  Entitlement: /admin/ent
\end{Verbatim}
%\begin{lstlisting}[language=reflex]
%ret = #entitlement.findEntitlementsByGroup(groupname);
%\end{lstlisting}
This call is used to find all of the entitlements that a given group is a member of. Primarily for use
in a UI context.



\rule{12cm}{2pt}
\section{FindEntitlementsBySelf}
\index{FindEntitlementsBySelf}
\label{Api:FindEntitlementsBySelf}
\begin{lstlisting}[style=nonumbers]
   List<RaptureEntitlement> findEntitlementsBySelf (
   )
\end{lstlisting}
\begin{Verbatim}[formatcom=\color{Maroon}]
  Entitlement: /everyone
\end{Verbatim}
%\begin{lstlisting}[language=reflex]
%ret = #entitlement.findEntitlementsBySelf();
%\end{lstlisting}
This call can be used (by a user, usually as part of a UI) to determine the entitlements of the
current user.



\rule{12cm}{2pt}

\chapter{Environment API}
\index{Environment API}

The document API for \Rapture is often abbreviated to \emph{Doc}. The API is used
to manipulate the presence and the content of document repositories in \Rapture.

In the abstract a document repository in \Rapture is a key/value store with optional
enhancements. The key in \Rapture corresponds to a URI for the document and where the
context is not obvious the scheme of the uri is \verb+document://+. In all document
API calls this scheme may be omitted.

Document repositories in \Rapture are backed by concrete data storage systems. When
you define a repository in \Rapture you provide a configuration string that is used
by \Rapture to route your request to a low level driver that interacts with the
underlying system. The format of this configuration string will be described in
the API call for creating a repository.

Document repositories can also be versioned. When you update a document in a
versioned repository the previous history of that document is preserved. In fact you can
qualify the URI of a document with the @ symbol and a version number to retrieve
previous versions of a document. Omitting the @ symbol will always retrieve the
latest version of a document.

Documents in repositories can also have metadata associated with them. \Rapture
automatically maintains some of this metadata - the time the document was created, the
user that created it. But a developer can use metadata update calls to add their
own attributes to documents in \Rapture.

The URI of a document in a repository implies a folder-like structure with the
forward slash delineating these folders. There are document API calls to treat a
document repository like a file system -- these are useful when constructing
browsable user interfaces to a repository.

\subsection{Methods}

\section{GetThisServer}
\index{GetThisServer}
\label{Api:GetThisServer}
\begin{lstlisting}[style=nonumbers]
   RaptureServerInfo getThisServer (
   )
\end{lstlisting}
\begin{Verbatim}[formatcom=\color{Maroon}]
  Entitlement: /env/common
\end{Verbatim}
%\begin{lstlisting}[language=reflex]
%ret = #environment.getThisServer();
%\end{lstlisting}
This call is used to retrieve information about the server that responded to this API call. It is often only relevant
when called from a server side context.

The server info is simply an id (usually autogenerated on \Rapture startup by the kernel) and an optional name.



\rule{12cm}{2pt}
\section{GetServers}
\index{GetServers}
\label{Api:GetServers}
\begin{lstlisting}[style=nonumbers]
   List<RaptureServerInfo> getServers (
   )
\end{lstlisting}
\begin{Verbatim}[formatcom=\color{Maroon}]
  Entitlement: /env/common
\end{Verbatim}
%\begin{lstlisting}[language=reflex]
%ret = #environment.getServers();
%\end{lstlisting}
This call returns all of the servers that are part of this \Rapture environment, as reported by them
calling \verb+setThisServer+.



\rule{12cm}{2pt}
\section{SetThisServer}
\index{SetThisServer}
\label{Api:SetThisServer}
\begin{lstlisting}[style=nonumbers]
   RaptureServerInfo setThisServer (
           RaptureServerInfo    info
   )
\end{lstlisting}
\begin{Verbatim}[formatcom=\color{Maroon}]
  Entitlement: /env/admin
\end{Verbatim}
%\begin{lstlisting}[language=reflex]
%ret = #environment.setThisServer(info);
%\end{lstlisting}
The \verb+setThisServer+ api call is primarily used by the \Rapture kernel upon startup.



\rule{12cm}{2pt}
\section{SetApplianceMode}
\index{SetApplianceMode}
\label{Api:SetApplianceMode}
\begin{lstlisting}[style=nonumbers]
   void setApplianceMode (
           boolean    mode
   )
\end{lstlisting}
\begin{Verbatim}[formatcom=\color{Maroon}]
  Entitlement: /env/admin
\end{Verbatim}
%\begin{lstlisting}[language=reflex]
%ret = #environment.setApplianceMode(mode);
%\end{lstlisting}
The \verb+setApplianceMode+ sets the appliance mode flag in \Rapture, which can be
interpreted by \Rapture servers to enforce a single user mode.



\rule{12cm}{2pt}
\section{GetApplianceMode}
\index{GetApplianceMode}
\label{Api:GetApplianceMode}
\begin{lstlisting}[style=nonumbers]
   boolean getApplianceMode (
   )
\end{lstlisting}
\begin{Verbatim}[formatcom=\color{Maroon}]
  Entitlement: /env/common
\end{Verbatim}
%\begin{lstlisting}[language=reflex]
%ret = #environment.getApplianceMode();
%\end{lstlisting}
This call retrieves the current appliance mode setting.



\rule{12cm}{2pt}
\section{GetServerStatus}
\index{GetServerStatus}
\label{Api:GetServerStatus}
\begin{lstlisting}[style=nonumbers]
   List<RaptureServerStatus> getServerStatus (
   )
\end{lstlisting}
\begin{Verbatim}[formatcom=\color{Maroon}]
  Entitlement: /env/common
\end{Verbatim}
%\begin{lstlisting}[language=reflex]
%ret = #environment.getServerStatus();
%\end{lstlisting}
This call returns the status of each \Rapture server in the environment.



\rule{12cm}{2pt}

\chapter{Lock API}
\index{Lock API}

The document API for \Rapture is often abbreviated to \emph{Doc}. The API is used
to manipulate the presence and the content of document repositories in \Rapture.

In the abstract a document repository in \Rapture is a key/value store with optional
enhancements. The key in \Rapture corresponds to a URI for the document and where the
context is not obvious the scheme of the uri is \verb+document://+. In all document
API calls this scheme may be omitted.

Document repositories in \Rapture are backed by concrete data storage systems. When
you define a repository in \Rapture you provide a configuration string that is used
by \Rapture to route your request to a low level driver that interacts with the
underlying system. The format of this configuration string will be described in
the API call for creating a repository.

Document repositories can also be versioned. When you update a document in a
versioned repository the previous history of that document is preserved. In fact you can
qualify the URI of a document with the @ symbol and a version number to retrieve
previous versions of a document. Omitting the @ symbol will always retrieve the
latest version of a document.

Documents in repositories can also have metadata associated with them. \Rapture
automatically maintains some of this metadata - the time the document was created, the
user that created it. But a developer can use metadata update calls to add their
own attributes to documents in \Rapture.

The URI of a document in a repository implies a folder-like structure with the
forward slash delineating these folders. There are document API calls to treat a
document repository like a file system -- these are useful when constructing
browsable user interfaces to a repository.

\subsection{Methods}

\section{GetLockManagerConfigs}
\index{GetLockManagerConfigs}
\label{Api:GetLockManagerConfigs}
\begin{lstlisting}[style=nonumbers]
   List<RaptureLockConfig> getLockManagerConfigs (
           String    managerUri
   )
\end{lstlisting}
\begin{Verbatim}[formatcom=\color{Maroon}]
  Entitlement: /admin/lock
\end{Verbatim}
%\begin{lstlisting}[language=reflex]
%ret = #lock.getLockManagerConfigs(managerUri);
%\end{lstlisting}
This call retrieves all of the lock manager configurations available on this environment.



\rule{12cm}{2pt}
\section{CreateLockManager}
\index{CreateLockManager}
\label{Api:CreateLockManager}
\begin{lstlisting}[style=nonumbers]
   RaptureLockConfig createLockManager (
           String    managerUri
           String    config
           String    pathPosition
   )
\end{lstlisting}
\begin{Verbatim}[formatcom=\color{Maroon}]
  Entitlement: /admin/lock
\end{Verbatim}
%\begin{lstlisting}[language=reflex]
%ret = #lock.createLockManager(managerUri,config,pathPosition);
%\end{lstlisting}
This call creates a named lock manager in \Rapture. The configuration string is
in fact a complex instruction written in a lock domain specific language (DSL) that is used to define the
capabilities and underlying implementation of the repository.

The typical configuration string for a locking provider backed by MongoDB is reproduced below:

\begin{Verbatim}
LOCK {} USING MONGODB { prefix = 'test' }
\end{Verbatim}

The general form of the configuration is:

\begin{Verbatim}
LOCK { }
     USING [underlying implementation] { [ config ]}
     [ ON [ instance] ]
\end{Verbatim}

The second part of the configuration string defines the underlying implementation and its configuration. In
most cases the configuration associated with the implementation has a \verb+prefix+ parameter that is used to
define a table or a collection or a prefix for such entities in the underlying storage. The underlying implementation
defines what lower level software is used to host the data managed by \Rapture. The following table shows the current
implementations:

\begin{table}[h]
\begin{center}
\begin{tabular}{r l p{8cm}}
  Keyword & Underlying & Configuration \\
  \hline
  MONGODB & MongoDb & The prefix parameter defines the name of the collections used by this repository. To avoid
  conflict this is usually a function of the name of the \Rapture repository. \\
  CASSANDRA & Cassandra & The prefix parameter defines the name of the collections used by this repository. To avoid
  conflict this is usually a function of the name of the \Rapture repository. \\
\end{tabular}
\end{center}
\end{table}



\rule{12cm}{2pt}
\section{LockManagerExists}
\index{LockManagerExists}
\label{Api:LockManagerExists}
\begin{lstlisting}[style=nonumbers]
   boolean lockManagerExists (
           String    managerUri
   )
\end{lstlisting}
\begin{Verbatim}[formatcom=\color{Maroon}]
  Entitlement: /admin/lock
\end{Verbatim}
%\begin{lstlisting}[language=reflex]
%ret = #lock.lockManagerExists(managerUri);
%\end{lstlisting}
This call is simply used to test whether a given lock manager is defined on the system. It is often
used in setup scripts.



\rule{12cm}{2pt}
\section{GetLockManagerConfig}
\index{GetLockManagerConfig}
\label{Api:GetLockManagerConfig}
\begin{lstlisting}[style=nonumbers]
   RaptureLockConfig getLockManagerConfig (
           String    managerUri
   )
\end{lstlisting}
\begin{Verbatim}[formatcom=\color{Maroon}]
  Entitlement: /admin/lock/$f(managerUri)
\end{Verbatim}
%\begin{lstlisting}[language=reflex]
%ret = #lock.getLockManagerConfig(managerUri);
%\end{lstlisting}
This call retrieves the configuration associated with a named lock manager.



\rule{12cm}{2pt}
\section{DeleteLockManager}
\index{DeleteLockManager}
\label{Api:DeleteLockManager}
\begin{lstlisting}[style=nonumbers]
   void deleteLockManager (
           String    managerUri
   )
\end{lstlisting}
\begin{Verbatim}[formatcom=\color{Maroon}]
  Entitlement: /admin/lock/$f(managerUri)
\end{Verbatim}
%\begin{lstlisting}[language=reflex]
%ret = #lock.deleteLockManager(managerUri);
%\end{lstlisting}
This call is used to remove a lock provider previously created by \verb+createLockManager+.



\rule{12cm}{2pt}
\section{AcquireLock}
\index{AcquireLock}
\label{Api:AcquireLock}
\begin{lstlisting}[style=nonumbers]
   LockHandle acquireLock (
           String    managerUri
           String    lockName
           long    secondsToWait
           long    secondsToKeep
   )
\end{lstlisting}
\begin{Verbatim}[formatcom=\color{Maroon}]
  Entitlement: /admin/lock/$f(managerUri)
\end{Verbatim}
%\begin{lstlisting}[language=reflex]
%ret = #lock.acquireLock(managerUri,lockName,secondsToWait,secondsToKeep);
%\end{lstlisting}
This call is used to acquire a lock. The lock is "hosted" by a lock manager and has a path within that manager. Other parameters determine
how long the api call should block until returning with a failed attempt to lock. Another parameter
defines the default amount of time this lock will be held for before being auto released.

If the call is successful a \verb+LockHandle+ is returned that can be used in the \verb+releaseLock+ call. If a lock is stuck the \verb+forceReleaseLock+ call can be used
to remove it.



\rule{12cm}{2pt}
\section{ReleaseLock}
\index{ReleaseLock}
\label{Api:ReleaseLock}
\begin{lstlisting}[style=nonumbers]
   boolean releaseLock (
           String    managerUri
           String    lockName
           LockHandle    lockHandle
   )
\end{lstlisting}
\begin{Verbatim}[formatcom=\color{Maroon}]
  Entitlement: /admin/lock/$f(managerUri)
\end{Verbatim}
%\begin{lstlisting}[language=reflex]
%ret = #lock.releaseLock(managerUri,lockName,lockHandle);
%\end{lstlisting}
The \verb+releaseLock+ call is used to free a lock previously obtained using \verb+acquireLock+.



\rule{12cm}{2pt}
\section{ForceReleaseLock}
\index{ForceReleaseLock}
\label{Api:ForceReleaseLock}
\begin{lstlisting}[style=nonumbers]
   void forceReleaseLock (
           String    managerUri
           String    lockName
   )
\end{lstlisting}
\begin{Verbatim}[formatcom=\color{Maroon}]
  Entitlement: /admin/lock/$f(managerUri)
\end{Verbatim}
%\begin{lstlisting}[language=reflex]
%ret = #lock.forceReleaseLock(managerUri,lockName);
%\end{lstlisting}
The \verb+forceReleaseLock+ is used to remove a lock from the system without holding the lock in the first place.



\rule{12cm}{2pt}

\section{IdGen API}

The document API for \Rapture is often abbreviated to \emph{Doc}. The API is used
to manipulate the presence and the content of document repositories in \Rapture.

In the abstract a document repository in \Rapture is a key/value store with optional
enhancements. The key in \Rapture corresponds to a URI for the document and where the
context is not obvious the scheme of the uri is \verb+document://+. In all document
API calls this scheme may be omitted.

Document repositories in \Rapture are backed by concrete data storage systems. When
you define a repository in \Rapture you provide a configuration string that is used
by \Rapture to route your request to a low level driver that interacts with the
underlying system. The format of this configuration string will be described in
the API call for creating a repository.

Document repositories can also be versioned. When you update a document in a
versioned repository the previous history of that document is preserved. In fact you can
qualify the URI of a document with the @ symbol and a version number to retrieve
previous versions of a document. Omitting the @ symbol will always retrieve the
latest version of a document.

Documents in repositories can also have metadata associated with them. \Rapture
automatically maintains some of this metadata - the time the document was created, the
user that created it. But a developer can use metadata update calls to add their
own attributes to documents in \Rapture.

The URI of a document in a repository implies a folder-like structure with the
forward slash delineating these folders. There are document API calls to treat a
document repository like a file system -- these are useful when constructing
browsable user interfaces to a repository.

\subsection{Methods}

\subsubsection{GetIdGenConfigs}
\label{Api:GetIdGenConfigs}
\begin{verbatim}
   List<RaptureIdGenConfig> getIdGenConfigs (
           String    authority
   )
\end{verbatim}
\begin{lstlisting}[language=reflex]
// Reflex use
ret = #idgen.getIdGenConfigs(authority);
\end{lstlisting}
This call is used to retrieve all of the ID generator providers registered in the system.



\rule{15cm}{2pt}
\subsubsection{GetIdGenConfig}
\label{Api:GetIdGenConfig}
\begin{verbatim}
   RaptureIdGenConfig getIdGenConfig (
           String    idGenUri
   )
\end{verbatim}
\begin{lstlisting}[language=reflex]
// Reflex use
ret = #idgen.getIdGenConfig(idGenUri);
\end{lstlisting}
This call can retrieve the configuration of a given id generator provider.



\rule{15cm}{2pt}
\subsubsection{CreateIdGen}
\label{Api:CreateIdGen}
\begin{verbatim}
   RaptureIdGenConfig createIdGen (
           String    idGenUri
           String    config
   )
\end{verbatim}
\begin{lstlisting}[language=reflex]
// Reflex use
ret = #idgen.createIdGen(idGenUri,config);
\end{lstlisting}
The \verb+createIdGen+ call is used to define an ID generator source.

The typical configuration string for a locking provider backed by MongoDB is reproduced below:

\begin{Verbatim}
IDGEN { initial="1",
        base="36",
        length="8",
        prefix="ID" } USING MONGODB { prefix = 'test' }
\end{Verbatim}

The general form of the configuration is:

\begin{Verbatim}
IDGEN { config }
     USING [underlying implementation] { [ config ]}
     [ ON [ instance] ]
\end{Verbatim}

The first part of the configuration defines the way in which the provider will generate ids. The fields
and the meanings are reproduced below:

\begin{table}[h]
\begin{center}
\begin{tabular}{r l p{8cm}}
  Field & Example & Description \\
  \hline
  initial & 1 & The id number of the first id to be produced by this provider. \\
  base & 36 & The base of the number system to be used by the provider. Base 36 implies letters and digits. \\
  length & 8 & The width of the id. The id will be padded with zeroes to this length. \\
  prefix & ID & The id will be prefixed with this string. \\
\end{tabular}
\end{center}
\end{table}

As an example, this producer configuration would produce the following ids:

\begin{Verbatim}
  ID00000199
  ID0000019A
  ID0000019B
\end{Verbatim}

The second part of the configuration string defines the underlying implementation and its configuration. In
most cases the configuration associated with the implementation has a \verb+prefix+ parameter that is used to
define a table or a collection or a prefix for such entities in the underlying storage. The underlying implementation
defines what lower level software is used to host the data managed by \Rapture. The following table shows the current
implementations:

\begin{table}[h]
\begin{center}
\begin{tabular}{r l p{8cm}}
  Keyword & Underlying & Configuration \\
  \hline
  MONGODB & MongoDb & The prefix parameter defines the name of the collections used by this repository. To avoid
  conflict this is usually a function of the name of the \Rapture repository. \\
  CASSANDRA & Cassandra & The prefix parameter defines the name of the collections used by this repository. To avoid
  conflict this is usually a function of the name of the \Rapture repository. \\
\end{tabular}
\end{center}
\end{table}



\rule{15cm}{2pt}
\subsubsection{IdGenExists}
\label{Api:IdGenExists}
\begin{verbatim}
   boolean idGenExists (
           String    idGenUri
   )
\end{verbatim}
\begin{lstlisting}[language=reflex]
// Reflex use
ret = #idgen.idGenExists(idGenUri);
\end{lstlisting}
This call can be used to determine whether a given ID generator exists or not. It is primarily used
in installation scripts.



\rule{15cm}{2pt}
\subsubsection{DeleteIdGen}
\label{Api:DeleteIdGen}
\begin{verbatim}
   void deleteIdGen (
           String    idGenUri
   )
\end{verbatim}
\begin{lstlisting}[language=reflex]
// Reflex use
ret = #idgen.deleteIdGen(idGenUri);
\end{lstlisting}
This call is used to remove an Id generator from the system.



\rule{15cm}{2pt}
\subsubsection{SetIdGen}
\label{Api:SetIdGen}
\begin{verbatim}
   void setIdGen (
           String    idGenUri
           Long    count
   )
\end{verbatim}
\begin{lstlisting}[language=reflex]
// Reflex use
ret = #idgen.setIdGen(idGenUri,count);
\end{lstlisting}
The \verb+setIdGen+ resets an id generator to the given count.



\rule{15cm}{2pt}
\subsubsection{Next}
\label{Api:Next}
\begin{verbatim}
   String next (
           String    idGenUri
   )
\end{verbatim}
\begin{lstlisting}[language=reflex]
// Reflex use
ret = #idgen.next(idGenUri);
\end{lstlisting}
This call returns the next unique ID from the given provider.



\rule{15cm}{2pt}
\subsubsection{NextIds}
\label{Api:NextIds}
\begin{verbatim}
   String nextIds (
           String    idGenUri
           Long    num
   )
\end{verbatim}
\begin{lstlisting}[language=reflex]
// Reflex use
ret = #idgen.nextIds(idGenUri,num);
\end{lstlisting}
The \verb+nextIds+ call increments the id generator by the count given and then returns the unique id it represents.

The \verb+next+ call can be likened to \verb+nextId(1)+.



\rule{15cm}{2pt}
\subsubsection{SetupDefaultIdGens}
\label{Api:SetupDefaultIdGens}
\begin{verbatim}
   void setupDefaultIdGens (
           boolean    force
   )
\end{verbatim}
\begin{lstlisting}[language=reflex]
// Reflex use
ret = #idgen.setupDefaultIdGens(force);
\end{lstlisting}
This call is used internally by \Rapture to setup the default id generators needed by the system.

Default generators are used for the ids of work orders (workflow runs).



\rule{15cm}{2pt}

\chapter{Activity API}
\index{Activity API}

The document API for \Rapture is often abbreviated to \emph{Doc}. The API is used
to manipulate the presence and the content of document repositories in \Rapture.

In the abstract a document repository in \Rapture is a key/value store with optional
enhancements. The key in \Rapture corresponds to a URI for the document and where the
context is not obvious the scheme of the uri is \verb+document://+. In all document
API calls this scheme may be omitted.

Document repositories in \Rapture are backed by concrete data storage systems. When
you define a repository in \Rapture you provide a configuration string that is used
by \Rapture to route your request to a low level driver that interacts with the
underlying system. The format of this configuration string will be described in
the API call for creating a repository.

Document repositories can also be versioned. When you update a document in a
versioned repository the previous history of that document is preserved. In fact you can
qualify the URI of a document with the @ symbol and a version number to retrieve
previous versions of a document. Omitting the @ symbol will always retrieve the
latest version of a document.

Documents in repositories can also have metadata associated with them. \Rapture
automatically maintains some of this metadata - the time the document was created, the
user that created it. But a developer can use metadata update calls to add their
own attributes to documents in \Rapture.

The URI of a document in a repository implies a folder-like structure with the
forward slash delineating these folders. There are document API calls to treat a
document repository like a file system -- these are useful when constructing
browsable user interfaces to a repository.

\subsection{Methods}

\subsection{CreateActivity}
\index{CreateActivity}
\label{Api:CreateActivity}
\begin{Verbatim}
   String createActivity (
           String    description
           String    message
           Long    progress
           Long    max
   )
\end{Verbatim}
\begin{Verbatim}[formatcom=\color{Maroon}]
  Entitlement: /activity/write
\end{Verbatim}
%\begin{lstlisting}[language=reflex]
%ret = #activity.createActivity(description,message,progress,max);
%\end{lstlisting}
The \verb+createActivity+ call is used to regster a new activity. The
return value is an activity id that can be used in subsequent calls. The
\verb+progress+ and \verb+max+ parameters can be used to indicate some measure of
completion.



\rule{12cm}{2pt}
\subsection{UpdateActivity}
\index{UpdateActivity}
\label{Api:UpdateActivity}
\begin{Verbatim}
   boolean updateActivity (
           String    activityId
           String    message
           Long    progress
           Long    max
   )
\end{Verbatim}
\begin{Verbatim}[formatcom=\color{Maroon}]
  Entitlement: /activity/write
\end{Verbatim}
%\begin{lstlisting}[language=reflex]
%ret = #activity.updateActivity(activityId,message,progress,max);
%\end{lstlisting}
The \verb+updateActivity+ call takes an existing \verb+activityId+ returned by
the \verb+createActivity+ call and updates its status with respect to how complete it is.
The message parameter can be used to pass status messages to viewing consoles.

The return value indicates whether an external process has requested that this activity
should be aborted (by using the \verb+requestAbortActivity+ call). It is entirely at the processes
discretion to actually abort based on a request.



\rule{12cm}{2pt}
\subsection{FinishActivity}
\index{FinishActivity}
\label{Api:FinishActivity}
\begin{Verbatim}
   boolean finishActivity (
           String    activityId
           String    message
   )
\end{Verbatim}
\begin{Verbatim}[formatcom=\color{Maroon}]
  Entitlement: /activity/write
\end{Verbatim}
%\begin{lstlisting}[language=reflex]
%ret = #activity.finishActivity(activityId,message);
%\end{lstlisting}
The \verb+finishActivity+ call is used to mark an activity as completed. The \verb+activityId+ parameter
is that returned by the \verb+createActivity+ call.



\rule{12cm}{2pt}
\subsection{RequestAbortActivity}
\index{RequestAbortActivity}
\label{Api:RequestAbortActivity}
\begin{Verbatim}
   boolean requestAbortActivity (
           String    activityId
           String    message
   )
\end{Verbatim}
\begin{Verbatim}[formatcom=\color{Maroon}]
  Entitlement: /activity/write
\end{Verbatim}
%\begin{lstlisting}[language=reflex]
%ret = #activity.requestAbortActivity(activityId,message);
%\end{lstlisting}
The \verb+requestAbortActivity+ requests that an activity should be aborted. The request is passed
along to an activity through the \verb+updateActivity+ call -- and therefore
it is encumbent on a process using the activity API to call this method periodically.



\rule{12cm}{2pt}
\subsection{QueryByExpiryTime}
\index{QueryByExpiryTime}
\label{Api:QueryByExpiryTime}
\begin{Verbatim}
   ActivityQueryResponse queryByExpiryTime (
           String    nextBatchId
           Long    batchSize
           Long    lastSeen
   )
\end{Verbatim}
\begin{Verbatim}[formatcom=\color{Maroon}]
  Entitlement: /activity/read
\end{Verbatim}
%\begin{lstlisting}[language=reflex]
%ret = #activity.queryByExpiryTime(nextBatchId,batchSize,lastSeen);
%\end{lstlisting}
The \verb+queryByExpiryTIme+ can be used to request changes to activity statuses.

The \verb+nextBatchId+ can be set to a zero length string initially -- the id used for
the next call to this method can be obtained from the \verb+nextBatchId+ field of the return value.

The \verb+activities+ field of the return value contains the updated activities since the last call.



\rule{12cm}{2pt}
\subsection{GetById}
\index{GetById}
\label{Api:GetById}
\begin{Verbatim}
   Activity getById (
           String    activityId
   )
\end{Verbatim}
\begin{Verbatim}[formatcom=\color{Maroon}]
  Entitlement: /activity/read
\end{Verbatim}
%\begin{lstlisting}[language=reflex]
%ret = #activity.getById(activityId);
%\end{lstlisting}
The \verb+getById+ call is used to return the details of an activity given its id.



\rule{12cm}{2pt}

\chapter{Async API}
\index{Async API}

The document API for \Rapture is often abbreviated to \emph{Doc}. The API is used
to manipulate the presence and the content of document repositories in \Rapture.

In the abstract a document repository in \Rapture is a key/value store with optional
enhancements. The key in \Rapture corresponds to a URI for the document and where the
context is not obvious the scheme of the uri is \verb+document://+. In all document
API calls this scheme may be omitted.

Document repositories in \Rapture are backed by concrete data storage systems. When
you define a repository in \Rapture you provide a configuration string that is used
by \Rapture to route your request to a low level driver that interacts with the
underlying system. The format of this configuration string will be described in
the API call for creating a repository.

Document repositories can also be versioned. When you update a document in a
versioned repository the previous history of that document is preserved. In fact you can
qualify the URI of a document with the @ symbol and a version number to retrieve
previous versions of a document. Omitting the @ symbol will always retrieve the
latest version of a document.

Documents in repositories can also have metadata associated with them. \Rapture
automatically maintains some of this metadata - the time the document was created, the
user that created it. But a developer can use metadata update calls to add their
own attributes to documents in \Rapture.

The URI of a document in a repository implies a folder-like structure with the
forward slash delineating these folders. There are document API calls to treat a
document repository like a file system -- these are useful when constructing
browsable user interfaces to a repository.

\subsection{Methods}

\subsection{AsyncReflexScript}
\index{AsyncReflexScript}
\label{Api:AsyncReflexScript}
\begin{Verbatim}
   String asyncReflexScript (
           String    reflexScript
           Map<String,String>    parameters
   )
\end{Verbatim}
\begin{Verbatim}[formatcom=\color{Maroon}]
  Entitlement: /async/reflex
\end{Verbatim}
%\begin{lstlisting}[language=reflex]
%ret = #async.asyncReflexScript(reflexScript,parameters);
%\end{lstlisting}
The \verb+asyncReflexScript+ call instructs \Rapture to invoke a \Reflex script asynchronously.

The script itself is passed as the first parameter to this call, unlike the \verb+asyncReflexReference+ call which
takes a script URI.

The second parameter is used to pass parameters to this invocation. The script is run using the calling context
of the user that invoked this API call.

The return value is a status id (handle) that can be used with the \verb+asyncStatus+ call.



\rule{12cm}{2pt}
\subsection{AsyncReflexReference}
\index{AsyncReflexReference}
\label{Api:AsyncReflexReference}
\begin{Verbatim}
   String asyncReflexReference (
           String    scriptURI
           Map<String,String>    parameters
   )
\end{Verbatim}
\begin{Verbatim}[formatcom=\color{Maroon}]
  Entitlement: /async/reflex
\end{Verbatim}
%\begin{lstlisting}[language=reflex]
%ret = #async.asyncReflexReference(scriptURI,parameters);
%\end{lstlisting}
The \verb+asyncReflexReference+ call instructs \Rapture to invoke a \Reflex script asynchronously.

The second parameter is used to pass parameters to this invocation. The script is run using the calling context
of the user that invoked this API call.

The return value is a status id (handle) that can be used with the \verb+asyncStatus+ call.



\rule{12cm}{2pt}
\subsection{AsyncStatus}
\index{AsyncStatus}
\label{Api:AsyncStatus}
\begin{Verbatim}
   WorkOrderStatus asyncStatus (
           String    taskId
   )
\end{Verbatim}
\begin{Verbatim}[formatcom=\color{Maroon}]
  Entitlement: /admin/main
\end{Verbatim}
%\begin{lstlisting}[language=reflex]
%ret = #async.asyncStatus(taskId);
%\end{lstlisting}
The \verb+asyncStatus+ call returns the current status of the task returned by either
\verb+asyncReflexScript+ or \verb+asyncReflexReference+ calls.

The \verb+workOrderStatus+ return type simply contains the list of internal worker ids
(which in this case should be of length 1, depending on the \Reflex script itself) and
the current status of this async request.

Given the workerId the appropriate Decision API calls can be invoked for more detailed
analysis.



\rule{12cm}{2pt}
\subsection{SetupDefaultWorkflows}
\index{SetupDefaultWorkflows}
\label{Api:SetupDefaultWorkflows}
\begin{Verbatim}
   void setupDefaultWorkflows (
           boolean    force
   )
\end{Verbatim}
\begin{Verbatim}[formatcom=\color{Maroon}]
  Entitlement: /admin/async
\end{Verbatim}
%\begin{lstlisting}[language=reflex]
%ret = #async.setupDefaultWorkflows(force);
%\end{lstlisting}
The \verb+setupDefaultWorkflows+ call is used internally during \Rapture initialization to ensure
that the capability exists to run async calls.



\rule{12cm}{2pt}

\chapter{Audit API}
\index{Audit API}

The document API for \Rapture is often abbreviated to \emph{Doc}. The API is used
to manipulate the presence and the content of document repositories in \Rapture.

In the abstract a document repository in \Rapture is a key/value store with optional
enhancements. The key in \Rapture corresponds to a URI for the document and where the
context is not obvious the scheme of the uri is \verb+document://+. In all document
API calls this scheme may be omitted.

Document repositories in \Rapture are backed by concrete data storage systems. When
you define a repository in \Rapture you provide a configuration string that is used
by \Rapture to route your request to a low level driver that interacts with the
underlying system. The format of this configuration string will be described in
the API call for creating a repository.

Document repositories can also be versioned. When you update a document in a
versioned repository the previous history of that document is preserved. In fact you can
qualify the URI of a document with the @ symbol and a version number to retrieve
previous versions of a document. Omitting the @ symbol will always retrieve the
latest version of a document.

Documents in repositories can also have metadata associated with them. \Rapture
automatically maintains some of this metadata - the time the document was created, the
user that created it. But a developer can use metadata update calls to add their
own attributes to documents in \Rapture.

The URI of a document in a repository implies a folder-like structure with the
forward slash delineating these folders. There are document API calls to treat a
document repository like a file system -- these are useful when constructing
browsable user interfaces to a repository.

\subsection{Methods}

\section{Setup}
\index{Setup}
\label{Api:Setup}
\begin{lstlisting}[style=nonumbers]
   void setup (
           boolean    force
   )
\end{lstlisting}
\begin{Verbatim}[formatcom=\color{Maroon}]
  Entitlement: /audit/admin
\end{Verbatim}
%\begin{lstlisting}[language=reflex]
%ret = #audit.setup(force);
%\end{lstlisting}
[bdist_wheel]
# Set universal=1 if and when we support Python 2 and Python 3.
# As it is I think we only support Python 2
universal=0



\rule{12cm}{2pt}
\section{CreateAuditLog}
\index{CreateAuditLog}
\label{Api:CreateAuditLog}
\begin{lstlisting}[style=nonumbers]
   void createAuditLog (
           String    name
           String    config
   )
\end{lstlisting}
\begin{Verbatim}[formatcom=\color{Maroon}]
  Entitlement: /audit/admin
\end{Verbatim}
%\begin{lstlisting}[language=reflex]
%ret = #audit.createAuditLog(name,config);
%\end{lstlisting}
The \verb+createAuditLog+ call is used to define an audit log in \Rapture.

The parameters to the call
look straightforward -- simply the name of the new audit log and a configuration string. The configuration string is
in fact a complex instruction written in a audit log domain specific language (DSL) that is used to define the
capabilities and underlying implementation of the repository.

The typical configuration string for a versioned repository backed by LOG4J is reproduced below:

\begin{Verbatim}
LOG {} USING LOG4J { }
\end{Verbatim}

The general form of the configuration is:

\begin{Verbatim}
LOG {  }
     USING [underlying implementation] { [ config ]}
     [ ON [ instance] ]
\end{Verbatim}

The second part of the configuration string defines the underlying implementation and its configuration. In
most cases the configuration associated with the implementation has a \verb+prefix+ parameter that is used to
define a table or a collection or a prefix for such entities in the underlying storage. The underlying implementation
defines what lower level software is used to host the data managed by \Rapture. The following table shows the current
implementations:

\begin{table}[h]
\begin{center}
\begin{tabular}{r l p{8cm}}
  Keyword & Underlying & Configuration \\
  \hline
  MONGODB & MongoDb & The prefix parameter defines the name of the collections used by this repository. To avoid
  conflict this is usually a function of the name of the \Rapture repository. \\
  LOG4J & LOG4J & The logs are written to a log4j appender. \\
\end{tabular}
\end{center}
\end{table}



\rule{12cm}{2pt}
\section{DoesAuditLogExist}
\index{DoesAuditLogExist}
\label{Api:DoesAuditLogExist}
\begin{lstlisting}[style=nonumbers]
   boolean doesAuditLogExist (
           String    logURI
   )
\end{lstlisting}
\begin{Verbatim}[formatcom=\color{Maroon}]
  Entitlement: /audit/main
\end{Verbatim}
%\begin{lstlisting}[language=reflex]
%ret = #audit.doesAuditLogExist(logURI);
%\end{lstlisting}
The \verb+doesAuditLogExist+ call is used to test for the existence of an audit log. It is often used
as part of a setup/configuration script.



\rule{12cm}{2pt}
\section{GetChildren}
\index{GetChildren}
\label{Api:GetChildren}
\begin{lstlisting}[style=nonumbers]
   List<RaptureFolderInfo> getChildren (
           String    prefix
   )
\end{lstlisting}
\begin{Verbatim}[formatcom=\color{Maroon}]
  Entitlement: /audit/main
\end{Verbatim}
%\begin{lstlisting}[language=reflex]
%ret = #audit.getChildren(prefix);
%\end{lstlisting}
The \verb+getChildren+ call can be used by User Interfaces that wish to enumerate the audit logs
in a system.



\rule{12cm}{2pt}
\section{DeleteAuditLog}
\index{DeleteAuditLog}
\label{Api:DeleteAuditLog}
\begin{lstlisting}[style=nonumbers]
   void deleteAuditLog (
           String    logURI
   )
\end{lstlisting}
\begin{Verbatim}[formatcom=\color{Maroon}]
  Entitlement: /audit/admin
\end{Verbatim}
%\begin{lstlisting}[language=reflex]
%ret = #audit.deleteAuditLog(logURI);
%\end{lstlisting}
The \verb+deleteAuditLog+ call is used to remove an audit log provider from the system, previously
created using \verb+createAuditLog+.



\rule{12cm}{2pt}
\section{GetAuditLog}
\index{GetAuditLog}
\label{Api:GetAuditLog}
\begin{lstlisting}[style=nonumbers]
   AuditLogConfig getAuditLog (
           String    logURI
   )
\end{lstlisting}
\begin{Verbatim}[formatcom=\color{Maroon}]
  Entitlement: /audit/admin
\end{Verbatim}
%\begin{lstlisting}[language=reflex]
%ret = #audit.getAuditLog(logURI);
%\end{lstlisting}
The \verb+getAuditLog+ call is used to retrieve the configuration for an audit log.



\rule{12cm}{2pt}
\section{WriteAuditEntry}
\index{WriteAuditEntry}
\label{Api:WriteAuditEntry}
\begin{lstlisting}[style=nonumbers]
   void writeAuditEntry (
           String    logURI
           String    category
           int    level
           String    message
   )
\end{lstlisting}
\begin{Verbatim}[formatcom=\color{Maroon}]
  Entitlement: /audit/admin
\end{Verbatim}
%\begin{lstlisting}[language=reflex]
%ret = #audit.writeAuditEntry(logURI,category,level,message);
%\end{lstlisting}
The \verb+writeAuditEntry+ call is used to record an audit entry to an audit log. The parameters
can be used to categorize or colorize entries in a UI.



\rule{12cm}{2pt}
\section{WriteAuditEntryData}
\index{WriteAuditEntryData}
\label{Api:WriteAuditEntryData}
\begin{lstlisting}[style=nonumbers]
   void writeAuditEntryData (
           String    logURI
           String    category
           int    level
           String    message
           Map<String,Object>    data
   )
\end{lstlisting}
\begin{Verbatim}[formatcom=\color{Maroon}]
  Entitlement: /audit/admin
\end{Verbatim}
%\begin{lstlisting}[language=reflex]
%ret = #audit.writeAuditEntryData(logURI,category,level,message,data);
%\end{lstlisting}
The \verb+writeAuditEntryData+ call is similar to the \verb+writeAuditEntry+ call except that an extra "data"
parameter can be used to write custom fields into the log.



\rule{12cm}{2pt}
\section{GetRecentLogEntries}
\index{GetRecentLogEntries}
\label{Api:GetRecentLogEntries}
\begin{lstlisting}[style=nonumbers]
   List<AuditLogEntry> getRecentLogEntries (
           String    logURI
           int    count
   )
\end{lstlisting}
\begin{Verbatim}[formatcom=\color{Maroon}]
  Entitlement: /audit/admin
\end{Verbatim}
%\begin{lstlisting}[language=reflex]
%ret = #audit.getRecentLogEntries(logURI,count);
%\end{lstlisting}
The \verb+getRecentLogEntries+ retrieves log entries from an audit log in reverse time order, up to a maximum
count.



\rule{12cm}{2pt}
\section{GetEntriesSince}
\index{GetEntriesSince}
\label{Api:GetEntriesSince}
\begin{lstlisting}[style=nonumbers]
   List<AuditLogEntry> getEntriesSince (
           String    logURI
           AuditLogEntry    when
   )
\end{lstlisting}
\begin{Verbatim}[formatcom=\color{Maroon}]
  Entitlement: /audit/main
\end{Verbatim}
%\begin{lstlisting}[language=reflex]
%ret = #audit.getEntriesSince(logURI,when);
%\end{lstlisting}
The \verb+getEntriesSince+ call can be used in a "long polling" technique for receiving log updates. The
"when" parameter can be the latest entry already known about -- this call will return all entries since that
record was written. When this call returns a zero length array the client can be said to have caught up.



\rule{12cm}{2pt}
\section{GetRecentUserActivity}
\index{GetRecentUserActivity}
\label{Api:GetRecentUserActivity}
\begin{lstlisting}[style=nonumbers]
   List<AuditLogEntry> getRecentUserActivity (
           String    user
           int    count
   )
\end{lstlisting}
\begin{Verbatim}[formatcom=\color{Maroon}]
  Entitlement: /audit/admin
\end{Verbatim}
%\begin{lstlisting}[language=reflex]
%ret = #audit.getRecentUserActivity(user,count);
%\end{lstlisting}
The \verb+getRecentUserActivity+ call returns entries that are associated with a given user account.



\rule{12cm}{2pt}

\chapter{Bootstrap API}
\index{Bootstrap API}

The document API for \Rapture is often abbreviated to \emph{Doc}. The API is used
to manipulate the presence and the content of document repositories in \Rapture.

In the abstract a document repository in \Rapture is a key/value store with optional
enhancements. The key in \Rapture corresponds to a URI for the document and where the
context is not obvious the scheme of the uri is \verb+document://+. In all document
API calls this scheme may be omitted.

Document repositories in \Rapture are backed by concrete data storage systems. When
you define a repository in \Rapture you provide a configuration string that is used
by \Rapture to route your request to a low level driver that interacts with the
underlying system. The format of this configuration string will be described in
the API call for creating a repository.

Document repositories can also be versioned. When you update a document in a
versioned repository the previous history of that document is preserved. In fact you can
qualify the URI of a document with the @ symbol and a version number to retrieve
previous versions of a document. Omitting the @ symbol will always retrieve the
latest version of a document.

Documents in repositories can also have metadata associated with them. \Rapture
automatically maintains some of this metadata - the time the document was created, the
user that created it. But a developer can use metadata update calls to add their
own attributes to documents in \Rapture.

The URI of a document in a repository implies a folder-like structure with the
forward slash delineating these folders. There are document API calls to treat a
document repository like a file system -- these are useful when constructing
browsable user interfaces to a repository.

\subsection{Methods}

\section{SetEmphemeralRepo}
\index{SetEmphemeralRepo}
\label{Api:SetEmphemeralRepo}
\begin{lstlisting}[style=nonumbers]
   void setEmphemeralRepo (
           String    config
   )
\end{lstlisting}
\begin{Verbatim}[formatcom=\color{Maroon}]
  Entitlement: /admin/bootstrap
\end{Verbatim}
%\begin{lstlisting}[language=reflex]
%ret = #bootstrap.setEmphemeralRepo(config);
%\end{lstlisting}
The \verb+setEphemeralRepo+ defines the document repository configuration used to store transient data
in \Rapture. Ideally this call should be made once and once only in a \Rapture environment. If you wish to
change this setting in the future you should call \verb+migrateEphemeralRepo+.



\rule{12cm}{2pt}
\section{SetConfigRepo}
\index{SetConfigRepo}
\label{Api:SetConfigRepo}
\begin{lstlisting}[style=nonumbers]
   void setConfigRepo (
           String    config
   )
\end{lstlisting}
\begin{Verbatim}[formatcom=\color{Maroon}]
  Entitlement: /admin/bootstrap
\end{Verbatim}
%\begin{lstlisting}[language=reflex]
%ret = #bootstrap.setConfigRepo(config);
%\end{lstlisting}
\input{bootstrap/setConfigRepo}


\rule{12cm}{2pt}
\section{SetSettingsRepo}
\index{SetSettingsRepo}
\label{Api:SetSettingsRepo}
\begin{lstlisting}[style=nonumbers]
   void setSettingsRepo (
           String    config
   )
\end{lstlisting}
\begin{Verbatim}[formatcom=\color{Maroon}]
  Entitlement: /void/bootstrap
\end{Verbatim}
%\begin{lstlisting}[language=reflex]
%ret = #bootstrap.setSettingsRepo(config);
%\end{lstlisting}
The \verb+setSettingsRepo+ defines the document repository configuration used to store user definitions
in \Rapture. Ideally this call should be made once and once only in a \Rapture environment. If you wish to
change this setting in the future you should call \verb+migrateSettingsRepo+.



\rule{12cm}{2pt}
\section{MigrateConfigRepo}
\index{MigrateConfigRepo}
\label{Api:MigrateConfigRepo}
\begin{lstlisting}[style=nonumbers]
   void migrateConfigRepo (
           String    newConfig
   )
\end{lstlisting}
\begin{Verbatim}[formatcom=\color{Maroon}]
  Entitlement: /admin/bootstrap
\end{Verbatim}
%\begin{lstlisting}[language=reflex]
%ret = #bootstrap.migrateConfigRepo(newConfig);
%\end{lstlisting}
The \verb+migrateConfigRepo+ call can be used to change the config repository settings. Ideally this will be
called in a single server environment. \Rapture will copy the repository data over into this new location and then
repoint to use this new repository.



\rule{12cm}{2pt}
\section{MigrateEphemeralRepo}
\index{MigrateEphemeralRepo}
\label{Api:MigrateEphemeralRepo}
\begin{lstlisting}[style=nonumbers]
   void migrateEphemeralRepo (
           String    newConfig
   )
\end{lstlisting}
\begin{Verbatim}[formatcom=\color{Maroon}]
  Entitlement: /admin/bootstrap
\end{Verbatim}
%\begin{lstlisting}[language=reflex]
%ret = #bootstrap.migrateEphemeralRepo(newConfig);
%\end{lstlisting}
The \verb+migrateEphemeralRepo+ call can be used to change the ephemeral repository settings. Ideally this will be
called in a single server environment. \Rapture will copy the repository data over into this new location and then
repoint to use this new repository.



\rule{12cm}{2pt}
\section{MigrateSettingsRepo}
\index{MigrateSettingsRepo}
\label{Api:MigrateSettingsRepo}
\begin{lstlisting}[style=nonumbers]
   void migrateSettingsRepo (
           String    newConfig
   )
\end{lstlisting}
\begin{Verbatim}[formatcom=\color{Maroon}]
  Entitlement: /admin/bootstrap
\end{Verbatim}
%\begin{lstlisting}[language=reflex]
%ret = #bootstrap.migrateSettingsRepo(newConfig);
%\end{lstlisting}
The \verb+migrateSettingsRepo+ call can be used to change the settings repository settings. Ideally this will be
called in a single server environment. \Rapture will copy the repository data over into this new location and then
repoint to use this new repository.



\rule{12cm}{2pt}
\section{GetConfigRepo}
\index{GetConfigRepo}
\label{Api:GetConfigRepo}
\begin{lstlisting}[style=nonumbers]
   String getConfigRepo (
   )
\end{lstlisting}
\begin{Verbatim}[formatcom=\color{Maroon}]
  Entitlement: /admin/main
\end{Verbatim}
%\begin{lstlisting}[language=reflex]
%ret = #bootstrap.getConfigRepo();
%\end{lstlisting}
The \verb+getConfigRepo+ returns the current configuration of the configuration repository.



\rule{12cm}{2pt}
\section{GetSettingsRepo}
\index{GetSettingsRepo}
\label{Api:GetSettingsRepo}
\begin{lstlisting}[style=nonumbers]
   String getSettingsRepo (
   )
\end{lstlisting}
\begin{Verbatim}[formatcom=\color{Maroon}]
  Entitlement: /admin/main
\end{Verbatim}
%\begin{lstlisting}[language=reflex]
%ret = #bootstrap.getSettingsRepo();
%\end{lstlisting}
The \verb+getSettingsRepo+ returns the current configuration of the settings repository.



\rule{12cm}{2pt}
\section{GetEphemeralRepo}
\index{GetEphemeralRepo}
\label{Api:GetEphemeralRepo}
\begin{lstlisting}[style=nonumbers]
   String getEphemeralRepo (
   )
\end{lstlisting}
\begin{Verbatim}[formatcom=\color{Maroon}]
  Entitlement: /admin/main
\end{Verbatim}
%\begin{lstlisting}[language=reflex]
%ret = #bootstrap.getEphemeralRepo();
%\end{lstlisting}
The \verb+getEphemeralRepo+ returns the current configuration of the ephemeral repository.



\rule{12cm}{2pt}
\section{RestartBootstrap}
\index{RestartBootstrap}
\label{Api:RestartBootstrap}
\begin{lstlisting}[style=nonumbers]
   void restartBootstrap (
   )
\end{lstlisting}
\begin{Verbatim}[formatcom=\color{Maroon}]
  Entitlement: /admin/bootstrap
\end{Verbatim}
%\begin{lstlisting}[language=reflex]
%ret = #bootstrap.restartBootstrap();
%\end{lstlisting}
The \verb+restartBootstrap+ method can be used to instruct a \Rapture environment to re-read its bootstrap
configuration, removing any cached data.



\rule{12cm}{2pt}
\section{AddScriptClass}
\index{AddScriptClass}
\label{Api:AddScriptClass}
\begin{lstlisting}[style=nonumbers]
   void addScriptClass (
           String    keyword
           String    className
   )
\end{lstlisting}
\begin{Verbatim}[formatcom=\color{Maroon}]
  Entitlement: /admin/bootstrap
\end{Verbatim}
%\begin{lstlisting}[language=reflex]
%ret = #bootstrap.addScriptClass(keyword,className);
%\end{lstlisting}
Script classes are injected into the startup \Reflex script environment for use by those scripts. When \Rapture
starts up a set of initialization \Reflex scripts will be run that will ensure that the environment is setup
correctly. Typically a \Rapture server will call this api call to inject classes into that container before those scripts are run.



\rule{12cm}{2pt}
\section{GetScriptClasses}
\index{GetScriptClasses}
\label{Api:GetScriptClasses}
\begin{lstlisting}[style=nonumbers]
   Map<String,String> getScriptClasses (
   )
\end{lstlisting}
\begin{Verbatim}[formatcom=\color{Maroon}]
  Entitlement: /admin/bootstrap
\end{Verbatim}
%\begin{lstlisting}[language=reflex]
%ret = #bootstrap.getScriptClasses();
%\end{lstlisting}
The \verb+getScriptClasses+ call returns the list of all script classes defined in the system.



\rule{12cm}{2pt}
\section{DeleteScriptClass}
\index{DeleteScriptClass}
\label{Api:DeleteScriptClass}
\begin{lstlisting}[style=nonumbers]
   boolean deleteScriptClass (
           String    keyword
   )
\end{lstlisting}
\begin{Verbatim}[formatcom=\color{Maroon}]
  Entitlement: /admin/bootstrap
\end{Verbatim}
%\begin{lstlisting}[language=reflex]
%ret = #bootstrap.deleteScriptClass(keyword);
%\end{lstlisting}
The \verb+deleteScriptClass+ call removes a script class previously added by \verb+addScriptClass+.



\rule{12cm}{2pt}

\chapter{Sys API}
\index{Sys API}

The document API for \Rapture is often abbreviated to \emph{Doc}. The API is used
to manipulate the presence and the content of document repositories in \Rapture.

In the abstract a document repository in \Rapture is a key/value store with optional
enhancements. The key in \Rapture corresponds to a URI for the document and where the
context is not obvious the scheme of the uri is \verb+document://+. In all document
API calls this scheme may be omitted.

Document repositories in \Rapture are backed by concrete data storage systems. When
you define a repository in \Rapture you provide a configuration string that is used
by \Rapture to route your request to a low level driver that interacts with the
underlying system. The format of this configuration string will be described in
the API call for creating a repository.

Document repositories can also be versioned. When you update a document in a
versioned repository the previous history of that document is preserved. In fact you can
qualify the URI of a document with the @ symbol and a version number to retrieve
previous versions of a document. Omitting the @ symbol will always retrieve the
latest version of a document.

Documents in repositories can also have metadata associated with them. \Rapture
automatically maintains some of this metadata - the time the document was created, the
user that created it. But a developer can use metadata update calls to add their
own attributes to documents in \Rapture.

The URI of a document in a repository implies a folder-like structure with the
forward slash delineating these folders. There are document API calls to treat a
document repository like a file system -- these are useful when constructing
browsable user interfaces to a repository.

\subsection{Methods}

\section{RetrieveSystemConfig}
\index{RetrieveSystemConfig}
\label{Api:RetrieveSystemConfig}
\begin{lstlisting}[style=nonumbers]
   String retrieveSystemConfig (
           String    area
           String    path
   )
\end{lstlisting}
\begin{Verbatim}[formatcom=\color{Maroon}]
  Entitlement: /admin/root
\end{Verbatim}
%\begin{lstlisting}[language=reflex]
%ret = #sys.retrieveSystemConfig(area,path);
%\end{lstlisting}
The \verb+retrieveSystemConfig+ can be used to retrieve the low level configuration stored by \Rapture at a given path.



\rule{12cm}{2pt}
\section{WriteSystemConfig}
\index{WriteSystemConfig}
\label{Api:WriteSystemConfig}
\begin{lstlisting}[style=nonumbers]
   String writeSystemConfig (
           String    area
           String    path
           String    content
   )
\end{lstlisting}
\begin{Verbatim}[formatcom=\color{Maroon}]
  Entitlement: /admin/root
\end{Verbatim}
%\begin{lstlisting}[language=reflex]
%ret = #sys.writeSystemConfig(area,path,content);
%\end{lstlisting}
The \verb+writeSystemConfig+ call updates the configuration at the path specified. This is a dangerous call to make -- the
correct approach is to use the most appropriate API call for removing an element in \Rapture (e.g. \verb+deleteDocRepo+ followed by \verb+createDocRepo+). Using this call will
not do any ancilliary tasks that are perhaps required by the entity in \Rapture -- it just updates the configuration. You would probably have to
restart all \Rapture instances for any changes to be noticed by other servers.



\rule{12cm}{2pt}
\section{RemoveSystemConfig}
\index{RemoveSystemConfig}
\label{Api:RemoveSystemConfig}
\begin{lstlisting}[style=nonumbers]
   void removeSystemConfig (
           String    area
           String    path
   )
\end{lstlisting}
\begin{Verbatim}[formatcom=\color{Maroon}]
  Entitlement: /admin/root
\end{Verbatim}
%\begin{lstlisting}[language=reflex]
%ret = #sys.removeSystemConfig(area,path);
%\end{lstlisting}
The \verb+removeSystemConfig+ call deletes from \Rapture the configuration at the path specified. This is a dangerous call to make -- the
correct approach is to use the most appropriate API call for removing an element in \Rapture (e.g. \verb+deleteDocRepo+). Using this call will
not do any ancilliary tasks that are perhaps required by the entity in \Rapture -- it just deletes the configuration. You would probably have to
restart all \Rapture instances for any changes to be noticed by other servers.



\rule{12cm}{2pt}
\section{GetSystemFolders}
\index{GetSystemFolders}
\label{Api:GetSystemFolders}
\begin{lstlisting}[style=nonumbers]
   List<RaptureFolderInfo> getSystemFolders (
           String    area
           String    path
   )
\end{lstlisting}
\begin{Verbatim}[formatcom=\color{Maroon}]
  Entitlement: /admin/root
\end{Verbatim}
%\begin{lstlisting}[language=reflex]
%ret = #sys.getSystemFolders(area,path);
%\end{lstlisting}
The \verb+getSystemFolders+ call is used to return the configuration documents and folders managed in the system respositories.

The area parameter reflects a general area withn the system, as enumerated by the \verb+getAllTopLevelRepos+ call. The path will start at
"/" and then subsequent calls can append onto that prefix the value of any folders returned by this call. The concepts returned by this call
can also represent configuration documents -- their contents can be retrieved using the \verb+retrieveSystemConfig+ call.

The \verb+RaptureFolderInfo+ structure returned by this call is described below:

\begin{table}[ht]
\begin{center}
\begin{tabular}{r l p{8cm}}
  Field & Type & Description \\
  \hline
  name & String & The name of this element. \\
  folder & Boolean & Whether the name refers to a document or a sub-folder \\
\end{tabular}
\end{center}
\end{table}



\rule{12cm}{2pt}
\section{GetAllTopLevelRepos}
\index{GetAllTopLevelRepos}
\label{Api:GetAllTopLevelRepos}
\begin{lstlisting}[style=nonumbers]
   List<String> getAllTopLevelRepos (
   )
\end{lstlisting}
\begin{Verbatim}[formatcom=\color{Maroon}]
  Entitlement: /repo/read
\end{Verbatim}
%\begin{lstlisting}[language=reflex]
%ret = #sys.getAllTopLevelRepos();
%\end{lstlisting}
The \verb+getAllTopLevelRepos+ call is used to retrieve the top level names of repositories managed by the Sys API.

Elements within the list of strings returned by this call can be used as the "area" parameter in other calls.



\rule{12cm}{2pt}
\section{ListByUriPrefix}
\index{ListByUriPrefix}
\label{Api:ListByUriPrefix}
\begin{lstlisting}[style=nonumbers]
   ChildrenTransferObject listByUriPrefix (
           String    raptureURI
           String    marker
           int    depth
           Long    maximum
           Long    millisUntilCacheExpiry
   )
\end{lstlisting}
\begin{Verbatim}[formatcom=\color{Maroon}]
  Entitlement: /repo/read
\end{Verbatim}
%\begin{lstlisting}[language=reflex]
%ret = #sys.listByUriPrefix(raptureURI,marker,depth,maximum,millisUntilCacheExpiry);
%\end{lstlisting}
\input{sys/listByUriPrefix}


\rule{12cm}{2pt}
\section{GetChildren}
\index{GetChildren}
\label{Api:GetChildren}
\begin{lstlisting}[style=nonumbers]
   ChildrenTransferObject getChildren (
           String    raptureURI
   )
\end{lstlisting}
\begin{Verbatim}[formatcom=\color{Maroon}]
  Entitlement: /repo/read
\end{Verbatim}
%\begin{lstlisting}[language=reflex]
%ret = #sys.getChildren(raptureURI);
%\end{lstlisting}
The \verb+getChildren+ call can be used by User Interfaces that wish to enumerate the audit logs
in a system.



\rule{12cm}{2pt}
\section{GetAllChildren}
\index{GetAllChildren}
\label{Api:GetAllChildren}
\begin{lstlisting}[style=nonumbers]
   ChildrenTransferObject getAllChildren (
           String    raptureURI
           String    marker
           Long    maximum
   )
\end{lstlisting}
\begin{Verbatim}[formatcom=\color{Maroon}]
  Entitlement: /repo/read
\end{Verbatim}
%\begin{lstlisting}[language=reflex]
%ret = #sys.getAllChildren(raptureURI,marker,maximum);
%\end{lstlisting}
\input{sys/getAllChildren}


\rule{12cm}{2pt}
\section{GetFolderInfo}
\index{GetFolderInfo}
\label{Api:GetFolderInfo}
\begin{lstlisting}[style=nonumbers]
   NodeEnum getFolderInfo (
           String    raptureURI
   )
\end{lstlisting}
\begin{Verbatim}[formatcom=\color{Maroon}]
  Entitlement: /repo/read
\end{Verbatim}
%\begin{lstlisting}[language=reflex]
%ret = #sys.getFolderInfo(raptureURI);
%\end{lstlisting}
This is an internal call that has little use outside of the \Rapture kernel.



\rule{12cm}{2pt}
\section{GetConnectionInfo}
\index{GetConnectionInfo}
\label{Api:GetConnectionInfo}
\begin{lstlisting}[style=nonumbers]
   Map<String,ConnectionInfo> getConnectionInfo (
           String    connectionType
   )
\end{lstlisting}
\begin{Verbatim}[formatcom=\color{Maroon}]
  Entitlement: /repo/read
\end{Verbatim}
%\begin{lstlisting}[language=reflex]
%ret = #sys.getConnectionInfo(connectionType);
%\end{lstlisting}
The \verb+getConnectionInfo+ retrieves the connection information associated with a connection area.



\rule{12cm}{2pt}
\section{PutConnectionInfo}
\index{PutConnectionInfo}
\label{Api:PutConnectionInfo}
\begin{lstlisting}[style=nonumbers]
   void putConnectionInfo (
           String    connectionType
           ConnectionInfo    connectionInfo
   )
\end{lstlisting}
\begin{Verbatim}[formatcom=\color{Maroon}]
  Entitlement: /repo/write
\end{Verbatim}
%\begin{lstlisting}[language=reflex]
%ret = #sys.putConnectionInfo(connectionType,connectionInfo);
%\end{lstlisting}
\input{sys/putConnectionInfo}


\rule{12cm}{2pt}
\section{SetConnectionInfo}
\index{SetConnectionInfo}
\label{Api:SetConnectionInfo}
\begin{lstlisting}[style=nonumbers]
   void setConnectionInfo (
           String    connectionType
           ConnectionInfo    connectionInfo
   )
\end{lstlisting}
\begin{Verbatim}[formatcom=\color{Maroon}]
  Entitlement: /repo/write
\end{Verbatim}
%\begin{lstlisting}[language=reflex]
%ret = #sys.setConnectionInfo(connectionType,connectionInfo);
%\end{lstlisting}
The \verb+setConnectionInfo+ call can be used to update how \Rapture connects to a lower level repository. The connection type
argument defines a connection domain -- for example "POSTGRES" defines how \Rapture connects to a Postgres database.

The \verb+ConnectionInfo+ parameter contains the fields necessary to construct a url that can be used to connect to that environment.



\rule{12cm}{2pt}

\chapter{Structured API}
\index{Structured API}

The document API for \Rapture is often abbreviated to \emph{Doc}. The API is used
to manipulate the presence and the content of document repositories in \Rapture.

In the abstract a document repository in \Rapture is a key/value store with optional
enhancements. The key in \Rapture corresponds to a URI for the document and where the
context is not obvious the scheme of the uri is \verb+document://+. In all document
API calls this scheme may be omitted.

Document repositories in \Rapture are backed by concrete data storage systems. When
you define a repository in \Rapture you provide a configuration string that is used
by \Rapture to route your request to a low level driver that interacts with the
underlying system. The format of this configuration string will be described in
the API call for creating a repository.

Document repositories can also be versioned. When you update a document in a
versioned repository the previous history of that document is preserved. In fact you can
qualify the URI of a document with the @ symbol and a version number to retrieve
previous versions of a document. Omitting the @ symbol will always retrieve the
latest version of a document.

Documents in repositories can also have metadata associated with them. \Rapture
automatically maintains some of this metadata - the time the document was created, the
user that created it. But a developer can use metadata update calls to add their
own attributes to documents in \Rapture.

The URI of a document in a repository implies a folder-like structure with the
forward slash delineating these folders. There are document API calls to treat a
document repository like a file system -- these are useful when constructing
browsable user interfaces to a repository.

\subsection{Methods}

\subsection{CreateStructuredRepo}
\index{CreateStructuredRepo}
\label{Api:CreateStructuredRepo}
\begin{verbatim}
   void createStructuredRepo (
           String    uri
           String    config
   )
\end{verbatim}
\begin{Verbatim}[fontsize=\small, formatcom=\color{Maroon}]
  Entitlement: /structured/write
\end{Verbatim}
%\begin{lstlisting}[language=reflex]
%ret = #structured.createStructuredRepo(uri,config);
%\end{lstlisting}
The \verb+createStructuredRepo+ defines a structured repository in \Rapture. The underlying connection
to a data environment (as needed by the sql drivers) is defined using the \verb+setConnectionInfo+ call. This call is used
to associated a structured repository uri with a connection and implementation.

The typical configuration string for a Postgres database is reproduced below:

\begin{Verbatim}
STRUCTURED {} USING POSTGRES { } ON sample
\end{Verbatim}

The general form of the configuration is:

\begin{Verbatim}
STRUCTURED { }
     USING [underlying implementation] { [ config ]}
     [ ON [ instance] ]
\end{Verbatim}

The \verb+instance+ part of the configuration is defined in the connection settings. As an example:

\begin{Verbatim}
  info = {};
  info.host = 'postgres';
  info.port = 5432;
  info.username = 'rapture';
  info.password = 'rapture';
  info.dbName = 'raptureSample';
  info.instanceName = 'sample';

  #sys.setConnectionInfo("sample", info);

  #structured.createStructuredRepo('//sample',
     'STRUCTURED {} USING POSTGRES ON sample');
\end{Verbatim}



\rule{12cm}{2pt}
\subsection{DeleteStructuredRepo}
\index{DeleteStructuredRepo}
\label{Api:DeleteStructuredRepo}
\begin{verbatim}
   void deleteStructuredRepo (
           String    uri
   )
\end{verbatim}
\begin{Verbatim}[fontsize=\small, formatcom=\color{Maroon}]
  Entitlement: /structured/write
\end{Verbatim}
%\begin{lstlisting}[language=reflex]
%ret = #structured.deleteStructuredRepo(uri);
%\end{lstlisting}
The \verb+deleteStructuredRepo+ call removes a reference to an underlying repository (that was previously
created using \verb+createStructuredRepo+).



\rule{12cm}{2pt}
\subsection{StructuredRepoExists}
\index{StructuredRepoExists}
\label{Api:StructuredRepoExists}
\begin{verbatim}
   boolean structuredRepoExists (
           String    uri
   )
\end{verbatim}
\begin{Verbatim}[fontsize=\small, formatcom=\color{Maroon}]
  Entitlement: /structured/read
\end{Verbatim}
%\begin{lstlisting}[language=reflex]
%ret = #structured.structuredRepoExists(uri);
%\end{lstlisting}
The \verb+structuredRepoExists+ call tests for the existing of a reference to a structured repository and is often
used in setup scripts.



\rule{12cm}{2pt}
\subsection{GetStructuredRepoConfig}
\index{GetStructuredRepoConfig}
\label{Api:GetStructuredRepoConfig}
\begin{verbatim}
   StructuredRepoConfig getStructuredRepoConfig (
           String    uri
   )
\end{verbatim}
\begin{Verbatim}[fontsize=\small, formatcom=\color{Maroon}]
  Entitlement: /structured/read
\end{Verbatim}
%\begin{lstlisting}[language=reflex]
%ret = #structured.getStructuredRepoConfig(uri);
%\end{lstlisting}
The \verb+getStructuredRepoConfig+ retrieves the definition of a structured repository.



\rule{12cm}{2pt}
\subsection{GetStructuredRepoConfigs}
\index{GetStructuredRepoConfigs}
\label{Api:GetStructuredRepoConfigs}
\begin{verbatim}
   List<StructuredRepoConfig> getStructuredRepoConfigs (
   )
\end{verbatim}
\begin{Verbatim}[fontsize=\small, formatcom=\color{Maroon}]
  Entitlement: /structured/read
\end{Verbatim}
%\begin{lstlisting}[language=reflex]
%ret = #structured.getStructuredRepoConfigs();
%\end{lstlisting}
The \verb+getStructuredRepoConfigs+ returns a list of all of the defined structured repository configurations in \Rapture.



\rule{12cm}{2pt}
\subsection{CreateTableUsingSql}
\index{CreateTableUsingSql}
\label{Api:CreateTableUsingSql}
\begin{verbatim}
   void createTableUsingSql (
           String    schema
           String    rawSql
   )
\end{verbatim}
\begin{Verbatim}[fontsize=\small, formatcom=\color{Maroon}]
  Entitlement: /structured/admin/$f(schema)
\end{Verbatim}
%\begin{lstlisting}[language=reflex]
%ret = #structured.createTableUsingSql(schema,rawSql);
%\end{lstlisting}
The \verb+createTableUsingSql+ call creates a table on a structured repository by defining that table using raw sql.



\rule{12cm}{2pt}
\subsection{CreateTable}
\index{CreateTable}
\label{Api:CreateTable}
\begin{verbatim}
   void createTable (
           String    tableUri
           Map<String,String>    columns
   )
\end{verbatim}
\begin{Verbatim}[fontsize=\small, formatcom=\color{Maroon}]
  Entitlement: /structured/admin/$f(tableUri)
\end{Verbatim}
%\begin{lstlisting}[language=reflex]
%ret = #structured.createTable(tableUri,columns);
%\end{lstlisting}
The \verb+createTable+ call is used to create a table on a database. The table is defined
by specifying a map of column names to types, and the table URI is hosted on the structured
database defined by the first part of the uri.

As an example, consider the following sample \Reflex code that creates a table on
the structured store referenced by the uri \verb+//test+. (This structured store reference
is created using createStructuredRepo)

\begin{Verbatim}
  columns = {};
  columns.id = 'int';
  columns.firstname = 'varchar(255)';
  columns.lastname = 'varchar(255)';
  columns.age = 'int';
  #structured.createTable('//test/table', columns);
\end{Verbatim}



\rule{12cm}{2pt}
\subsection{DropTable}
\index{DropTable}
\label{Api:DropTable}
\begin{verbatim}
   void dropTable (
           String    tableUri
   )
\end{verbatim}
\begin{Verbatim}[fontsize=\small, formatcom=\color{Maroon}]
  Entitlement: /structured/admin/$f(tableUri)
\end{Verbatim}
%\begin{lstlisting}[language=reflex]
%ret = #structured.dropTable(tableUri);
%\end{lstlisting}
The \verb+dropTable+ call is used to remove a table from a structured repository (database). The
format of the tableUri follows that in the other table calls -- it is the concatenation of a structured
repo reference (e.g. \verb+//test+) with a table reference (e.g. \verb+sampleTable+).



\rule{12cm}{2pt}
\subsection{TableExists}
\index{TableExists}
\label{Api:TableExists}
\begin{verbatim}
   boolean tableExists (
           String    tableUri
   )
\end{verbatim}
\begin{Verbatim}[fontsize=\small, formatcom=\color{Maroon}]
  Entitlement: /structured/read/$f(tableUri)
\end{Verbatim}
%\begin{lstlisting}[language=reflex]
%ret = #structured.tableExists(tableUri);
%\end{lstlisting}
The \verb+tableExists+ api call tests for the existence of a table on a structured repository. This call is often
used during a setup script -- giving the ability to create a table if it does not already exist.



\rule{12cm}{2pt}
\subsection{GetSchemas}
\index{GetSchemas}
\label{Api:GetSchemas}
\begin{verbatim}
   List<String> getSchemas (
   )
\end{verbatim}
\begin{Verbatim}[fontsize=\small, formatcom=\color{Maroon}]
  Entitlement: /structured/read
\end{Verbatim}
%\begin{lstlisting}[language=reflex]
%ret = #structured.getSchemas();
%\end{lstlisting}
The \verb+getSchemas+ call returns all of the structured store repository uris registered in a \Rapture environment.

Schemas are defined using the \verb+createStructuredRepo+ call and the \verb+getStructuredRepoConfig+ call can be used
to retrieve more information about a given schema.

If all of the configurations for all of the schemas is desired then the \verb+getStructuredRepoConfigs+ call is more appropriate.



\rule{12cm}{2pt}
\subsection{GetTables}
\index{GetTables}
\label{Api:GetTables}
\begin{verbatim}
   List<String> getTables (
           String    repoUri
   )
\end{verbatim}
\begin{Verbatim}[fontsize=\small, formatcom=\color{Maroon}]
  Entitlement: /structured/read
\end{Verbatim}
%\begin{lstlisting}[language=reflex]
%ret = #structured.getTables(repoUri);
%\end{lstlisting}
The \verb+getTables+ call returns all of the tables of a database referenced using the structured repository uri.

The return value contains a list of the table paths without the structured repo prefix.



\rule{12cm}{2pt}
\subsection{DescribeTable}
\index{DescribeTable}
\label{Api:DescribeTable}
\begin{verbatim}
   Map<String,String> describeTable (
           String    tableUri
   )
\end{verbatim}
\begin{Verbatim}[fontsize=\small, formatcom=\color{Maroon}]
  Entitlement: /structured/read/$f(tableUri)
\end{Verbatim}
%\begin{lstlisting}[language=reflex]
%ret = #structured.describeTable(tableUri);
%\end{lstlisting}
The \verb+describeTable+ call returns the column definitions for a table in a structured repository. The format
of the return map is the same as that used in the \verb+createTable+ call.



\rule{12cm}{2pt}
\subsection{AddTableColumns}
\index{AddTableColumns}
\label{Api:AddTableColumns}
\begin{verbatim}
   void addTableColumns (
           String    tableUri
           Map<String,String>    columns
   )
\end{verbatim}
\begin{Verbatim}[fontsize=\small, formatcom=\color{Maroon}]
  Entitlement: /structured/admin/$f(tableUri)
\end{Verbatim}
%\begin{lstlisting}[language=reflex]
%ret = #structured.addTableColumns(tableUri,columns);
%\end{lstlisting}
The \verb+addTableColumns+ is used to add a set of columns to a table in a structured database. The table is defined
by a tableURI.

The columns parameter defines the name and type of each column. The type field is what you would
normal use in a \verb+ALTER TABLE+ or \verb+CREATE TABLE+ type call. As an example:

\begin{Verbatim}
  columns = {};
  columns.firstname = 'varchar(255)';
  columns.lastname = 'varchar(255)';

  #structured.addTableColumns('//test/table', columns);
\end{Verbatim}



\rule{12cm}{2pt}
\subsection{DeleteTableColumns}
\index{DeleteTableColumns}
\label{Api:DeleteTableColumns}
\begin{verbatim}
   void deleteTableColumns (
           String    tableUri
           List<String>    columnNames
   )
\end{verbatim}
\begin{Verbatim}[fontsize=\small, formatcom=\color{Maroon}]
  Entitlement: /structured/admin/$f(tableUri)
\end{Verbatim}
%\begin{lstlisting}[language=reflex]
%ret = #structured.deleteTableColumns(tableUri,columnNames);
%\end{lstlisting}
The \verb+deleteTableColumns+ call drops columns from an existing table. The second parameter
specifies the names of the columns to be deleted.



\rule{12cm}{2pt}
\subsection{UpdateTableColumns}
\index{UpdateTableColumns}
\label{Api:UpdateTableColumns}
\begin{verbatim}
   void updateTableColumns (
           String    tableUri
           Map<String,String>    columns
   )
\end{verbatim}
\begin{Verbatim}[fontsize=\small, formatcom=\color{Maroon}]
  Entitlement: /structured/admin/$f(tableUri)
\end{Verbatim}
%\begin{lstlisting}[language=reflex]
%ret = #structured.updateTableColumns(tableUri,columns);
%\end{lstlisting}
The \verb+updateTableColumns+ call modifies the existing column set to those passed. This is usually used to correct the name
of a column.



\rule{12cm}{2pt}
\subsection{RenameTableColumns}
\index{RenameTableColumns}
\label{Api:RenameTableColumns}
\begin{verbatim}
   void renameTableColumns (
           String    tableUri
           Map<String,String>    columnNames
   )
\end{verbatim}
\begin{Verbatim}[fontsize=\small, formatcom=\color{Maroon}]
  Entitlement: /structured/admin/$f(tableUri)
\end{Verbatim}
%\begin{lstlisting}[language=reflex]
%ret = #structured.renameTableColumns(tableUri,columnNames);
%\end{lstlisting}
The \verb+renameTableColumns+ calls alters the name of a set of a columns. The key in the passed parameter is the
current column name, the value is the new column name.

As an example consider the following \Reflex script.

\begin{Verbatim}
  columns = {};
  columns.id = 'int';
  columns.firstname = 'varchar(255)';
  columns.lastname = 'varchar(255)';
  columns.age = 'int';

  #structured.createTable('//test/table', columns);

  // Now rename

  renameMap = {};
  renameMap.firstname = 'first';
  renameMap.lastname = 'last';
  #structured.renameTableColumns('//test/table', renameMap);

\end{Verbatim}



\rule{12cm}{2pt}
\subsection{CreateIndex}
\index{CreateIndex}
\label{Api:CreateIndex}
\begin{verbatim}
   void createIndex (
           String    tableUri
           String    indexName
           List<String>    columnNames
   )
\end{verbatim}
\begin{Verbatim}[fontsize=\small, formatcom=\color{Maroon}]
  Entitlement: /structured/admin/$f(tableUri)
\end{Verbatim}
%\begin{lstlisting}[language=reflex]
%ret = #structured.createIndex(tableUri,indexName,columnNames);
%\end{lstlisting}
The \verb+createIndex+ creates a database index on a table. The \verb+columnNames+ parameter indicates
the columns that this index should be based upon.



\rule{12cm}{2pt}
\subsection{DropIndex}
\index{DropIndex}
\label{Api:DropIndex}
\begin{verbatim}
   void dropIndex (
           String    tableUri
           String    indexName
   )
\end{verbatim}
\begin{Verbatim}[fontsize=\small, formatcom=\color{Maroon}]
  Entitlement: /structured/admin/$f(tableUri)
\end{Verbatim}
%\begin{lstlisting}[language=reflex]
%ret = #structured.dropIndex(tableUri,indexName);
%\end{lstlisting}
The \verb+dropIndex+ call reverses a call to \verb+createIndex+.



\rule{12cm}{2pt}
\subsection{GetIndexes}
\index{GetIndexes}
\label{Api:GetIndexes}
\begin{verbatim}
   List<TableIndex> getIndexes (
           String    tableUri
   )
\end{verbatim}
\begin{Verbatim}[fontsize=\small, formatcom=\color{Maroon}]
  Entitlement: /structured/read/$f(tableUri)
\end{Verbatim}
%\begin{lstlisting}[language=reflex]
%ret = #structured.getIndexes(tableUri);
%\end{lstlisting}
The \verb+getIndexes+ call returns all of the indices on a given table in a database (structured repository).

The \verb+TableIndex+ return type simply contains the name of the index and the list of columns it is an index on.



\rule{12cm}{2pt}
\subsection{SelectJoinedRows}
\index{SelectJoinedRows}
\label{Api:SelectJoinedRows}
\begin{verbatim}
   List<Map<String,Object>> selectJoinedRows (
           List<String>    tableUris
           List<String>    columnNames
           String    from
           String    where
           List<String>    order
           boolean    ascending
           int    limit
   )
\end{verbatim}
\begin{Verbatim}[fontsize=\small, formatcom=\color{Maroon}]
  Entitlement: /structured/read
\end{Verbatim}
%\begin{lstlisting}[language=reflex]
%ret = #structured.selectJoinedRows(tableUris,columnNames,from,where,order,ascending,limit);
%\end{lstlisting}
The \verb+selectJoinedRows+ call is used to perform a select with join across a number of tables in a
structured repository (database).

The parameters require some explanation, the following table and the example code serve this purpose.

\begin{table}[H]
\begin{center}
\begin{tabular}{r p{10cm}}
  Field & Purpose \\
  \hline
  tableUris & This is simply a list of the full tableUris used in the join. \\
  columnNames & This lists the column names that will be returned as part of this call. The column names should be prefixed with the table name. \\
  from & The from clause, which will include the JOIN clause. (See the example). \\
  where & A suitable WHERE clause if required. \\
  order & Which columns should be used to order the results. \\
  ascending & Whether the order should be ascending (true) or descending (false). \\
  limit & The maximum number of rows that will be returned. \\
\end{tabular}
\end{center}
\end{table}

\begin{Verbatim}
  tables = [ '//test/one', '//test/two'];
  columns = [ 'one.firstname', 'two.city'];
  from = 'test.one INNER JOIN test.two ON one.id=two.id';
  where = '';
  order = [ 'one.firstname' ];
  ascending = true;
  limit = 10;

  results = #structured.selectJoinedRows(tables,
    columns, from, where, order, ascending, limit);
  println(json(results));
\end{Verbatim}



\rule{12cm}{2pt}
\subsection{SelectUsingSql}
\index{SelectUsingSql}
\label{Api:SelectUsingSql}
\begin{verbatim}
   List<Map<String,Object>> selectUsingSql (
           String    schema
           String    rawSql
   )
\end{verbatim}
\begin{Verbatim}[fontsize=\small, formatcom=\color{Maroon}]
  Entitlement: /structured/read/$f(schema)
\end{Verbatim}
%\begin{lstlisting}[language=reflex]
%ret = #structured.selectUsingSql(schema,rawSql);
%\end{lstlisting}
The \verb+selectUsingSql+ call simply executes a select statement that is defined in raw sql against
the context of a structured repository.



\rule{12cm}{2pt}
\subsection{SelectRows}
\index{SelectRows}
\label{Api:SelectRows}
\begin{verbatim}
   List<Map<String,Object>> selectRows (
           String    tableUri
           List<String>    columnNames
           String    where
           List<String>    order
           boolean    ascending
           int    limit
   )
\end{verbatim}
\begin{Verbatim}[fontsize=\small, formatcom=\color{Maroon}]
  Entitlement: /structured/read/$f(tableUri)
\end{Verbatim}
%\begin{lstlisting}[language=reflex]
%ret = #structured.selectRows(tableUri,columnNames,where,order,ascending,limit);
%\end{lstlisting}
The \verb+selectRows+ call retrieves rows from a single database table given some criteria.

The parameters require some explanation, the following table and the example code serve this purpose.

\begin{table}[H]
\begin{center}
\begin{tabular}{r p{10cm}}
  Field & Purpose \\
  \hline
  table & This is simply a full tableUris used in the select. \\
  columnNames & This lists the column names that will be returned as part of this call. The column names should be prefixed with the table name. \\
  where & A suitable WHERE clause if required. \\
  order & Which columns should be used to order the results. \\
  ascending & Whether the order should be ascending (true) or descending (false). \\
  limit & The maximum number of rows that will be returned. \\
\end{tabular}
\end{center}
\end{table}

\begin{Verbatim}
  table = '//test/one';
  columns = [ 'firstname', 'lastname'];
  where = "lastname LIKE 'Mo%'";
  order = [ 'firstname' ];
  ascending = true;
  limit = 10;

  results = #structured.selectRows(table, columns, where, order, ascending, limit);
  println(json(results));
\end{Verbatim}



\rule{12cm}{2pt}
\subsection{InsertUsingSql}
\index{InsertUsingSql}
\label{Api:InsertUsingSql}
\begin{verbatim}
   void insertUsingSql (
           String    schema
           String    rawSql
   )
\end{verbatim}
\begin{Verbatim}[fontsize=\small, formatcom=\color{Maroon}]
  Entitlement: /structured/write/$f(schema)
\end{Verbatim}
%\begin{lstlisting}[language=reflex]
%ret = #structured.insertUsingSql(schema,rawSql);
%\end{lstlisting}
The \verb+insertUsingSql+ call is used to add new entries to a database given some raw sql.



\rule{12cm}{2pt}
\subsection{InsertRow}
\index{InsertRow}
\label{Api:InsertRow}
\begin{verbatim}
   void insertRow (
           String    tableUri
           Map<String,Object>    values
   )
\end{verbatim}
\begin{Verbatim}[fontsize=\small, formatcom=\color{Maroon}]
  Entitlement: /structured/write/$f(tableUri)
\end{Verbatim}
%\begin{lstlisting}[language=reflex]
%ret = #structured.insertRow(tableUri,values);
%\end{lstlisting}
The \verb+insertRow+ call adds a new row to a database table. The \verb+values+ parameter is a map
of column names to values for that column.



\rule{12cm}{2pt}
\subsection{InsertRows}
\index{InsertRows}
\label{Api:InsertRows}
\begin{verbatim}
   void insertRows (
           String    tableUri
           List<Map<String,Object>>    values
   )
\end{verbatim}
\begin{Verbatim}[fontsize=\small, formatcom=\color{Maroon}]
  Entitlement: /structured/write/$f(tableUri)
\end{Verbatim}
%\begin{lstlisting}[language=reflex]
%ret = #structured.insertRows(tableUri,values);
%\end{lstlisting}
The \verb+insertRows+ call is similar to the \verb+insertRow+ call except that the \verb+values+ parameter
is now a list of mappings from columns to values.



\rule{12cm}{2pt}
\subsection{DeleteUsingSql}
\index{DeleteUsingSql}
\label{Api:DeleteUsingSql}
\begin{verbatim}
   void deleteUsingSql (
           String    schema
           String    rawSql
   )
\end{verbatim}
\begin{Verbatim}[fontsize=\small, formatcom=\color{Maroon}]
  Entitlement: /structured/write/$f(schema)
\end{Verbatim}
%\begin{lstlisting}[language=reflex]
%ret = #structured.deleteUsingSql(schema,rawSql);
%\end{lstlisting}
The \verb+deleteUsingSql+ executes a delete call on a database using raw sql.



\rule{12cm}{2pt}
\subsection{DeleteRows}
\index{DeleteRows}
\label{Api:DeleteRows}
\begin{verbatim}
   void deleteRows (
           String    tableUri
           String    where
   )
\end{verbatim}
\begin{Verbatim}[fontsize=\small, formatcom=\color{Maroon}]
  Entitlement: /structured/write/$f(tableUri)
\end{Verbatim}
%\begin{lstlisting}[language=reflex]
%ret = #structured.deleteRows(tableUri,where);
%\end{lstlisting}
The \verb+deleteRows+ call removes rows from a database table that match a given where clause.



\rule{12cm}{2pt}
\subsection{UpdateUsingSql}
\index{UpdateUsingSql}
\label{Api:UpdateUsingSql}
\begin{verbatim}
   void updateUsingSql (
           String    schema
           String    rawSql
   )
\end{verbatim}
\begin{Verbatim}[fontsize=\small, formatcom=\color{Maroon}]
  Entitlement: /structured/write/$f(schema)
\end{Verbatim}
%\begin{lstlisting}[language=reflex]
%ret = #structured.updateUsingSql(schema,rawSql);
%\end{lstlisting}
The \verb+updateUsingSql+ call executes a SQL update call against a databse table as defined with native sql.



\rule{12cm}{2pt}
\subsection{UpdateRows}
\index{UpdateRows}
\label{Api:UpdateRows}
\begin{verbatim}
   void updateRows (
           String    tableUri
           Map<String,Object>    values
           String    where
   )
\end{verbatim}
\begin{Verbatim}[fontsize=\small, formatcom=\color{Maroon}]
  Entitlement: /structured/write/$f(tableUri)
\end{Verbatim}
%\begin{lstlisting}[language=reflex]
%ret = #structured.updateRows(tableUri,values,where);
%\end{lstlisting}
The \verb+updateRows+ call executes an update call against a table, applying the value set passed in the
second parameter (mapping column names to values) to entries that match the where clause.



\rule{12cm}{2pt}
\subsection{Begin}
\index{Begin}
\label{Api:Begin}
\begin{verbatim}
   boolean begin (
   )
\end{verbatim}
\begin{Verbatim}[fontsize=\small, formatcom=\color{Maroon}]
  Entitlement: /structured/write
\end{Verbatim}
%\begin{lstlisting}[language=reflex]
%ret = #structured.begin();
%\end{lstlisting}
The \verb+begin+ call starts a transaction on the environment specified. The transaction can be
ultimately managed using the \verb+commit+, \verb+rollback+ and \verb+abort+ calls.



\rule{12cm}{2pt}
\subsection{Commit}
\index{Commit}
\label{Api:Commit}
\begin{verbatim}
   boolean commit (
   )
\end{verbatim}
\begin{Verbatim}[fontsize=\small, formatcom=\color{Maroon}]
  Entitlement: /structured/write
\end{Verbatim}
%\begin{lstlisting}[language=reflex]
%ret = #structured.commit();
%\end{lstlisting}
The \verb+commit+ call commits a transaction previously started with \verb+begin+.



\rule{12cm}{2pt}
\subsection{Rollback}
\index{Rollback}
\label{Api:Rollback}
\begin{verbatim}
   boolean rollback (
   )
\end{verbatim}
\begin{Verbatim}[fontsize=\small, formatcom=\color{Maroon}]
  Entitlement: /structured/write
\end{Verbatim}
%\begin{lstlisting}[language=reflex]
%ret = #structured.rollback();
%\end{lstlisting}
The \verb+rollback+ API call is used to rollback an existing transaction. The transaction will usually
be initiated using the \verb+begin+ call. When the \verb+begin+ call is made it both returns a transaction id and
associates that transaction id with the calling context. The \verb+rollback+ call retrieves the transaction id from the calling context.

The \verb+abort+ call can also be used with the same effect if the transaction id is known to the caller.



\rule{12cm}{2pt}
\subsection{Abort}
\index{Abort}
\label{Api:Abort}
\begin{verbatim}
   boolean abort (
           String    transactionId
   )
\end{verbatim}
\begin{Verbatim}[fontsize=\small, formatcom=\color{Maroon}]
  Entitlement: /structured/admin
\end{Verbatim}
%\begin{lstlisting}[language=reflex]
%ret = #structured.abort(transactionId);
%\end{lstlisting}
The \verb+abort+ API call is used to rollback an existing transaction. The transaction will usually
be initiated using the \verb+begin+ call. When the \verb+begin+ call is made it both returns a transaction id and
associates that transaction id with the calling context. The \verb+rollback+ call is identical to the \verb+abort+ call except
that the \verb+rollback+ call retrieves the transaction id from the calling context.



\rule{12cm}{2pt}
\subsection{GetTransactions}
\index{GetTransactions}
\label{Api:GetTransactions}
\begin{verbatim}
   List<String> getTransactions (
   )
\end{verbatim}
\begin{Verbatim}[fontsize=\small, formatcom=\color{Maroon}]
  Entitlement: /structured/admin
\end{Verbatim}
%\begin{lstlisting}[language=reflex]
%ret = #structured.getTransactions();
%\end{lstlisting}
The \verb+getTransactions+ retrieves the transactions currently in play on a given \Rapture environment.



\rule{12cm}{2pt}
\subsection{GetDdl}
\index{GetDdl}
\label{Api:GetDdl}
\begin{verbatim}
   String getDdl (
           String    uri
           boolean    includeTableData
   )
\end{verbatim}
\begin{Verbatim}[fontsize=\small, formatcom=\color{Maroon}]
  Entitlement: /structured/read/$f(uri)
\end{Verbatim}
%\begin{lstlisting}[language=reflex]
%ret = #structured.getDdl(uri,includeTableData);
%\end{lstlisting}
The \verb+getDdl+ call is used to retrieve the definition of a structured repo.



\rule{12cm}{2pt}
\subsection{GetCursorUsingSql}
\index{GetCursorUsingSql}
\label{Api:GetCursorUsingSql}
\begin{verbatim}
   String getCursorUsingSql (
           String    schema
           String    rawSql
   )
\end{verbatim}
\begin{Verbatim}[fontsize=\small, formatcom=\color{Maroon}]
  Entitlement: /structured/read/$f(schema)
\end{Verbatim}
%\begin{lstlisting}[language=reflex]
%ret = #structured.getCursorUsingSql(schema,rawSql);
%\end{lstlisting}
The \verb+getCursorUsingSql+ executes a piece of sql against a database that will return a cursor object. Such an
object can then be used with the \verb+next+ and \verb+previous+ calls and closed with \verb+closeCursor+.



\rule{12cm}{2pt}
\subsection{GetCursor}
\index{GetCursor}
\label{Api:GetCursor}
\begin{verbatim}
   String getCursor (
           String    tableUri
           List<String>    columnNames
           String    where
           List<String>    order
           boolean    ascending
           int    limit
   )
\end{verbatim}
\begin{Verbatim}[fontsize=\small, formatcom=\color{Maroon}]
  Entitlement: /structured/read/$f(tableUri)
\end{Verbatim}
%\begin{lstlisting}[language=reflex]
%ret = #structured.getCursor(tableUri,columnNames,where,order,ascending,limit);
%\end{lstlisting}
The \verb+getCursor+ call executes a query against a table that will return a cursor that can be paged through. Such an
object can then be used with the \verb+next+ and \verb+previous+ calls and closed with \verb+closeCursor+.

\begin{table}[H]
\begin{center}
\begin{tabular}{r p{10cm}}
  Field & Purpose \\
  \hline
  table & This is simply a full tableUris used in the select. \\
  columnNames & This lists the column names that will be returned as part of this call. The column names should be prefixed with the table name. \\
  where & A suitable WHERE clause if required. \\
  order & Which columns should be used to order the results. \\
  ascending & Whether the order should be ascending (true) or descending (false). \\
  limit & The maximum number of rows that will be returned. \\
\end{tabular}
\end{center}
\end{table}

\begin{Verbatim}
  table = '//test/one';
  columns = [ 'firstname', 'lastname'];
  where = "lastname LIKE 'Mo%'";
  order = [ 'firstname' ];
  ascending = true;
  limit = 100;

  c = #structured.getCursor(table,
      columns, where, order, ascending, limit);
  v = #structure.next(table, c, 10);
  #structure.closeCursor(table, c);
  println(v);
\end{Verbatim}



\rule{12cm}{2pt}
\subsection{GetCursorForJoin}
\index{GetCursorForJoin}
\label{Api:GetCursorForJoin}
\begin{verbatim}
   String getCursorForJoin (
           List<String>    tableUris
           List<String>    columnNames
           String    from
           String    where
           List<String>    order
           boolean    ascending
           int    limit
   )
\end{verbatim}
\begin{Verbatim}[fontsize=\small, formatcom=\color{Maroon}]
  Entitlement: /structured/read
\end{Verbatim}
%\begin{lstlisting}[language=reflex]
%ret = #structured.getCursorForJoin(tableUris,columnNames,from,where,order,ascending,limit);
%\end{lstlisting}
\input{structured/getCursorForJoin}


\rule{12cm}{2pt}
\subsection{Next}
\index{Next}
\label{Api:Next}
\begin{verbatim}
   List<Map<String,Object>> next (
           String    tableUri
           String    cursorId
           int    count
   )
\end{verbatim}
\begin{Verbatim}[fontsize=\small, formatcom=\color{Maroon}]
  Entitlement: /structured/read/$f(tableUri)
\end{Verbatim}
%\begin{lstlisting}[language=reflex]
%ret = #structured.next(tableUri,cursorId,count);
%\end{lstlisting}
This call returns the next unique ID from the given provider.



\rule{12cm}{2pt}
\subsection{Previous}
\index{Previous}
\label{Api:Previous}
\begin{verbatim}
   List<Map<String,Object>> previous (
           String    tableUri
           String    cursorId
           int    count
   )
\end{verbatim}
\begin{Verbatim}[fontsize=\small, formatcom=\color{Maroon}]
  Entitlement: /structured/read/$f(tableUri)
\end{Verbatim}
%\begin{lstlisting}[language=reflex]
%ret = #structured.previous(tableUri,cursorId,count);
%\end{lstlisting}
The \verb+next+ call moves the cursor backward and returns the entries that it has
navigated over. The cursor id is returned by the \verb+getCursor+ calls.



\rule{12cm}{2pt}
\subsection{CloseCursor}
\index{CloseCursor}
\label{Api:CloseCursor}
\begin{verbatim}
   void closeCursor (
           String    tableUri
           String    cursorId
   )
\end{verbatim}
\begin{Verbatim}[fontsize=\small, formatcom=\color{Maroon}]
  Entitlement: /structured/read/$f(tableUri)
\end{Verbatim}
%\begin{lstlisting}[language=reflex]
%ret = #structured.closeCursor(tableUri,cursorId);
%\end{lstlisting}
The \verb+closeCursor+ call closes a cursor previously created using the \verb+getCursor+ calls.



\rule{12cm}{2pt}
\subsection{CreateProcedureCallUsingSql}
\index{CreateProcedureCallUsingSql}
\label{Api:CreateProcedureCallUsingSql}
\begin{verbatim}
   void createProcedureCallUsingSql (
           String    procUri
           String    rawSql
   )
\end{verbatim}
\begin{Verbatim}[fontsize=\small, formatcom=\color{Maroon}]
  Entitlement: /structured/admin/$f(procUri)
\end{Verbatim}
%\begin{lstlisting}[language=reflex]
%ret = #structured.createProcedureCallUsingSql(procUri,rawSql);
%\end{lstlisting}
The \verb+createProcedureCallUsingSql+ call creates a stored procedure on a structured repository.



\rule{12cm}{2pt}
\subsection{CallProcedure}
\index{CallProcedure}
\label{Api:CallProcedure}
\begin{verbatim}
   StoredProcedureResponse callProcedure (
           String    procUri
           StoredProcedureParams    params
   )
\end{verbatim}
\begin{Verbatim}[fontsize=\small, formatcom=\color{Maroon}]
  Entitlement: /structured/admin/$f(procUri)
\end{Verbatim}
%\begin{lstlisting}[language=reflex]
%ret = #structured.callProcedure(procUri,params);
%\end{lstlisting}
The \verb+callProcedure+ invokes a stored procedure with parameters and returns the result.

The second parameter is a complex type that defines the input parameters, output parameters and the \verb+inOut+ parameters as required by the
stored procedure call.

The return value is a response that indicates whether the call was successful and if it where the values of any out parameters to the call.



\rule{12cm}{2pt}
\subsection{DropProcedureUsingSql}
\index{DropProcedureUsingSql}
\label{Api:DropProcedureUsingSql}
\begin{verbatim}
   void dropProcedureUsingSql (
           String    procUri
           String    rawSql
   )
\end{verbatim}
\begin{Verbatim}[fontsize=\small, formatcom=\color{Maroon}]
  Entitlement: /structured/admin/$f(procUri)
\end{Verbatim}
%\begin{lstlisting}[language=reflex]
%ret = #structured.dropProcedureUsingSql(procUri,rawSql);
%\end{lstlisting}
The \verb+dropProcedureUsingSql+ call removes a stored procedure usually created using the associated createProcedure call.



\rule{12cm}{2pt}
\subsection{GetPrimaryKey}
\index{GetPrimaryKey}
\label{Api:GetPrimaryKey}
\begin{verbatim}
   String getPrimaryKey (
           String    tableUri
   )
\end{verbatim}
\begin{Verbatim}[fontsize=\small, formatcom=\color{Maroon}]
  Entitlement: /structured/read/$f(tableUri)
\end{Verbatim}
%\begin{lstlisting}[language=reflex]
%ret = #structured.getPrimaryKey(tableUri);
%\end{lstlisting}
The \verb+getPrimaryKey+ call returns the primary key for a table in a structured repository.



\rule{12cm}{2pt}
\subsection{GetForeignKeys}
\index{GetForeignKeys}
\label{Api:GetForeignKeys}
\begin{verbatim}
   List<ForeignKey> getForeignKeys (
           String    tableUri
   )
\end{verbatim}
\begin{Verbatim}[fontsize=\small, formatcom=\color{Maroon}]
  Entitlement: /structured/read/$f(tableUri)
\end{Verbatim}
%\begin{lstlisting}[language=reflex]
%ret = #structured.getForeignKeys(tableUri);
%\end{lstlisting}
The \verb+getForeignKeys+ call returns all of the foreign keys for a table in a structured repository.



\rule{12cm}{2pt}

\chapter{Schedule API}
\index{Schedule API}

The document API for \Rapture is often abbreviated to \emph{Doc}. The API is used
to manipulate the presence and the content of document repositories in \Rapture.

In the abstract a document repository in \Rapture is a key/value store with optional
enhancements. The key in \Rapture corresponds to a URI for the document and where the
context is not obvious the scheme of the uri is \verb+document://+. In all document
API calls this scheme may be omitted.

Document repositories in \Rapture are backed by concrete data storage systems. When
you define a repository in \Rapture you provide a configuration string that is used
by \Rapture to route your request to a low level driver that interacts with the
underlying system. The format of this configuration string will be described in
the API call for creating a repository.

Document repositories can also be versioned. When you update a document in a
versioned repository the previous history of that document is preserved. In fact you can
qualify the URI of a document with the @ symbol and a version number to retrieve
previous versions of a document. Omitting the @ symbol will always retrieve the
latest version of a document.

Documents in repositories can also have metadata associated with them. \Rapture
automatically maintains some of this metadata - the time the document was created, the
user that created it. But a developer can use metadata update calls to add their
own attributes to documents in \Rapture.

The URI of a document in a repository implies a folder-like structure with the
forward slash delineating these folders. There are document API calls to treat a
document repository like a file system -- these are useful when constructing
browsable user interfaces to a repository.

\subsection{Methods}

\subsection{CreateJob}
\index{CreateJob}
\label{Api:CreateJob}
\begin{Verbatim}
   RaptureJob createJob (
           String    jobURI
           String    description
           String    scriptURI
           String    cronExpression
           String    timeZone
           Map<String,String>    jobParams
           boolean    autoActivate
   )
\end{Verbatim}
\begin{Verbatim}[formatcom=\color{Maroon}]
  Entitlement: /admin/schedule
\end{Verbatim}
%\begin{lstlisting}[language=reflex]
%ret = #schedule.createJob(jobURI,description,scriptURI,cronExpression,timeZone,jobParams,autoActivate);
%\end{lstlisting}
The \verb+createJob+ call can be used to define a script job in \Rapture. A script job will execute
a \Reflex script according to the schedule provided in the cron expression and the time zone.

The parameters to the call are described in the table below:

\begin{table}[H]
\begin{center}
\begin{tabular}{r l l p{4cm}}
  Parameter & Type & Example & Description \\
  \hline
  jobURI & String & \verb+//test/one+ & The unique name of this job \\
  description & String & test job & A description of this job \\
  scriptURI & String & \verb+//runme+ & The script to run \\
  cronExpression & String & \verb+ * * 3 10 * *+ & The cron expression that defines when this script should run \\
  timeZone & String & Americas/New York & The time zone of the job \\
  jobParams & Map & \verb+param -> one+ & The parameters that should be passed to the script \\
  autoActivate & bool & true & Whether the job will always run according to the schedule \\
\end{tabular}
\end{center}
\end{table}



\rule{12cm}{2pt}
\subsection{CreateWorkflowJob}
\index{CreateWorkflowJob}
\label{Api:CreateWorkflowJob}
\begin{Verbatim}
   RaptureJob createWorkflowJob (
           String    jobURI
           String    description
           String    workflowURI
           String    cronExpression
           String    timeZone
           Map<String,String>    jobParams
           boolean    autoActivate
           int    maxRuntimeMinutes
           String    appStatusNamePattern
   )
\end{Verbatim}
\begin{Verbatim}[formatcom=\color{Maroon}]
  Entitlement: /admin/schedule
\end{Verbatim}
%\begin{lstlisting}[language=reflex]
%ret = #schedule.createWorkflowJob(jobURI,description,workflowURI,cronExpression,timeZone,jobParams,autoActivate,maxRuntimeMinutes,appStatusNamePattern);
%\end{lstlisting}
The \verb+createWorkflowJob+ call can be used to define a workflow job in \Rapture. A workflow job will execute
a workflow according to the schedule provided in the cron expression and the time zone.

The parameters to the call are described in the table below:

\begin{table}[H]
\begin{center}
\begin{tabular}{r l p{4cm}}
  Parameter & Example & Description \\
  \hline
  jobURI &  \verb+//test/one+ & The unique name of this job \\
  description  & test job & A description of this job \\
  workflowURI  & \verb+//runme+ & The workflow to run \\
  cronExpression  & \verb+ * * 3 10 * *+ & The cron expression that defines when this script should run \\
  timeZone &  Americas/New York & The time zone of the job \\
  jobParams &  \verb+param -> one+ & The parameters that should be passed to the script \\
  autoActivate &  true & Whether the job will always run according to the schedule \\
  maxRuntimeMinutes &  10 & How long the job is expected to execute for. \\
  appStatusNamePattern & runJob & If the logs for this workflow are to be combined, what should be the unique name of this log.\\
\end{tabular}
\end{center}
\end{table}



\rule{12cm}{2pt}
\subsection{ActivateJob}
\index{ActivateJob}
\label{Api:ActivateJob}
\begin{Verbatim}
   void activateJob (
           String    jobURI
           Map<String,String>    extraParams
   )
\end{Verbatim}
\begin{Verbatim}[formatcom=\color{Maroon}]
  Entitlement: /admin/schedule
\end{Verbatim}
%\begin{lstlisting}[language=reflex]
%ret = #schedule.activateJob(jobURI,extraParams);
%\end{lstlisting}
The \verb+activateJob+ call is used to make an inactive job active. An active job can be scheduled for execution.

Jobs that were created with the \verb+autoActivate+ flag set will be activated automatically.



\rule{12cm}{2pt}
\subsection{DeactivateJob}
\index{DeactivateJob}
\label{Api:DeactivateJob}
\begin{Verbatim}
   void deactivateJob (
           String    jobURI
   )
\end{Verbatim}
\begin{Verbatim}[formatcom=\color{Maroon}]
  Entitlement: /admin/schedule
\end{Verbatim}
%\begin{lstlisting}[language=reflex]
%ret = #schedule.deactivateJob(jobURI);
%\end{lstlisting}
The \verb+deactivateJob+ call sets the active flag of a job to be inactive. Inactive jobs will not be
scheduled for execution.



\rule{12cm}{2pt}
\subsection{RetrieveJob}
\index{RetrieveJob}
\label{Api:RetrieveJob}
\begin{Verbatim}
   RaptureJob retrieveJob (
           String    jobURI
   )
\end{Verbatim}
\begin{Verbatim}[formatcom=\color{Maroon}]
  Entitlement: /admin/schedule
\end{Verbatim}
%\begin{lstlisting}[language=reflex]
%ret = #schedule.retrieveJob(jobURI);
%\end{lstlisting}
The \verb+retrieveJob+ call returns configuration information about the current job.



\rule{12cm}{2pt}
\subsection{RetrieveJobs}
\index{RetrieveJobs}
\label{Api:RetrieveJobs}
\begin{Verbatim}
   List<RaptureJob> retrieveJobs (
           String    uriPrefix
   )
\end{Verbatim}
\begin{Verbatim}[formatcom=\color{Maroon}]
  Entitlement: /admin/schedule
\end{Verbatim}
%\begin{lstlisting}[language=reflex]
%ret = #schedule.retrieveJobs(uriPrefix);
%\end{lstlisting}
The \verb+retrieveJobs+ call returns all of the jobs defined in the system.



\rule{12cm}{2pt}
\subsection{RunJobNow}
\index{RunJobNow}
\label{Api:RunJobNow}
\begin{Verbatim}
   void runJobNow (
           String    jobURI
           Map<String,String>    extraParams
   )
\end{Verbatim}
\begin{Verbatim}[formatcom=\color{Maroon}]
  Entitlement: /admin/schedule
\end{Verbatim}
%\begin{lstlisting}[language=reflex]
%ret = #schedule.runJobNow(jobURI,extraParams);
%\end{lstlisting}
The \verb+runJobNow+ call is used to force a job to be executed as soon as possible.



\rule{12cm}{2pt}
\subsection{ResetJob}
\index{ResetJob}
\label{Api:ResetJob}
\begin{Verbatim}
   void resetJob (
           String    jobURI
   )
\end{Verbatim}
\begin{Verbatim}[formatcom=\color{Maroon}]
  Entitlement: /admin/schedule
\end{Verbatim}
%\begin{lstlisting}[language=reflex]
%ret = #schedule.resetJob(jobURI);
%\end{lstlisting}
The \verb+resetJob+ call removes any pending job executions for a job and recomputes them based on
the activation status and configuration.



\rule{12cm}{2pt}
\subsection{RetrieveJobExec}
\index{RetrieveJobExec}
\label{Api:RetrieveJobExec}
\begin{Verbatim}
   RaptureJobExec retrieveJobExec (
           String    jobURI
           Long    execTime
   )
\end{Verbatim}
\begin{Verbatim}[formatcom=\color{Maroon}]
  Entitlement: /admin/schedule
\end{Verbatim}
%\begin{lstlisting}[language=reflex]
%ret = #schedule.retrieveJobExec(jobURI,execTime);
%\end{lstlisting}
The \verb+retrieveJobExec+ call returns information about the next execution of a job - job executions are
pre-computed when a job is made active so the system can have a view of upcoming tasks.



\rule{12cm}{2pt}
\subsection{DeleteJob}
\index{DeleteJob}
\label{Api:DeleteJob}
\begin{Verbatim}
   void deleteJob (
           String    jobURI
   )
\end{Verbatim}
\begin{Verbatim}[formatcom=\color{Maroon}]
  Entitlement: /admin/schedule
\end{Verbatim}
%\begin{lstlisting}[language=reflex]
%ret = #schedule.deleteJob(jobURI);
%\end{lstlisting}
The \verb+deleteJob+ call removes a job previously added with the \verb+createJob+ calls.



\rule{12cm}{2pt}
\subsection{GetJobs}
\index{GetJobs}
\label{Api:GetJobs}
\begin{Verbatim}
   List<String> getJobs (
   )
\end{Verbatim}
\begin{Verbatim}[formatcom=\color{Maroon}]
  Entitlement: /admin/schedule
\end{Verbatim}
%\begin{lstlisting}[language=reflex]
%ret = #schedule.getJobs();
%\end{lstlisting}
The \verb+getJobs+ call returns a list of the names of the jobs in the system.



\rule{12cm}{2pt}
\subsection{GetUpcomingJobs}
\index{GetUpcomingJobs}
\label{Api:GetUpcomingJobs}
\begin{Verbatim}
   List<RaptureJobExec> getUpcomingJobs (
   )
\end{Verbatim}
\begin{Verbatim}[formatcom=\color{Maroon}]
  Entitlement: /admin/schedule
\end{Verbatim}
%\begin{lstlisting}[language=reflex]
%ret = #schedule.getUpcomingJobs();
%\end{lstlisting}
The \verb+getUpcomingJobs+ call is used to retrieve the list if job executions that are next in line to be run.



\rule{12cm}{2pt}
\subsection{GetWorkflowExecsStatus}
\index{GetWorkflowExecsStatus}
\label{Api:GetWorkflowExecsStatus}
\begin{Verbatim}
   WorkflowExecsStatus getWorkflowExecsStatus (
   )
\end{Verbatim}
\begin{Verbatim}[formatcom=\color{Maroon}]
  Entitlement: /admin/schedule
\end{Verbatim}
%\begin{lstlisting}[language=reflex]
%ret = #schedule.getWorkflowExecsStatus();
%\end{lstlisting}
The \verb+getWorkflowExecsStatus+ is a helper function used by UIs to retrieve the status of all active workflow job executions.



\rule{12cm}{2pt}
\subsection{AckJobError}
\index{AckJobError}
\label{Api:AckJobError}
\begin{Verbatim}
   JobErrorAck ackJobError (
           String    jobURI
           Long    execTime
           String    jobErrorType
   )
\end{Verbatim}
\begin{Verbatim}[formatcom=\color{Maroon}]
  Entitlement: /admin/schedule
\end{Verbatim}
%\begin{lstlisting}[language=reflex]
%ret = #schedule.ackJobError(jobURI,execTime,jobErrorType);
%\end{lstlisting}
The \verb+ackJobError+ can be used by a UI to make a running job that has failed for some reason as acknowledged. It is
useful in a support environment.



\rule{12cm}{2pt}
\subsection{GetNextExec}
\index{GetNextExec}
\label{Api:GetNextExec}
\begin{Verbatim}
   RaptureJobExec getNextExec (
           String    jobURI
   )
\end{Verbatim}
\begin{Verbatim}[formatcom=\color{Maroon}]
  Entitlement: /admin/schedule
\end{Verbatim}
%\begin{lstlisting}[language=reflex]
%ret = #schedule.getNextExec(jobURI);
%\end{lstlisting}
The \verb+getNextExec+ call will return the next predicted run time for a given job.



\rule{12cm}{2pt}
\subsection{GetJobExecs}
\index{GetJobExecs}
\label{Api:GetJobExecs}
\begin{Verbatim}
   List<RaptureJobExec> getJobExecs (
           String    jobURI
           int    start
           int    count
           boolean    reversed
   )
\end{Verbatim}
\begin{Verbatim}[formatcom=\color{Maroon}]
  Entitlement: /admin/schedule
\end{Verbatim}
%\begin{lstlisting}[language=reflex]
%ret = #schedule.getJobExecs(jobURI,start,count,reversed);
%\end{lstlisting}
The \verb+getJobExecs+ returns the list of all pending job executions in the system.



\rule{12cm}{2pt}
\subsection{BatchGetJobExecs}
\index{BatchGetJobExecs}
\label{Api:BatchGetJobExecs}
\begin{Verbatim}
   List<RaptureJobExec> batchGetJobExecs (
           List<String>    jobURI
           int    start
           int    count
           boolean    reversed
   )
\end{Verbatim}
\begin{Verbatim}[formatcom=\color{Maroon}]
  Entitlement: /admin/schedule
\end{Verbatim}
%\begin{lstlisting}[language=reflex]
%ret = #schedule.batchGetJobExecs(jobURI,start,count,reversed);
%\end{lstlisting}
The \verb+batchGetJobExecs+ can be used to page through the list of upcoming job executions.



\rule{12cm}{2pt}
\subsection{IsJobReadyToRun}
\index{IsJobReadyToRun}
\label{Api:IsJobReadyToRun}
\begin{Verbatim}
   boolean isJobReadyToRun (
           String    toJobURI
   )
\end{Verbatim}
\begin{Verbatim}[formatcom=\color{Maroon}]
  Entitlement: /admin/schedule
\end{Verbatim}
%\begin{lstlisting}[language=reflex]
%ret = #schedule.isJobReadyToRun(toJobURI);
%\end{lstlisting}
The \verb+isJobReadyToRun+ is a call that determines whether a job has satisfied the conditions needed to execute.



\rule{12cm}{2pt}
\subsection{GetCurrentWeekTimeRecords}
\index{GetCurrentWeekTimeRecords}
\label{Api:GetCurrentWeekTimeRecords}
\begin{Verbatim}
   List<TimedEventRecord> getCurrentWeekTimeRecords (
           int    weekOffsetfromNow
   )
\end{Verbatim}
\begin{Verbatim}[formatcom=\color{Maroon}]
  Entitlement: /admin/schedule
\end{Verbatim}
%\begin{lstlisting}[language=reflex]
%ret = #schedule.getCurrentWeekTimeRecords(weekOffsetfromNow);
%\end{lstlisting}
The \verb+getCurrentWeekTimeRecords+ returns the jobs scheduled for the current week.



\rule{12cm}{2pt}
\subsection{GetCurrentDayJobs}
\index{GetCurrentDayJobs}
\label{Api:GetCurrentDayJobs}
\begin{Verbatim}
   List<TimedEventRecord> getCurrentDayJobs (
   )
\end{Verbatim}
\begin{Verbatim}[formatcom=\color{Maroon}]
  Entitlement: /admin/schedule
\end{Verbatim}
%\begin{lstlisting}[language=reflex]
%ret = #schedule.getCurrentDayJobs();
%\end{lstlisting}
The \verb+getCurrentDayJobs+ call returns the jobs scheduled for the current day.



\rule{12cm}{2pt}

\chapter{Jar API}
\index{Jar API}

The document API for \Rapture is often abbreviated to \emph{Doc}. The API is used
to manipulate the presence and the content of document repositories in \Rapture.

In the abstract a document repository in \Rapture is a key/value store with optional
enhancements. The key in \Rapture corresponds to a URI for the document and where the
context is not obvious the scheme of the uri is \verb+document://+. In all document
API calls this scheme may be omitted.

Document repositories in \Rapture are backed by concrete data storage systems. When
you define a repository in \Rapture you provide a configuration string that is used
by \Rapture to route your request to a low level driver that interacts with the
underlying system. The format of this configuration string will be described in
the API call for creating a repository.

Document repositories can also be versioned. When you update a document in a
versioned repository the previous history of that document is preserved. In fact you can
qualify the URI of a document with the @ symbol and a version number to retrieve
previous versions of a document. Omitting the @ symbol will always retrieve the
latest version of a document.

Documents in repositories can also have metadata associated with them. \Rapture
automatically maintains some of this metadata - the time the document was created, the
user that created it. But a developer can use metadata update calls to add their
own attributes to documents in \Rapture.

The URI of a document in a repository implies a folder-like structure with the
forward slash delineating these folders. There are document API calls to treat a
document repository like a file system -- these are useful when constructing
browsable user interfaces to a repository.

\subsection{Methods}

\subsection{JarExists}
\index{JarExists}
\label{Api:JarExists}
\begin{Verbatim}
   boolean jarExists (
           String    jarUri
   )
\end{Verbatim}
\begin{Verbatim}[formatcom=\color{Maroon}]
  Entitlement: /data/list/$f(jarUri)
\end{Verbatim}
%\begin{lstlisting}[language=reflex]
%ret = #jar.jarExists(jarUri);
%\end{lstlisting}
\input{jar/jarExists}


\rule{12cm}{2pt}
\subsection{PutJar}
\index{PutJar}
\label{Api:PutJar}
\begin{Verbatim}
   void putJar (
           String    jarUri
           byte[]    jarContent
   )
\end{Verbatim}
\begin{Verbatim}[formatcom=\color{Maroon}]
  Entitlement: /data/write/$f(jarUri)
\end{Verbatim}
%\begin{lstlisting}[language=reflex]
%ret = #jar.putJar(jarUri,jarContent);
%\end{lstlisting}
\input{jar/putJar}


\rule{12cm}{2pt}
\subsection{GetJar}
\index{GetJar}
\label{Api:GetJar}
\begin{Verbatim}
   BlobContainer getJar (
           String    jarUri
   )
\end{Verbatim}
\begin{Verbatim}[formatcom=\color{Maroon}]
  Entitlement: /data/read/$f(jarUri)
\end{Verbatim}
%\begin{lstlisting}[language=reflex]
%ret = #jar.getJar(jarUri);
%\end{lstlisting}
\input{jar/getJar}


\rule{12cm}{2pt}
\subsection{DeleteJar}
\index{DeleteJar}
\label{Api:DeleteJar}
\begin{Verbatim}
   void deleteJar (
           String    jarUri
   )
\end{Verbatim}
\begin{Verbatim}[formatcom=\color{Maroon}]
  Entitlement: /data/write/$f(jarUri)
\end{Verbatim}
%\begin{lstlisting}[language=reflex]
%ret = #jar.deleteJar(jarUri);
%\end{lstlisting}
\input{jar/deleteJar}


\rule{12cm}{2pt}
\subsection{GetJarSize}
\index{GetJarSize}
\label{Api:GetJarSize}
\begin{Verbatim}
   Long getJarSize (
           String    jarUri
   )
\end{Verbatim}
\begin{Verbatim}[formatcom=\color{Maroon}]
  Entitlement: /data/list/$f(jarUri)
\end{Verbatim}
%\begin{lstlisting}[language=reflex]
%ret = #jar.getJarSize(jarUri);
%\end{lstlisting}
\input{jar/getJarSize}


\rule{12cm}{2pt}
\subsection{GetJarMetaData}
\index{GetJarMetaData}
\label{Api:GetJarMetaData}
\begin{Verbatim}
   Map<String,String> getJarMetaData (
           String    jarUri
   )
\end{Verbatim}
\begin{Verbatim}[formatcom=\color{Maroon}]
  Entitlement: /data/list/$f(jarUri)
\end{Verbatim}
%\begin{lstlisting}[language=reflex]
%ret = #jar.getJarMetaData(jarUri);
%\end{lstlisting}
\input{jar/getJarMetaData}


\rule{12cm}{2pt}
\subsection{ListJarsByUriPrefix}
\index{ListJarsByUriPrefix}
\label{Api:ListJarsByUriPrefix}
\begin{Verbatim}
   Map<String,RaptureFolderInfo> listJarsByUriPrefix (
           String    uriPrefix
           int    depth
   )
\end{Verbatim}
\begin{Verbatim}[formatcom=\color{Maroon}]
  Entitlement: /data/list/$f(uriPrefix)
\end{Verbatim}
%\begin{lstlisting}[language=reflex]
%ret = #jar.listJarsByUriPrefix(uriPrefix,depth);
%\end{lstlisting}
\input{jar/listJarsByUriPrefix}


\rule{12cm}{2pt}
\subsection{JarIsEnabled}
\index{JarIsEnabled}
\label{Api:JarIsEnabled}
\begin{Verbatim}
   boolean jarIsEnabled (
           String    jarUri
   )
\end{Verbatim}
\begin{Verbatim}[formatcom=\color{Maroon}]
  Entitlement: /data/list/$f(jarUri)
\end{Verbatim}
%\begin{lstlisting}[language=reflex]
%ret = #jar.jarIsEnabled(jarUri);
%\end{lstlisting}
\input{jar/jarIsEnabled}


\rule{12cm}{2pt}
\subsection{EnableJar}
\index{EnableJar}
\label{Api:EnableJar}
\begin{Verbatim}
   void enableJar (
           String    jarUri
   )
\end{Verbatim}
\begin{Verbatim}[formatcom=\color{Maroon}]
  Entitlement: /admin/jar
\end{Verbatim}
%\begin{lstlisting}[language=reflex]
%ret = #jar.enableJar(jarUri);
%\end{lstlisting}
\input{jar/enableJar}


\rule{12cm}{2pt}
\subsection{DisableJar}
\index{DisableJar}
\label{Api:DisableJar}
\begin{Verbatim}
   void disableJar (
           String    jarUri
   )
\end{Verbatim}
\begin{Verbatim}[formatcom=\color{Maroon}]
  Entitlement: /admin/jar
\end{Verbatim}
%\begin{lstlisting}[language=reflex]
%ret = #jar.disableJar(jarUri);
%\end{lstlisting}
\input{jar/disableJar}


\rule{12cm}{2pt}

\section{Plugin API}

The document API for \Rapture is often abbreviated to \emph{Doc}. The API is used
to manipulate the presence and the content of document repositories in \Rapture.

In the abstract a document repository in \Rapture is a key/value store with optional
enhancements. The key in \Rapture corresponds to a URI for the document and where the
context is not obvious the scheme of the uri is \verb+document://+. In all document
API calls this scheme may be omitted.

Document repositories in \Rapture are backed by concrete data storage systems. When
you define a repository in \Rapture you provide a configuration string that is used
by \Rapture to route your request to a low level driver that interacts with the
underlying system. The format of this configuration string will be described in
the API call for creating a repository.

Document repositories can also be versioned. When you update a document in a
versioned repository the previous history of that document is preserved. In fact you can
qualify the URI of a document with the @ symbol and a version number to retrieve
previous versions of a document. Omitting the @ symbol will always retrieve the
latest version of a document.

Documents in repositories can also have metadata associated with them. \Rapture
automatically maintains some of this metadata - the time the document was created, the
user that created it. But a developer can use metadata update calls to add their
own attributes to documents in \Rapture.

The URI of a document in a repository implies a folder-like structure with the
forward slash delineating these folders. There are document API calls to treat a
document repository like a file system -- these are useful when constructing
browsable user interfaces to a repository.

\subsection{Methods}

\subsubsection{GetInstalledPlugins}
\label{Api:GetInstalledPlugins}
\begin{verbatim}
   List<PluginConfig> getInstalledPlugins (
   )
\end{verbatim}
\begin{lstlisting}[language=reflex]
// Reflex use
ret = #plugin.getInstalledPlugins();
\end{lstlisting}
The \verb+getInstalledPlugins+ call can be used by an operator console to list
all of the plugins installed in an environment.



\rule{15cm}{2pt}
\subsubsection{GetPluginManifest}
\label{Api:GetPluginManifest}
\begin{verbatim}
   PluginManifest getPluginManifest (
           String    manifestUri
   )
\end{verbatim}
\begin{lstlisting}[language=reflex]
// Reflex use
ret = #plugin.getPluginManifest(manifestUri);
\end{lstlisting}
The \verb+getPluginManifest+ returns all of the items in a plugin. This corresponds to the
files contained within the original plugin.



\rule{15cm}{2pt}
\subsubsection{RecordPlugin}
\label{Api:RecordPlugin}
\begin{verbatim}
   void recordPlugin (
           PluginConfig    plugin
   )
\end{verbatim}
\begin{lstlisting}[language=reflex]
// Reflex use
ret = #plugin.recordPlugin(plugin);
\end{lstlisting}
The \verb+recordPlugin+ call is used by the installer to record the fact that a plugin
has been installed.



\rule{15cm}{2pt}
\subsubsection{InstallPlugin}
\label{Api:InstallPlugin}
\begin{verbatim}
   void installPlugin (
           PluginManifest    manifest
           Map<String,PluginTransportItem>    payload
   )
\end{verbatim}
\begin{lstlisting}[language=reflex]
// Reflex use
ret = #plugin.installPlugin(manifest,payload);
\end{lstlisting}
The \verb+installPlugin+ call can be used (by an installer) to install a given
set of elements into \Rapture. The PluginTransportItems are formed from the original
files in the plugin itself.



\rule{15cm}{2pt}
\subsubsection{InstallPluginItem}
\label{Api:InstallPluginItem}
\begin{verbatim}
   void installPluginItem (
           String    pluginName
           PluginTransportItem    item
   )
\end{verbatim}
\begin{lstlisting}[language=reflex]
// Reflex use
ret = #plugin.installPluginItem(pluginName,item);
\end{lstlisting}
The \verb+installPluginItem+ is used by an installer to add one individual item to an environment.



\rule{15cm}{2pt}
\subsubsection{UninstallPlugin}
\label{Api:UninstallPlugin}
\begin{verbatim}
   void uninstallPlugin (
           String    name
   )
\end{verbatim}
\begin{lstlisting}[language=reflex]
// Reflex use
ret = #plugin.uninstallPlugin(name);
\end{lstlisting}
The \verb+uninstallPlugin+ call is used to remove the effects of installing a plugin, usually reversing
all of the changes made.



\rule{15cm}{2pt}
\subsubsection{UninstallPluginItem}
\label{Api:UninstallPluginItem}
\begin{verbatim}
   void uninstallPluginItem (
           PluginTransportItem    item
   )
\end{verbatim}
\begin{lstlisting}[language=reflex]
// Reflex use
ret = #plugin.uninstallPluginItem(item);
\end{lstlisting}
The \verb+uninstallPluginItem+ call is used to remove a single installed item from an environment.



\rule{15cm}{2pt}
\subsubsection{DeletePluginManifest}
\label{Api:DeletePluginManifest}
\begin{verbatim}
   void deletePluginManifest (
           String    manifestUri
   )
\end{verbatim}
\begin{lstlisting}[language=reflex]
// Reflex use
ret = #plugin.deletePluginManifest(manifestUri);
\end{lstlisting}
The \verb+deletePluginManifest+ call is used by an installer when uninstalling a plugin.



\rule{15cm}{2pt}
\subsubsection{GetPluginItem}
\label{Api:GetPluginItem}
\begin{verbatim}
   PluginTransportItem getPluginItem (
           String    uri
   )
\end{verbatim}
\begin{lstlisting}[language=reflex]
// Reflex use
ret = #plugin.getPluginItem(uri);
\end{lstlisting}
The \verb+getPluginItem+ can be used by a plugin creator to retrieve a transportable item associated with
a \Rapture uri. The creator can use this to create the appropriate file in a file system.



\rule{15cm}{2pt}
\subsubsection{VerifyPlugin}
\label{Api:VerifyPlugin}
\begin{verbatim}
   Map<String,String> verifyPlugin (
           String    plugin
   )
\end{verbatim}
\begin{lstlisting}[language=reflex]
// Reflex use
ret = #plugin.verifyPlugin(plugin);
\end{lstlisting}
The \verb+verifyPlugin+ call is used to verify that a plugin is consistent.



\rule{15cm}{2pt}
\subsubsection{CreateManifest}
\label{Api:CreateManifest}
\begin{verbatim}
   void createManifest (
           String    pluginName
   )
\end{verbatim}
\begin{lstlisting}[language=reflex]
// Reflex use
ret = #plugin.createManifest(pluginName);
\end{lstlisting}
The \verb+createManifest+ call is used by plugin generators when they are defining a plugin from content already in \Rapture.



\rule{15cm}{2pt}
\subsubsection{AddManifestItem}
\label{Api:AddManifestItem}
\begin{verbatim}
   void addManifestItem (
           String    pluginName
           String    uri
   )
\end{verbatim}
\begin{lstlisting}[language=reflex]
// Reflex use
ret = #plugin.addManifestItem(pluginName,uri);
\end{lstlisting}
The \verb+addManifestItem+ is used to add an item to a previously created manifest. Once a manifest is complete it can
be downloaded "in plugin form" for saving offline for future installation.



\rule{15cm}{2pt}
\subsubsection{AddManifestDataFolder}
\label{Api:AddManifestDataFolder}
\begin{verbatim}
   void addManifestDataFolder (
           String    pluginName
           String    uri
   )
\end{verbatim}
\begin{lstlisting}[language=reflex]
// Reflex use
ret = #plugin.addManifestDataFolder(pluginName,uri);
\end{lstlisting}
The \verb+addManifestDataFolder+ call is used to add the contents of a folder to a manifest.



\rule{15cm}{2pt}
\subsubsection{RemoveManifestDataFolder}
\label{Api:RemoveManifestDataFolder}
\begin{verbatim}
   void removeManifestDataFolder (
           String    pluginName
           String    uri
   )
\end{verbatim}
\begin{lstlisting}[language=reflex]
// Reflex use
ret = #plugin.removeManifestDataFolder(pluginName,uri);
\end{lstlisting}
The \verb+removeManifestDataFolder+ reverses the effect of the call \verb+addManifestDataFolder+.



\rule{15cm}{2pt}
\subsubsection{SetManifestVersion}
\label{Api:SetManifestVersion}
\begin{verbatim}
   void setManifestVersion (
           String    pluginName
           String    version
   )
\end{verbatim}
\begin{lstlisting}[language=reflex]
// Reflex use
ret = #plugin.setManifestVersion(pluginName,version);
\end{lstlisting}
The \verb+setManifestVersion+ is used by a plugin generator to set the version of a plugin.



\rule{15cm}{2pt}
\subsubsection{RemoveItemFromManifest}
\label{Api:RemoveItemFromManifest}
\begin{verbatim}
   void removeItemFromManifest (
           String    pluginName
           String    uri
   )
\end{verbatim}
\begin{lstlisting}[language=reflex]
// Reflex use
ret = #plugin.removeItemFromManifest(pluginName,uri);
\end{lstlisting}
The \verb+removeItemFromManifest+ is used to remove an item previously added with \verb+addManifestItem+ or \verb+addManifestDataFolder+.



\rule{15cm}{2pt}
\subsubsection{ExportPlugin}
\label{Api:ExportPlugin}
\begin{verbatim}
   String exportPlugin (
           String    pluginName
           String    path
   )
\end{verbatim}
\begin{lstlisting}[language=reflex]
// Reflex use
ret = #plugin.exportPlugin(pluginName,path);
\end{lstlisting}
The \verb+exportPlugin+ is used to export a plugin (given its manifest or plugin name) to the local file system.



\rule{15cm}{2pt}


%*******************************************************
% Disclaimer
%*******************************************************
\clearpage
\vspace*{10pt}

\begin{center} \textbf{DISCLAIMER} \end{center}

\textbf{Copyright}: Unless otherwise noted, text, images and layout of this publication are the exclusive property of Incapture Technologies LLC and/or its related, affiliated and subsidiary companies and may not be copied or distributed, in whole or in part, without the express written consent of Incapture Technologies LLC or its related and affiliated companies.

\begin{center} \copyright 2012-2016 Incapture Technologies LLC \end{center}

\begin{center}
\large
\hfill
\vfill
\color{Maroon}\small\spacedallcaps{\myCompanyFull} \\
\color{Black}\small{\myCompanyAddress} \\

\end{center}

\end{document}
