Plugins in \Rapture are installable items that can be released to a \Rapture environment in
a consistent way. There are three parts to the Plugin API.

The first part relates to retrieving information about what is actually installed, and what items were installed with each
plugin. These are used by operator consoles and are informational.

The second part relates to installing an actual plugin. These calls are typically used by a specific
client side application "PluginInstaller" or a self installing library that uses the "SelfPluginInstallerLib". Plugins
defined in this way are jar archives that have a very specific structure. Typically a developer lays out this structure
in the file system and then uses a build process to package this information up into a jar file which can then be used with
the PluginInstaller application.

\section{File layout of a plugin}

A sample layout of a plugin "on disk" is reproduced below:

\begin{Verbatim}
  -- src/main/resources
     - PLUGIN
        - plugin.txt
        - content
          - testRepo
             - .rdoc
             someContent.rdoc
          - myscripts
             - test.script
\end{Verbatim}

The mandatory aspects of a plugin are the concept of a PLUGIN folder in the resources within
which is a plugin.txt file. This file has a specific format, an example is reproduced below:

\begin{Verbatim}
  {
     "depends":{},
     "description":"Curtis Web Core",
     "plugin":"CurtisWebCore",
     "version":{
     		"major":1,
     		"minor":1,
     		"release":32,
     		"timestamp":99999999999999
     }
  }
\end{Verbatim}

Most of these fields are self describing - the "plugin" field is the unique reference (uri) of this
plugin and the version section defines the version \emph{of this plugin}. The description is for operators
and the depends field can contain a map of other plugins and their version numbers required.

The content folder contains information and definitions for content and configuration that this plugin defines. The
file name extension defines what the plugin installer will do for this item.

\begin{table}[H]
\begin{center}
\begin{tabular}{r l p{8cm}}
  Keyword & Underlying & Configuration \\
  \hline
  MONGODB & MongoDb & The prefix parameter defines the name of the collections used by this repository. To avoid
  conflict this is usually a function of the name of the \Rapture repository. \\
  CASSANDRA & Cassandra & The prefix parameter defines the name of the collections used by this repository. To avoid
  conflict this is usually a function of the name of the \Rapture repository. \\
  POSTGRES & PostgresSql &  The prefix parameter defines the name of the tables used by this repository. To avoid
  conflict this is usually a function of the name of the \Rapture repository. \\
\end{tabular}
\end{center}
\end{table}
