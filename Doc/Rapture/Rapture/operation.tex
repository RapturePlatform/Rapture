\chapter{Operations - Documents as Objects}

The operation api of \Rapture can be used to treat \Rapture documents as a kind
of object from a programming perspective. The idea is simple -- fields in a \Rapture
document can contain a \Reflex script or a reference to a \Reflex script - an this
script can be invoked on the document object through the api. Furthermore special keys
in the document can refer to alternative documents that define a base implementation or
an interface implementation, with the operation call picking the most specific reference
to a function upon execution.

This approach gives a flexible way of dereferencing or delegating the functions that
can be applied to classes of document in \Rapture. Some specific examples are produced below but
as a preview consider a \emph{transform} function that can be used to convert a \Rapture document
to a different form -- perhaps as part of a general ETL process. This transform function could be
defined in a general way for all document objects of a given type and then overridden as necessary for
specific documents from an unknown source. In this way the \emph{transform} workflow process can run
on any set of document types -- the specialization is now data driven within the documents
themselves.

\section{API styles}

The calls in the operation API have two different variants, combining to create four calls. The first variant is
whether the document that is being acted upon is being modified by the function -- if so the \emph{Save} calls can
be used to save back the document after the function completes. The second variant provides an alternative source
for the function/method declaration. If large sets of documents have the same implementation this saves writing that reference
into each document. The \emph{Alt} calls have an additional parameter that is the uri of such a specification document.

\section{Method location}

In all cases we provide the operation API with a document uri and a method name. The implementation of the call searches
for the function to execute using the following strategy.

\subsection{Search in document}
The most specific strategy is one where the method is defined on the document given. For example, given a requirement to
search for the method \emph{claim} on a document, the following documents would be cases where this strategy would
succeed:

\begin{lstlisting}
  {
    "strategy" : "ABC",
    "trader" : "Alan",
    "face" : 12234,
    "claim" : "script://myoperations/order/claim"
  }
\end{lstlisting}

\begin{lstlisting}
  {
    "strategy" : "ABC",
    "trader" : "Alan",
    "face" : 12234,
    "claim" : "this['trader'] = params['trader']; return this;"
  }
\end{lstlisting}

\subsection{Search in interface}
A \Rapture document (or the \emph{Alt} API calls) can specify a document that contains an interface definition. So for example given
the following documents:

\begin{lstlisting}
  {
    "strategy" : "ABC",
    "trader" : "Alan",
    "face" : 12234,
    "$interface" : "//interface/order"
  }
\end{lstlisting}

\begin{lstlisting}[caption={//interface/order}]
  {
    "claim" : "this['trader'] = params['trader']; return this;"
  }
\end{lstlisting}

the following API call would work:

\begin{lstlisting}
  #operation.invokeSave('//orders/ORD1', 'claim', { 'trader' : 'Alan'});
\end{lstlisting}

Furthermore, given a document like this:

\begin{lstlisting}
  {
    "strategy" : "ABC",
    "trader" : "Alan",
    "face" : 12234
  }
\end{lstlisting}

the following call could be used:

\begin{lstlisting}
  #operation.invokeSaveAlt('//orders/ORD1', 'claim', '//interface/order', { 'trader' : 'Alan'});
\end{lstlisting}

\subsection{Search in parent}
A \Rapture document can specify a parent document that can contain an interface definition. This document works in the same
way as the interface document declaration except that any reference in the parent document can also be followed.

\begin{lstlisting}
  {
    "strategy" : "ABC",
    "trader" : "Alan",
    "face" : 12234,
    "$parent" : "//order/prime"
  }
\end{lstlisting}

\begin{lstlisting}[caption={//order/prime}]
  {
    "claim" : "this['trader'] = params['trader']; return this;"
  }
\end{lstlisting}

the following API call would work:

\begin{lstlisting}
  #operation.invokeSave('//orders/ORD1', 'claim', { 'trader' : 'Alan'});
\end{lstlisting}

\subsection{Order of search}
The order of function search is the following : direct reference, search in parent (up the hierarchy), search in interface, search in alt interface. For each parent search the interface
is also searched in those specification documents.

\section{Usage example}
In this approach a document has many fields and a method is used to retrieve some combination of their values according to some logic.
\begin{lstlisting}
  {
    "sources" : [
        "SS" : 122.23,
        "BBG" : 122.25,
        "Internal" : 122.26
    ],
    "$interface" : '//logic/preferredPrice'
  }
\end{lstlisting}
\begin{lstlisting}[caption={//logic/preferredPrice}]
  {
    "preferredPrice" : "
        preferredOrder = [ 'Internal', 'BBG', 'SS'];
        for x in preferredOrder do
          if this.sources[x] != null do
            return this.sources[x];
          end
        end
        throw 'No preferred price';
      "
  }
\end{lstlisting}

In the above example calling the method \emph{preferredPrice} on any \emph{price} document in a \Rapture system could imply some kind of
price structure that is preferred. However if a price document or instrument was special in some way, an override could be defined within
the document itself:

\begin{lstlisting}
  {
    "sources" : [
    ],
    "computedPrice" : 122.23,
    "preferredPrice" : "return this.computedPrice;
  }
\end{lstlisting}

In this way the method \emph{preferredPrice} can be called on all price documents with the majority following a price preference approach. Special
prices will follow a different computational approach but can still be used in a general workflow or computation that relies on the \emph{preferredPrice} approach.
